\documentclass[10pt,letterpaper,twoside]{report}
\usepackage[utf8]{inputenc}
\usepackage{array}
\usepackage{geometry}
\usepackage{fullpage}
\usepackage{fontspec}
\usepackage{longtable}
\usepackage{ amssymb }
\RequirePackage{xcolor}
\usepackage{lastpage}
\usepackage{textcomp}
\setmainfont{KGMissKindergarten.ttf}[
BoldFont = KGMissKindergarten.ttf,
ItalicFont = KGMissKindergarten.ttf,
BoldItalicFont = KGMissKindergarten.ttf]
\renewcommand\labelitemi{*}
\setcounter{secnumdepth}{4}
\setcounter{tocdepth}{1}

\setcounter{secnumdepth}{3}
\renewcommand \thechapter {\arabic{chapter}}
\renewcommand \thesection {\arabic{section}}
\renewcommand \thesubsection {\thesection\alph{subsection}}
\renewcommand \thesubsubsection {\thesubsection -\roman{subsubsection}}

\definecolor{Bright}{HTML}{58318f}
\definecolor{Green}{HTML}{319866}
\definecolor{Black}{HTML}{111111}
\definecolor{LightGrey}{HTML}{515c50}
\colorlet{heading}{Green}
\colorlet{accent}{Bright}
\colorlet{emphasis}{Black}
\colorlet{body}{LightGrey}
\usepackage{sectsty}
\chapterfont{\color{accent}\MakeUppercase} 
\sectionfont{\color{heading}\MakeUppercase} 
\subsectionfont{\color{heading}} 
\usepackage{fancyhdr}
\pagestyle{fancy}                                                   %   
\fancyhf{}
\renewcommand{\headrulewidth}{0pt}
%\setlength\headheight{15pt}
%\fancyhead[L]{}
%\fancyhead[R]{\MakeUppercase{BrailleNote Apex Lessons}}
\fancyfoot[L]{\copyright 2020 Michael Ryan Hunsaker, M.Ed., Ph.D. - BrailleNote Apex Skills Lessons}
\fancyfoot[R]{Page \thepage of \pageref{LastPage}}

%%%%%%%%%%%%%%%%%%%%%%%%%%%%%%%%%%%%%%%%%%%%%%%%%%%%%%%%%%%%%%%%%%%%%%%%%%
\begin{document}
\tableofcontents
%%%%%%%%%%%%%%%%%%%%%%%%%%%%%%%%%%%%%%%%%%
%%%%%%%%%%%%%%%%%%%%%%%%%%%%%%%%%%%%%%%%%%
%                                        %
%              PART 1                    %
%                                        %
%%%%%%%%%%%%%%%%%%%%%%%%%%%%%%%%%%%%%%%%%%
%%%%%%%%%%%%%%%%%%%%%%%%%%%%%%%%%%%%%%%%%%
\part{BrailleNote Apex}
\chapter{``School Skills'' Lessons}
General Instructions
These lessons are structured more as a list of skills that should be explored in each lesson. This is so the teacher has the freedom to develop lessons that are specifically catered to the preferences of each student

%%%%%%%%%%%%%%%%%%%%%%%%%%%%%%%%%%%%%%%%%%
%%%%%%%%%%%%%%%%%%%%%%%%%%%%%%%%%%%%%%%%%%
%                                        %
%              Lesson 1                  %
%                                        %
%%%%%%%%%%%%%%%%%%%%%%%%%%%%%%%%%%%%%%%%%%
%%%%%%%%%%%%%%%%%%%%%%%%%%%%%%%%%%%%%%%%%%
\clearpage
\section{Create a New Document}

\subsection{Tasks}
\begin{enumerate}
	\item Create a New Document
	      \begin{enumerate}
		      \item Turn on the BrailleNote \dotfill {\textcolor{accent}{\MakeUppercase{\textbf{ on/off rocker switch }}}}
		      \item Enter Main Menu \dotfill {\textcolor{accent}{\MakeUppercase{\textbf{ space + dots123456 }}}}
		      \item Change Speech mode \dotfill {\textcolor{accent}{\MakeUppercase{\textbf{ Previous Thumb-key + space }}}}
		      \item From Main Menu, open the Word Processor \dotfill {\textcolor{accent}{\MakeUppercase{\textbf{ letter w  }}}}
		      \item Create a document \dotfill {\textcolor{accent}{\MakeUppercase{\textbf{ letter c }}}}
		      \item For list of folders \dotfill {\textcolor{accent}{\MakeUppercase{\textbf{ space  }}}}
		      \item Scroll items in list \dotfill {\textcolor{accent}{\MakeUppercase{\textbf{ space }}}}
		      \item Open folder \dotfill {\textcolor{accent}{\MakeUppercase{\textbf{ enter }}}}
		      \item Writing a name of the new document (can use contracted UEB)
		      \item Create document \dotfill {\textcolor{accent}{\MakeUppercase{\textbf{ enter }}}}
		      \item Delete last character written \dotfill {\textcolor{accent}{\MakeUppercase{\textbf{ backspace }}}}
		      \item Exit current program and return to menu \dotfill {\textcolor{accent}{\MakeUppercase{\textbf{ space + e }}}}
		      \item There is no need to save the document. This is automatically done for you.
		      \item Turn off the BrailleNote \dotfill {\textcolor{accent}{\MakeUppercase{\textbf{ rocker switch on left side  }}}}
		      \item Repeat the lesson to mastery
	      \end{enumerate}
	\item Using the Context Dependent Help Menu
	      \begin{itemize}
		      \item Review the commands available \dotfill {\textcolor{accent}{\MakeUppercase{\textbf{ space + h. }}}}
		      \item Move forward in list of options \dotfill {\textcolor{accent}{\MakeUppercase{\textbf{ space  }}}}
		      \item Move backward in list of options \dotfill {\textcolor{accent}{\MakeUppercase{\textbf{ backspace  }}}}
		      \item Select or open a submenu or option \dotfill {\textcolor{accent}{\MakeUppercase{\textbf{ enter }}}}
	      \end{itemize}
\end{enumerate}
\subsection{Quiz}
If the statement is true, write or say "true". Write or say "false" if the statement is false.
\begin{enumerate}
	\item Space + dots123456 is a quick way to return to Main Menu.
	\item From Main Menu, letter w will open the Word Processor.
	\item The General folder is on a BrailleNote when you get it.
	\item Space in a document to delete the last character written.
	\item The BrailleNote automatically saves documents for you.
\end{enumerate}

%%%%%%%%%%%%%%%%%%%%%%%%%%%%%%%%%%%%%%%%%%
%%%%%%%%%%%%%%%%%%%%%%%%%%%%%%%%%%%%%%%%%%
%                                        %
%              Lesson 2                  %
%                                        %
%%%%%%%%%%%%%%%%%%%%%%%%%%%%%%%%%%%%%%%%%%
%%%%%%%%%%%%%%%%%%%%%%%%%%%%%%%%%%%%%%%%%%


\clearpage
\section{Set and Check Time and Date}
There are many tools on the BrailleNote to help a student be organized at school or home. The BrailleNote can be used to check the time and date. This is done quietly and quickly with simple shortcut commands. Students and adults can increase their independence by leaving and arriving on time and meeting various time commitments.
\subsection{Tasks}
\begin{enumerate}
	\item Navigate to Date and Time
	      \begin{enumerate}
		      \item Begin at Main Menu.
		      \item Check the time \dotfill {\textcolor{accent}{\MakeUppercase{\textbf{ enter + t }}}}
		      \item Clear the time \dotfill {\textcolor{accent}{\MakeUppercase{\textbf{ Advance thumb-key }}}}
		      \item Check the date \dotfill {\textcolor{accent}{\MakeUppercase{\textbf{ enter + d.  }}}}
		      \item Clear the date display \dotfill {\textcolor{accent}{\MakeUppercase{\textbf{ advance thumb-key }}}}
	      \end{enumerate}
	\item Set Time
	      \begin{enumerate}
		      \item Open Utilities menu \dotfill {\textcolor{accent}{\MakeUppercase{\textbf{ letter u  }}}}
		      \item Open Date and Time Set \dotfill {\textcolor{accent}{\MakeUppercase{\textbf{ letter d }}}}
		      \item Toggle UK and USA date/time setting \dotfill {\textcolor{accent}{\MakeUppercase{\textbf{ Space + dots34 }}}}
		      \item When "Date format? USA" is displayed \dotfill {\textcolor{accent}{\MakeUppercase{\textbf{ enter }}}}
		      \item Toggle 12 hr and 24 hr time options \dotfill {\textcolor{accent}{\MakeUppercase{\textbf{ Space + dots34, enter }}}}
		      \item Toggle time zones \dotfill {\textcolor{accent}{\MakeUppercase{\textbf{ space+ dot34 , enter }}}}
		            \begin{enumerate}
			            \item Time zome options, \dotfill {\textcolor{accent}{\MakeUppercase{\textbf{ m=Mountain, p=Pacific, c=Central, e=Eastern }}}}
		            \end{enumerate}
		      \item "Is daylight saving time?" \dotfill {\textcolor{accent}{\MakeUppercase{\textbf{ y for yes or n for no }}}}
		      \item Write the time in 12 hour format as hours and minutes, a space, then a for a.m. or p for p.m.
		      \item To move to Date menu \dotfill {\textcolor{accent}{\MakeUppercase{\textbf{ enter }}}}
	      \end{enumerate}
	\item Set Date
	      \begin{enumerate}
		      \item Type the date as number sign month, number sign day, number sign year. For example, 09 12 2013 = dots3456,245,24 dots3456,1,12 dots3456,12,245,1,14.
		      \item To delete last character \dotfill {\textcolor{accent}{\MakeUppercase{\textbf{ backspace }}}}
		      \item To accept typed date \dotfill {\textcolor{accent}{\MakeUppercase{\textbf{ enter }}}}
		      \item "Week starts on which day?" Toggle \dotfill {\textcolor{accent}{\MakeUppercase{\textbf{ space + dots34 }}}}
		            \begin{enumerate}
			            \item Direct selection \dotfill {\textcolor{accent}{\MakeUppercase{\textbf{ s=Sunday, m=Monday }}}}
		            \end{enumerate}
		      \item For context dependent help \dotfill {\textcolor{accent}{\MakeUppercase{\textbf{ space + h }}}}
		      \item Accept settings and return to Utilities \dotfill {\textcolor{accent}{\MakeUppercase{\textbf{ enter }}}}
	      \end{enumerate}
	\item Repeat the keyboard commands above to check the time and date.
	      \begin{enumerate}
		      \item If they are not correct, try again.
		      \item It may be that the "day light saving" option was set incorrectly.
	      \end{enumerate}
\end{enumerate}
\subsection{Quiz:}
Write or say the letter of the best answer choice.
\small{
	\begin{enumerate}
		\item What is the command to check the time?
		      \begin{enumerate}
			      \item space + t
			      \item enter + t
			      \item backspace + t
			      \item previous thumb-key
		      \end{enumerate}
		\item What will space + h do when using the BrailleNote?
		      \begin{enumerate}
			      \item a hard reset
			      \item put the BrailleNote in one-hand mode
			      \item give help
			      \item half re-set
		      \end{enumerate}
		\item What company makes the BrailleNote?
		      \begin{enumerate}
			      \item HIMS-Inc
			      \item Hadley
			      \item Freedome Scientific
			      \item Human Ware
		      \end{enumerate}
		\item What is the command to toggle among choices?
		      \begin{enumerate}
			      \item enter + dots3 4
			      \item Space + dots34
			      \item space
			      \item enter
		      \end{enumerate}
		\item From Main Menu, what is the fastest way to open the Utilities Menu?
		      \begin{enumerate}
			      \item letter u
			      \item space + u
			      \item press enter
			      \item backspace
		      \end{enumerate}
		\item How can you toggle among the "speech modes"?
		      \begin{enumerate}
			      \item Space + dots34
			      \item enter + d
			      \item previous thumb-key with space
			      \item previous thumb-key with enter
		      \end{enumerate}
		\item How can you delete the last character written?
		      \begin{enumerate}
			      \item backspace
			      \item space
			      \item enter
			      \item previous thumb-key
		      \end{enumerate}
		\item What will space with a full cell do?
		      \begin{enumerate}
			      \item a re-set
			      \item get help
			      \item open utilities menu
			      \item return to Main Menu
		      \end{enumerate}
	\end{enumerate} }

%%%%%%%%%%%%%%%%%%%%%%%%%%%%%%%%%%%%%%%%%%
%%%%%%%%%%%%%%%%%%%%%%%%%%%%%%%%%%%%%%%%%%
%                                        %
%              Lesson 3                  %
%                                        %
%%%%%%%%%%%%%%%%%%%%%%%%%%%%%%%%%%%%%%%%%%
%%%%%%%%%%%%%%%%%%%%%%%%%%%%%%%%%%%%%%%%%%
\clearpage

\section{Create a Folder Two Ways}
The BrailleNote is a computer. With it, you can complete word processing tasks, use the planner to set and keep appointments, and send and receive e-mail messages. Just like on the computer, there is often more than one way to complete a task. This is true for creating folders. In this lesson, you will learn two ways of creating folders. After practicing both methods, you may decide which one is best for you.
\subsection{Tasks}
\begin{enumerate}
	\item Method \#1 From File Manager
	      \begin{enumerate}
		      \item Begin at Main Menu \dotfill {\textcolor{accent}{\MakeUppercase{\textbf{ space + dots123456 }}}}
		      \item Open File Manager
		            \begin{itemize}
			            \item Move down options to File Manager \dotfill {\textcolor{accent}{\MakeUppercase{\textbf{ space, enter }}}}
			            \item Direct selection of File Manager \dotfill {\textcolor{accent}{\MakeUppercase{\textbf{ letter f }}}}
		            \end{itemize}
		      \item Open Folder Manager
		            \begin{itemize}
			            \item Move down options to Folder Manager \dotfill {\textcolor{accent}{\MakeUppercase{\textbf{ space, enter }}}}
			            \item Direct selection of Folder Manager \dotfill {\textcolor{accent}{\MakeUppercase{\textbf{ letter f }}}}
		            \end{itemize}
		      \item Move down options to new folder \dotfill {\textcolor{accent}{\MakeUppercase{\textbf{ space, enter }}}}
		      \item "Create folder on which drive?" is displayed. "Flash disk" is displayed. The flash disk is the BrailleNote, added memory is "SD Card"
		            \begin{itemize}
			            \item Move down options to SD card or Flash Disk \dotfill {\textcolor{accent}{\MakeUppercase{\textbf{ space, enter }}}}
			            \item Select most recently used drive \dotfill {\textcolor{accent}{\MakeUppercase{\textbf{ enter  }}}}
		            \end{itemize}
		      \item BrailleNote displays, "new folder name?" Begin writing a name for the folder.
		      \item Accept folder name \dotfill {\textcolor{accent}{\MakeUppercase{\textbf{ enter }}}}
		      \item The folder is created and you are put back at the original Folder Manager Menu.
		      \item To create another new folder from here, repeat the process
		      \item To leave the Folder Manager
		            \begin{itemize}
			            \item Exit and return to last level \dotfill {\textcolor{accent}{\MakeUppercase{\textbf{ space + e }}}}
			            \item To return to the main menu \dotfill {\textcolor{accent}{\MakeUppercase{\textbf{ space + dots123456 }}}}
		            \end{itemize}
		      \item Practice as needed.
	      \end{enumerate}
	\item Method \#2 From Word Processor
	      \begin{enumerate}
		      \item Begin at the Main Menu \dotfill {\textcolor{accent}{\MakeUppercase{\textbf{ space + dots123456 }}}}
		      \item To open Word Processor \dotfill {\textcolor{accent}{\MakeUppercase{\textbf{ letter w  }}}}
		      \item Toggle options until "create a document" \dotfill {\textcolor{accent}{\MakeUppercase{\textbf{ space, enter }}}}
		      \item BrailleNote says, "folder name?" and the name of the folder you last had open is displayed
		      \item To create a new folder, write the folder name
		      \item To accept typed name and continue \dotfill {\textcolor{accent}{\MakeUppercase{\textbf{ enter }}}}
		      \item At Create a new folder?" prompt \dotfill {\textcolor{accent}{\MakeUppercase{\textbf{ letter y=yes }}}}
		      \item BrailleNote displays, "document to create?" You may create a document to put in your new folder, but don't have to. Your folder is created
		      \item To exit back to menu \dotfill {\textcolor{accent}{\MakeUppercase{\textbf{ Press space + e }}}}
		      \item Practice as time allows or until the skill is mastered
	      \end{enumerate}
	\item Control the Voice
	      \begin{enumerate}
		      \item Turn the volume up \dotfill {\textcolor{accent}{\MakeUppercase{\textbf{ enter + dot4 }}}}
		      \item Turn volume down \dotfill {\textcolor{accent}{\MakeUppercase{\textbf{ enter + dot1  }}}}
		      \item Increase speech rate \dotfill {\textcolor{accent}{\MakeUppercase{\textbf{ enter + dot6 }}}}
		      \item Decrease speech rate \dotfill {\textcolor{accent}{\MakeUppercase{\textbf{ enter + dot3 }}}}
		      \item Increase speech pitch \dotfill {\textcolor{accent}{\MakeUppercase{\textbf{ enter + dot5 }}}}
		      \item Lower speech pitch \dotfill {\textcolor{accent}{\MakeUppercase{\textbf{ enter + dot2  }}}}
	      \end{enumerate}
\end{enumerate}
\subsection{Quiz:}
Tell what the result of each command or letter press will be.
\begin{enumerate}
	\item previous thumb-key with space
	\item At Main Menu, letter f
	\item enter + dot4
	\item enter + d
	\item space + 123456
	\item From Main Menu, letter w
	\item enter + dot2
	\item space + e
	\item enter + dot1
	\item enter + dot5
\end{enumerate}


%%%%%%%%%%%%%%%%%%%%%%%%%%%%%%%%%%%%%%%%%%
%%%%%%%%%%%%%%%%%%%%%%%%%%%%%%%%%%%%%%%%%%
%                                        %
%              Lesson 4                  %
%                                        %
%%%%%%%%%%%%%%%%%%%%%%%%%%%%%%%%%%%%%%%%%%
%%%%%%%%%%%%%%%%%%%%%%%%%%%%%%%%%%%%%%%%%%

\clearpage
\section{Use Cursor Routing Keys to Edit}

Navigation within a document will be important when you wish to read and edit your work. You can quickly jump from the top to the end of your document. With simple commands, it is possible to navigate by character, word, line, and paragraph. 

\subsection{Tasks}
\begin{enumerate}
	\item Check Power Status
	      \begin{enumerate}
		      \item Go to Options Menu \dotfill {\textcolor{accent}{\MakeUppercase{\textbf{ space + o }}}}
		      \item To get readout of power level \dotfill {\textcolor{accent}{\MakeUppercase{\textbf{ letter p }}}}
	      \end{enumerate}
	\item Review Creating a New Document:
	      \begin{enumerate}
		      \item Turn on the BrailleNote \dotfill {\textcolor{accent}{\MakeUppercase{\textbf{ rocker on/off switch }}}}
		      \item Enter Main Menu \dotfill {\textcolor{accent}{\MakeUppercase{\textbf{ space + dots123456 }}}}
		      \item Change Speech mode \dotfill {\textcolor{accent}{\MakeUppercase{\textbf{ Previous Thumb-key + space }}}}
		      \item From Main Menu, open the Word Processor \dotfill {\textcolor{accent}{\MakeUppercase{\textbf{ letter w  }}}}
		      \item Create a document \dotfill {\textcolor{accent}{\MakeUppercase{\textbf{ letter c }}}}
		      \item For list of folders \dotfill {\textcolor{accent}{\MakeUppercase{\textbf{ space  }}}}
		      \item Scroll items in list \dotfill {\textcolor{accent}{\MakeUppercase{\textbf{ space }}}}
		      \item Open folder \dotfill {\textcolor{accent}{\MakeUppercase{\textbf{ enter }}}}
		      \item Writing a name of the new document (contracted UEB okay)
		      \item Create document \dotfill {\textcolor{accent}{\MakeUppercase{\textbf{ enter }}}}
		      \item If speech is on, BrailleNote will say, "top of document". You may begin writing in your new document.
	      \end{enumerate}
	\item Navigate in and Edit the Document:
	      \begin{enumerate}
		      \item Delete last character \dotfill {\textcolor{accent}{\MakeUppercase{\textbf{ backspace }}}}
		      \item To move around the document without speech \dotfill {\textcolor{accent}{\MakeUppercase{\textbf{ thumb-keys }}}}
		            \begin{enumerate}
			            \item Move forward one display \dotfill {\textcolor{accent}{\MakeUppercase{\textbf{ advance thumb-key }}}}
			            \item Move backward one display \dotfill {\textcolor{accent}{\MakeUppercase{\textbf{ back thumb-key }}}}
		            \end{enumerate}
		      \item Move to top of document \dotfill {\textcolor{accent}{\MakeUppercase{\textbf{ space + dots123 }}}}
		      \item Move to end of document \dotfill {\textcolor{accent}{\MakeUppercase{\textbf{ space + dots456 }}}}
		      \item Move cursor within display \dotfill {\textcolor{accent}{\MakeUppercase{\textbf{ cursor routing keys  }}}}
		      \item At the cursor, you insert text by writing.
		      \item Delete the character tot he left of the cursor \dotfill {\textcolor{accent}{\MakeUppercase{\textbf{ backspace  }}}}
	      \end{enumerate}
	\item Try these additional commands within your document:
	      \begin{enumerate}
		      \item Move forward by word \dotfill {\textcolor{accent}{\MakeUppercase{\textbf{ space + dot5 }}}}
		      \item Back by word \dotfill {\textcolor{accent}{\MakeUppercase{\textbf{ space + dot2 }}}}
		      \item Read current word \dotfill {\textcolor{accent}{\MakeUppercase{\textbf{ space + dots25 }}}}
		      \item Move forward by character \dotfill {\textcolor{accent}{\MakeUppercase{\textbf{ space + dot6 }}}}
		      \item Back by character \dotfill {\textcolor{accent}{\MakeUppercase{\textbf{ space + dot3 }}}}
		      \item Read current character \dotfill {\textcolor{accent}{\MakeUppercase{\textbf{ space + dots36 }}}}
		      \item Move by line \dotfill {\textcolor{accent}{\MakeUppercase{\textbf{ space + dot4 }}}}
		      \item Back by line \dotfill {\textcolor{accent}{\MakeUppercase{\textbf{ space + dot1 }}}}
		      \item Read current line \dotfill {\textcolor{accent}{\MakeUppercase{\textbf{ space + dots14 }}}}
	      \end{enumerate}
	\item Exit document \dotfill {\textcolor{accent}{\MakeUppercase{\textbf{ space + e }}}}
	\item The document will save automatically. You will be back in the keyword word processor menu.
	\item Complete another word processor task or return to Main Menu.
\end{enumerate}

\clearpage 
\subsection{Quiz}:
Answer each question.
\begin{enumerate}
	\item In a document, what does the Advance thumb-key do?
	\item How can you jump to the top of a document?
	\item What will space + dots2 5 do?
	\item In a document, what does backspace do?
	\item  What does space + dot4 do?
	\item How can you exit a document?
	\item From anywhere on the BrailleNote, how can you get help?
	\item What is the BrailleNote command to speak the time?
	\item What does space + dot3 do?
	\item What company makes the BrailleNote?
\end{enumerate}

%%%%%%%%%%%%%%%%%%%%%%%%%%%%%%%%%%%%%%%%%%
%%%%%%%%%%%%%%%%%%%%%%%%%%%%%%%%%%%%%%%%%%
%                                        %
%              Lesson 5                  %
%                                        %
%%%%%%%%%%%%%%%%%%%%%%%%%%%%%%%%%%%%%%%%%%
%%%%%%%%%%%%%%%%%%%%%%%%%%%%%%%%%%%%%%%%%%
\clearpage
\section{Rename and Delete from File Manager}
Learning to delete and rename folders and documents can help your students to remain organized. There are a couple methods that can be used to complete these tasks. Try them all out to see what works best for you.  In this lesson, you will be working from within the File Manager
\subsection{Tasks}
You should have several folders and documents on your BrailleNote that we can use for practice in this lesson.
\begin{enumerate}
	\item Delete a folder from within File Manager:
	      \begin{enumerate}
		      \item From Main Menu, open the File Manager \dotfill {\textcolor{accent}{\MakeUppercase{\textbf{ letter f }}}}
		      \item Open Folder Manager menu \dotfill {\textcolor{accent}{\MakeUppercase{\textbf{ letter f }}}}
		      \item In Folder Manager Menu, erase folder menu \dotfill {\textcolor{accent}{\MakeUppercase{\textbf{ letter e  }}}}
		      \item "Erase folder on which drive?" is displayed.
		            \begin{itemize}
			            \item To toggle options for SD card or Flash disk \dotfill {\textcolor{accent}{\MakeUppercase{\textbf{ space, enter }}}}
			            \item To select most recently used disk \dotfill {\textcolor{accent}{\MakeUppercase{\textbf{ enter }}}}
		            \end{itemize}
		      \item Toggle to folder to erase \dotfill {\textcolor{accent}{\MakeUppercase{\textbf{ space, enter }}}}
		      \item "Erase folder name and all contained files and folders?" \dotfill {\textcolor{accent}{\MakeUppercase{\textbf{ letter y for yes }}}}
		      \item "1 folder erased. \# files erased" is displayed. P
		      \item Return to Folder Manager Menu \dotfill {\textcolor{accent}{\MakeUppercase{\textbf{ advance thumb-key }}}}
	      \end{enumerate}
	\item Delete a File (Document) from the File Manager Menu:
	      \begin{enumerate}
		      \item From Main Menu, Open File Manager Menu \dotfill {\textcolor{accent}{\MakeUppercase{\textbf{  press f }}}}
		      \item Enter erase file submenu \dotfill {\textcolor{accent}{\MakeUppercase{\textbf{ letter e }}}}
		      \item "Erase file on which drive?" is displayed.
		            \begin{itemize}
			            \item To toggle options for SD card or Flash disk \dotfill {\textcolor{accent}{\MakeUppercase{\textbf{ space, enter }}}}
			            \item To select most recently used disk \dotfill {\textcolor{accent}{\MakeUppercase{\textbf{ enter }}}}
		            \end{itemize}
		      \item To find folder containing file to be erased \dotfill {\textcolor{accent}{\MakeUppercase{\textbf{ space, enter. }}}}
		      \item To find file to be erased \dotfill {\textcolor{accent}{\MakeUppercase{\textbf{ space, enter }}}}
		      \item "Erase <file name>. sure?" \dotfill {\textcolor{accent}{\MakeUppercase{\textbf{ letter y for yes }}}}
		      \item To return to the File Manager Menu \dotfill {\textcolor{accent}{\MakeUppercase{\textbf{ advance thumb-key }}}}
		      \item Repeat process as needed.
	      \end{enumerate}
	\item Rename a Folder from within File Manager Menu:
	      \begin{enumerate}
		      \item Open File Manager \dotfill {\textcolor{accent}{\MakeUppercase{\textbf{ letter f }}}}
		      \item Open Folder Manager \dotfill {\textcolor{accent}{\MakeUppercase{\textbf{ letter f }}}}
		      \item Open rename option \dotfill {\textcolor{accent}{\MakeUppercase{\textbf{ letter r  }}}}
		      \item Move down list to find folder to rename \dotfill {\textcolor{accent}{\MakeUppercase{\textbf{ space, enter }}}}
		      \item Write the new name for your folder
		      \item Accept typed information \dotfill {\textcolor{accent}{\MakeUppercase{\textbf{ enter. }}}}
		      \item You will be taken back to the Folder Manager Menu.
		      \item Exit Menu
		            \begin{itemize}
			            \item Go back up one menu level \dotfill {\textcolor{accent}{\MakeUppercase{\textbf{ space + e }}}}
			            \item Return to Main Menu \dotfill {\textcolor{accent}{\MakeUppercase{\textbf{ space + dots123456 }}}}
		            \end{itemize}
	      \end{enumerate}
	\item Rename a File from File Manager Menu:
	      \begin{enumerate}
		      \item From Main Menu, Open File Manager \dotfill {\textcolor{accent}{\MakeUppercase{\textbf{ letter f  }}}}
		      \item Open Rename file submenu \dotfill {\textcolor{accent}{\MakeUppercase{\textbf{ letter r  }}}}
		      \item "Rename file on drive?" is displayed. To rename a file on the BrailleNote,
		            \begin{itemize}
			            \item To toggle options for SD card or Flash disk \dotfill {\textcolor{accent}{\MakeUppercase{\textbf{ space, enter }}}}
			            \item To select most recently used disk \dotfill {\textcolor{accent}{\MakeUppercase{\textbf{ enter }}}}
		            \end{itemize}
		      \item Toggle down the list of folders to the one  that contains the file you want to rename \dotfill {\textcolor{accent}{\MakeUppercase{\textbf{ space, enter }}}}
		      \item Toggle down the list of files \dotfill {\textcolor{accent}{\MakeUppercase{\textbf{ space, enter}}}}
		      \item "New name for file " is displayed. Write the new name for your file.
		      \item TO accept typed name \dotfill {\textcolor{accent}{\MakeUppercase{\textbf{ enter. }}}}
		      \item "File Manager Menu" is displayed.
	      \end{enumerate}
	\item Hotkeys:
	      \begin{itemize}
		      \item Open File Manager from anywhere on the BrailleNote \dotfill {\textcolor{accent}{\MakeUppercase{\textbf{ enter + backspace + f }}}}
		      \item Open Word Processor from anywhere on the BrailleNote \dotfill {\textcolor{accent}{\MakeUppercase{\textbf{ enter + backspace + w }}}}
	      \end{itemize}
\end{enumerate}
\clearpage
\subsection{Quiz}:
If the statement is true, write or say true. Write or say false if the statement is false.
\begin{enumerate}
	\item Enter + backspace + w will open word processor.
	\item The advance thumb-key will move forward by word.
	\item Dots1 2 3 will jump to the top of a document.
	\item Space + dots1 4 will read the current line.
	\item Move forward by line with space + dot4.
	\item You can erase or rename folders and documents by going to the File Manager Menu.
	\item Move back by character by pressing space + dot3.
	\item Space + dot2 will move forward by word.
	\item Space + dot4 will make the pitch higher.
	\item Toggle among choices by using Space + dots34.
\end{enumerate}

%%%%%%%%%%%%%%%%%%%%%%%%%%%%%%%%%%%%%%%%%%
%%%%%%%%%%%%%%%%%%%%%%%%%%%%%%%%%%%%%%%%%%
%                                        %
%              Lesson 6                  %
%                                        %
%%%%%%%%%%%%%%%%%%%%%%%%%%%%%%%%%%%%%%%%%%
%%%%%%%%%%%%%%%%%%%%%%%%%%%%%%%%%%%%%%%%%%
\clearpage
\section{Rename and Delete from List}
In the last Lesson, you learned to delete folders and files from within the File Manager. You also learned to rename folders and files by going through the BrailleNote File Manager. These tasks can be done from within a list when in the BrailleNote Word Processor. After this lesson, you can decide which method is most efficient for you.
\subsection{Tasks}
\begin{enumerate}
	\item Rename a Folder or Document:
	      \begin{enumerate}
		      \item Begin at Main Menu. Open Word Processor \dotfill {\textcolor{accent}{\MakeUppercase{\textbf{ letter w  }}}}
		      \item Open a Document
		            \begin{itemize}
			            \item Toggle to find "Open a Document"  \dotfill {\textcolor{accent}{\MakeUppercase{\textbf{ space, enter }}}}
			            \item Directly select "Open a Document" \dotfill {\textcolor{accent}{\MakeUppercase{\textbf{ letter o }}}}
		            \end{itemize}
		      \item Find folder you want to rename \dotfill {\textcolor{accent}{\MakeUppercase{\textbf{ space  }}}}
		      \item Rename the folder in focus, press  \dotfill {\textcolor{accent}{\MakeUppercase{\textbf{ backspace + r }}}}
		      \item BrailleNote displays, "new name for "foldername".
		      \item Write the new name for the folder.
		      \item To accept the new name \dotfill {\textcolor{accent}{\MakeUppercase{\textbf{ enter.  }}}}
		      \item The new name for the folder will be displayed.
		      \item Practice as needed.
		      \item The process is the same for renaming a file.
		            \begin{itemize}
			            \item Focus on the document you want to rename from within the list.
			            \item Start rename process \dotfill {\textcolor{accent}{\MakeUppercase{\textbf{ backspace + r }}}}
		            \end{itemize}
	      \end{enumerate}
	\item Erase a Folder or Document:
	      \begin{enumerate}
		      \item Begin at Main Menu. Open Word Processor \dotfill {\textcolor{accent}{\MakeUppercase{\textbf{ letter w  }}}}
		      \item Open a Document
		            \begin{itemize}
			            \item Toggle to find "Open a Document"  \dotfill {\textcolor{accent}{\MakeUppercase{\textbf{ space, enter }}}}
			            \item Directly select "Open a Document" \dotfill {\textcolor{accent}{\MakeUppercase{\textbf{ letter o }}}}
		            \end{itemize}
		      \item Find folder you want to erase \dotfill {\textcolor{accent}{\MakeUppercase{\textbf{ space  }}}}
		      \item This process can only be completed from within a list.
		      \item When the folder you want to delete has focus \dotfill {\textcolor{accent}{\MakeUppercase{\textbf{ backspace + dots14. }}}}
		      \item The process is the same for erasing a file.
		            \begin{itemize}
			            \item This process can only be completed from within a list.
			            \item Focus on the document you want to erase from within the list.
			            \item Delete the file \dotfill {\textcolor{accent}{\MakeUppercase{\textbf{ backspace + 14 }}}}
		            \end{itemize}
	      \end{enumerate}
	\item Practice as needed.
\end{enumerate}

\clearpage
\subsection{Quiz}:
Write the letter of the best answer choice.
\begin{enumerate}
	\item What is the command to rename a document?
	      \begin{enumerate}
		      \item enter + r
		      \item backspace + r
		      \item space + r
		      \item letter r
	      \end{enumerate}
	\item Space + dots1 2 3 4 5 6 will \_\_\_\_\_.
	      \begin{enumerate}
		      \item reset the BrailleNote.
		      \item back up by level.
		      \item return to Main Menu.
		      \item delete a folder.
	      \end{enumerate}
	\item Press \_\_\_\_\_ to toggle among the speech modes.
	      \begin{enumerate}
		      \item previous + next
		      \item space with a full cell
		      \item Previous with space
		      \item next with space
	      \end{enumerate}
	\item How can you check the time?
	      \begin{enumerate}
		      \item space + t
		      \item backspace + t
		      \item previous with space
		      \item backspace with t
	      \end{enumerate}
	\item Enter with dot1 will \_\_\_\_\_.
	      \begin{enumerate}
		      \item turn volume down
		      \item decrease rate
		      \item make the pitch lower
		      \item make the volume louder
	      \end{enumerate}
	\item To decrease the rate, press \_\_\_\_\_.
	      \begin{enumerate}
		      \item backspace + dot3
		      \item enter + dot6
		      \item enter + dot3
		      \item enter + dot2
	      \end{enumerate}
	\item Enter + dot2 will \_\_\_\_\_.
	      \begin{enumerate}
		      \item make the pitch higher
		      \item make the pitch lower
		      \item make the rate lower
		      \item make the volume lower
	      \end{enumerate}
	\item In a document, how can you move back by word?
	      \begin{enumerate}
		      \item space + dot3
		      \item space + dot1
		      \item space + dots25
		      \item space + dot2
	      \end{enumerate}
	\item Space with dot6 in a document will \_\_\_\_\_
	      \begin{enumerate}
		      \item read the current character.
		      \item read and move back by character
		      \item read forward by character
		      \item read forward by word
	      \end{enumerate}
	\item O with space, then letter p will \_\_\_\_\_
	      \begin{enumerate}
		      \item check the power status.
		      \item protect the BrailleNote.
		      \item move to previous word.
		      \item perform a reset.
	      \end{enumerate}
\end{enumerate}

%%%%%%%%%%%%%%%%%%%%%%%%%%%%%%%%%%%%%%%%%%
%%%%%%%%%%%%%%%%%%%%%%%%%%%%%%%%%%%%%%%%%%
%                                        %
%              Lesson 7                  %
%                                        %
%%%%%%%%%%%%%%%%%%%%%%%%%%%%%%%%%%%%%%%%%%
%%%%%%%%%%%%%%%%%%%%%%%%%%%%%%%%%%%%%%%%%%
\clearpage
\section{Calculator}
The BrailleNote has a calculator application. Though the calculator is scientific, beginners can use it for basic addition, subtraction, multiplication, and division.
\subsection{Tasks}
\begin{enumerate}
	\item Set UEB Code
	      \begin{enumerate}\item Start Scientific Calculator
		            \begin{itemize}
			            \item From the Main Menu \dotfill {\textcolor{accent}{\MakeUppercase{\textbf{ letter s }}}}
			            \item From anywhere \dotfill {\textcolor{accent}{\MakeUppercase{\textbf{ backspace + enter + s }}}}
		            \end{itemize}
		      \item Go to the Option Menu \dotfill {\textcolor{accent}{\MakeUppercase{\textbf{ space + o }}}}
		      \item Braille Settings \dotfill {\textcolor{accent}{\MakeUppercase{\textbf{ letter b }}}}
		      \item Select UEB as the braille code \dotfill {\textcolor{accent}{\MakeUppercase{\textbf{ letter e }}}}
	      \end{enumerate}
	\item Explore Calculator Functions
	      \begin{enumerate}
		      \item Open calculator from Main Menu \dotfill {\textcolor{accent}{\MakeUppercase{\textbf{ letter s }}}}
		      \item Write a simple math problem
		      \item For addition \dotfill {\textcolor{accent}{\MakeUppercase{\textbf{ + = dots5,235 }}}}
		      \item For subtraction \dotfill {\textcolor{accent}{\MakeUppercase{\textbf{ minus = dots5,36 }}}}
		      \item For multiplication \dotfill {\textcolor{accent}{\MakeUppercase{\textbf{ x = dots5,236 }}}}
		      \item For division \dotfill {\textcolor{accent}{\MakeUppercase{\textbf{ \textdiv = dots5,34 }}}}
		      \item For a decimal point \dotfill {\textcolor{accent}{\MakeUppercase{\textbf{ . = dots256. }}}}
		      \item To get the result \dotfill {\textcolor{accent}{\MakeUppercase{\textbf{ enter }}}}
		      \item To clear the calculation
		            \begin{itemize}
			            \item Clear the calculation \dotfill {\textcolor{accent}{\MakeUppercase{\textbf{ backspace + dots14 }}}}
			            \item  Clear the calculation\dotfill {\textcolor{accent}{\MakeUppercase{\textbf{ Space + dots356 }}}}
			            \item Clear the calculation\dotfill {\textcolor{accent}{\MakeUppercase{\textbf{ Scroll wheel down arrow }}}}
		            \end{itemize}
		      \item Practice with math problems at your level
		      \item Delete the last character written \dotfill {\textcolor{accent}{\MakeUppercase{\textbf{ backspace }}}}
		      \item To hear the last number or operator \dotfill {\textcolor{accent}{\MakeUppercase{\textbf{ space + dots25. }}}}
		      \item Alternate displaying the result or the whole calculation
		            \begin{itemize}
			            \item Clear the calculation \dotfill {\textcolor{accent}{\MakeUppercase{\textbf{ previous + next thumb-keys }}}}
			            \item Clear the calculation \dotfill {\textcolor{accent}{\MakeUppercase{\textbf{ scroll wheel left, right arrow }}}}
		            \end{itemize}
		      \item To hear more commands, use context sensitive help \dotfill {\textcolor{accent}{\MakeUppercase{\textbf{ press space + h }}}}
		      \item Practice as time.
	      \end{enumerate}
\end{enumerate}

\subsection{Quiz}:
Write or say the answer to each question.
\begin{enumerate}
	\item How can you change between displaying the result or the whole calculation?
	\item What dots are used for the decimal?
	\item What does space + dots356 do?
	\item What will space + dots25 do in the calculator?
	\item How can you delete the last key pressed?
	\item In the calculator, what are dots5,34 used for?
	\item What are the dots used for multiplication?
	\item What dots are used for addition?
	\item What are dots5,36 used for?
	\item How can you get help from anywhere on the BrailleNote?
\end{enumerate}
%%%%%%%%%%%%%%%%%%%%%%%%%%%%%%%%%%%%%%%%%%
%%%%%%%%%%%%%%%%%%%%%%%%%%%%%%%%%%%%%%%%%%
%                                        %
%              Lesson 8                  %
%                                        %
%%%%%%%%%%%%%%%%%%%%%%%%%%%%%%%%%%%%%%%%%%
%%%%%%%%%%%%%%%%%%%%%%%%%%%%%%%%%%%%%%%%%%
\clearpage
\section{Open and Read a Book}
The BrailleNote has an application used for reading books. The the Book Reader, you can open, navigate in, and read books. A book in braille ready format (.brf) can be read using the braille display or by using speech. Books in Daisy format are also accepted. Books can be transferred from an SD card to the BrailleNote for reading.
\subsection{Tasks}
\begin{enumerate}
	\item From Main Menu, open Book Reader \dotfill {\textcolor{accent}{\MakeUppercase{\textbf{ letter b }}}}
	\item Scroll down folders list to books folder ("My Books") \dotfill {\textcolor{accent}{\MakeUppercase{\textbf{ space, the enter }}}}
	\item Open a book \dotfill {\textcolor{accent}{\MakeUppercase{\textbf{ space, enter }}}}
	\item One way to read is by using the braille on the braille display.
	      \begin{itemize}
		      \item Advance braille 1 display width \dotfill {\textcolor{accent}{\MakeUppercase{\textbf{ advance thumb-key }}}}
		      \item Move braille back 1 display width \dotfill {\textcolor{accent}{\MakeUppercase{\textbf{ back thumb-key }}}}
	      \end{itemize}
	\item Automatic Reading Mode
	      \begin{itemize}
		      \item Start automatic display movement \dotfill {\textcolor{accent}{\MakeUppercase{\textbf{ space + dots23456 }}}}
		      \item Slow reading rate \dotfill {\textcolor{accent}{\MakeUppercase{\textbf{ previous thumb-key }}}}
		      \item Increase reading rate \dotfill {\textcolor{accent}{\MakeUppercase{\textbf{ next thumb-key }}}}
		      \item Stop automatic reading \dotfill {\textcolor{accent}{\MakeUppercase{\textbf{ previous + next }}}}
	      \end{itemize}
	\item If you would rather use speech, you may use the review commands that you used to review any document
	      \begin{itemize}
		      \item Forward reading \dotfill {\textcolor{accent}{\MakeUppercase{\textbf{ space + g }}}}
		      \item Stop reading \dotfill {\textcolor{accent}{\MakeUppercase{\textbf{ Backspace + space }}}}
	      \end{itemize}
	\item Practice reading the book as time.
\end{enumerate}

\clearpage
\subsection{Quiz}:
Match the tasks in 1 through 10 with the commands below.
Tasks:
\begin{enumerate}
	\item Stop reading after space + g was used.
	\item Start automatic reading with the braille display.
	\item Increase rate of automatic reading with braille display
	\item Top of file
	\item Slow the rate of automatic reading.
	\item Sentence back
	\item Sentence forward
	\item Bottom of file
	\item Stop automatic reading.
	\item Go forward reading
\end{enumerate}
Commands:
\begin{enumerate}
	\item space + g.
	\item space + dot4
	\item backspace + space
	\item space + dots123 or scroll wheel left
	\item Previous + next
	\item space + dots23456
	\item space + 456 or scroll wheel right
	\item next
	\item Previous
	\item space + dot1
\end{enumerate}
%%%%%%%%%%%%%%%%%%%%%%%%%%%%%%%%%%%%%%%%%%
%%%%%%%%%%%%%%%%%%%%%%%%%%%%%%%%%%%%%%%%%%
%                                        %
%              Lesson 9                  %
%                                        %
%%%%%%%%%%%%%%%%%%%%%%%%%%%%%%%%%%%%%%%%%%
%%%%%%%%%%%%%%%%%%%%%%%%%%%%%%%%%%%%%%%%%%
\clearpage
\section{Find Command}
When in a book, document, e-mail message, web page, or other such areas of the BrailleNote, you may often wish to find particular text. The BrailleNote has a find command used for this purpose. In this lesson, you will learn to use the find command.
\subsection{Tasks}
\begin{enumerate}
	\item Open a document
	\item Go to the top of the document \dotfill {\textcolor{accent}{\MakeUppercase{\textbf{ space + 123 }}}}
	\item To begin a search \dotfill {\textcolor{accent}{\MakeUppercase{\textbf{ space + f }}}}
	\item BrailleNote displays, "search forward or back".
	      \begin{itemize}
		      \item Search forward from the cursor \dotfill {\textcolor{accent}{\MakeUppercase{\textbf{ letter f }}}}
		      \item Search backward from the cursor \dotfill {\textcolor{accent}{\MakeUppercase{\textbf{ letter b }}}}
	      \end{itemize}
	\item Type in the word or text string that you want to find.
	      \begin{itemize}
		      \item Spelling must be exact (contracted and uncontracted UEB are different spellings)
		      \item Accept typed search string \dotfill {\textcolor{accent}{\MakeUppercase{\textbf{ enter }}}}
	      \end{itemize}
	\item If the BrailleNote found the text you are searching for, it will be displayed on the Braille display.
	\item If the BrailleNote cannot find the text you are looking for, "cannot find" will be displayed.
	\item To find the next occurrence of the text \dotfill {\textcolor{accent}{\MakeUppercase{\textbf{ space+ n (dots1345). }}}}
	\item Experiment with this command and practice as time.
\end{enumerate}

\clearpage
\subsection{Quiz}:
Write or say the answer to each question.
\begin{enumerate}
	\item What does space + n do?
	\item What is the command to toggle among the speech modes?
	\item How can you check the date?
	\item What command can often be used to toggle among choices?
	\item How can you turn the volume up?
	\item What does enter + dot6 do?
	\item How can you move forward by word?
	\item How can you read back by character?
	\item What does space + dot4 do?
	\item What is the command to jump to the top of the document?
\end{enumerate}
%%%%%%%%%%%%%%%%%%%%%%%%%%%%%%%%%%%%%%%%%%
%%%%%%%%%%%%%%%%%%%%%%%%%%%%%%%%%%%%%%%%%%
%                                        %
%              Lesson 10                 %
%                                        %
%%%%%%%%%%%%%%%%%%%%%%%%%%%%%%%%%%%%%%%%%%
%%%%%%%%%%%%%%%%%%%%%%%%%%%%%%%%%%%%%%%%%%
\clearpage 
\section{Cut, Copy, and Paste}
In many BrailleNote applications, it is helpful to use the Block Commands menu. By using this menu, you can complete a variety of tasks such as deleting a block of text, copying text to the clipboard, and pasting text in a new location. For example, when on a web page, you might find it helpful to copy a block of text onto the clipboard. That text may be placed in a word processor document or in an e-mail message. In this lesson, you will learn about the Block Commands Menu.
\subsection{Tasks}
\begin{enumerate}
	\item For this lesson you need at least two documents to practice with, so create or have more than one document that you can use for practice.
	\item "Copy" everything in one document onto the clipboard. When you copy text to the clipboard the text will remain in its original location. When you "cut or move" text to the clipboard, you are actually removing it from the original location and putting it onto the clipboard.
	      \begin{enumerate}
		      \item Go to the top of the document \dotfill {\textcolor{accent}{\MakeUppercase{\textbf{ space + 123 }}}}
		      \item Open the Block Commands menu \dotfill {\textcolor{accent}{\MakeUppercase{\textbf{ space + b }}}}
		      \item Toggle the menu options \dotfill {\textcolor{accent}{\MakeUppercase{\textbf{ space or backspace }}}}
		            \begin{itemize}
			            \item Append block to clipboard: add text to the end of what is already on the clipboard.
			            \item Copy block to clipboard: make a copy of the text within the block and place it on the clipboard. Text will remain in the original location as well as being placed on the clipboard.
			            \item Delete block: delete the text that is blocked.
			            \item Insert file: insert file at cursor position.
			            \item Move block to clipboard: remove text from original location and place on the clipboard. This is often referred to as "cutting" text to the clipboard.
			            \item Paste clipboard: take text from the clipboard and place it at the cursor position.
			            \item Read block: read the text within the block (top block marker and bottom block marker.
			            \item Store the block: store the block of text in a specified location.
			            \item Top marker insertion: set the top marker. This is where you want the block of text to begin. On a PC, this might be called "highlighted text".
			            \item Bottom marker insertion: this is the bottom of your block of text, where you want the block (highlighted text) to end.
			            \item Erase file and exit keyword: erase the file and exit word processing.
			            \item Zap: this will erase any block markers you have set.
			            \item Grade: change the braille grade of the block of text.
		            \end{itemize}
		      \item Selecting Blocks of Text
		            \begin{enumerate}
			            \item At this time we will set the top block marker. Your cursor is at the beginning of the document. We'll set the top block marker there.
			            \item In Block Commands Menu, Select "Top Marker Insertion"
			                  \begin{itemize}
				                  \item Toggle through list \dotfill {\textcolor{accent}{\MakeUppercase{\textbf{ space, enter  }}}}
				                  \item Direct selection \dotfill {\textcolor{accent}{\MakeUppercase{\textbf{ letter t }}}}
			                  \end{itemize}
			            \item To select the entire document. Jump to the end of the document \dotfill {\textcolor{accent}{\MakeUppercase{\textbf{ space + 456  }}}}
			            \item In Block Command Menu, Select "Bottom Marker Instruction"
			                  \begin{itemize}
				                  \item Toggle through list \dotfill {\textcolor{accent}{\MakeUppercase{\textbf{ space, enter  }}}}
				                  \item Direct selection \dotfill {\textcolor{accent}{\MakeUppercase{\textbf{ letter b }}}}
			                  \end{itemize}
			            \item BrailleNote keeps you in the block commands menu because it is expecting that you want to do something with the text you have selected.
			            \item "Copy block to clipboard"
			                  \begin{itemize}
				                  \item Toggle through list \dotfill {\textcolor{accent}{\MakeUppercase{\textbf{ space, enter  }}}}
				                  \item Direct selection \dotfill {\textcolor{accent}{\MakeUppercase{\textbf{ letter c }}}}
			                  \end{itemize}
		            \end{enumerate}
		      \item Paste the text into another document.
		            \begin{enumerate}
			            \item pen the document you previously had open \dotfill {\textcolor{accent}{\MakeUppercase{\textbf{ space + dots1256 }}}}
			            \item Find desired document in same folder  \dotfill {\textcolor{accent}{\MakeUppercase{\textbf{ space, enter }}}}
			            \item If in a different folder \dotfill {\textcolor{accent}{\MakeUppercase{\textbf{ backspace }}}}
			            \item Once the document is open, place the cursor to paste \dotfill {\textcolor{accent}{\MakeUppercase{\textbf{ cursor routing keys. }}}}
			            \item Block Commands Menu \dotfill {\textcolor{accent}{\MakeUppercase{\textbf{ space + b }}}}
			            \item "Paste clipboard"
			                  \begin{itemize}
				                  \item Toggle through list \dotfill {\textcolor{accent}{\MakeUppercase{\textbf{ space, enter  }}}}
				                  \item Direct selection \dotfill {\textcolor{accent}{\MakeUppercase{\textbf{ letter p }}}}
			                  \end{itemize}
			            \item Did the paste work? Read your document to find out.
		            \end{enumerate}
	      \end{enumerate}
\end{enumerate}
\clearpage
\subsection{Quiz}:
If the statement is true, write or say true. Write or say false if the statement is false.
\begin{enumerate}
	\item To append text to the clipboard means to add text to the end of the text that is currently on the clipboard.
	\item Space + b will open the Block Commands menu.
	\item In the Block Commands menu, using the initial letter of each item is a shortcut to carry out the action you want.
	\item It is possible to delete a block of text.
	\item Space + 2345 is the shortcut to open the previously opened document.
	\item To move text to the clipboard means the text will also be left in its original location.
	\item To check the time, press space + t.
	\item In the calculator, dots3 6 are the symbol for subtraction.
	\item Enter + dot4 will make the volume softer.
	\item Enter + dot3 will make the voice rate slower.
\end{enumerate}
%%%%%%%%%%%%%%%%%%%%%%%%%%%%%%%%%%%%%%%%%%
%%%%%%%%%%%%%%%%%%%%%%%%%%%%%%%%%%%%%%%%%%
%                                        %
%              Lesson 11                 %
%                                        %
%%%%%%%%%%%%%%%%%%%%%%%%%%%%%%%%%%%%%%%%%%
%%%%%%%%%%%%%%%%%%%%%%%%%%%%%%%%%%%%%%%%%%
\clearpage
\section{Spell Check}
As with most word processors, the BrailleNote word processor, Keyword, has a spelling checker. The BrailleNote will give you the choices of checking the entire document, one word, a paragraph or section, or from the cursor to the end of the document. You also have the option to look up a word. In this lesson, you or your student will learn to check an entire document for spelling errors.
\subsection{Tasks}
\begin{enumerate}
	\item Practice:
	      \begin{enumerate}
		      \item Create a new document.
		      \item Write several sentences. Misspell at least one word in each sentence.
		      \item Open the spelling checker \dotfill {\textcolor{accent}{\MakeUppercase{\textbf{ Press space + dots16. }}}}
		      \item "Spelling checker menu" is displayed by the BrailleNote.
		            \begin{itemize}
			            \item document check
			            \item look up word
			            \item word check
			            \item paragraph or section check
			            \item check from cursor to end of document
		            \end{itemize}
		      \item Focus on "document check"
		            \begin{itemize}
			            \item Toggle through list \dotfill {\textcolor{accent}{\MakeUppercase{\textbf{ space, enter  }}}}
			            \item Direct selection \dotfill {\textcolor{accent}{\MakeUppercase{\textbf{ letter d }}}}
		            \end{itemize}\dotfill {\textcolor{accent}{\MakeUppercase{\textbf{ letter d }}}}
		      \item BrailleNote displays the first misspelling and gives you the prompt "option?" so that you can decide what you want to do. You have the following choices. The options are listed here for you.
		            \begin{itemize}
			            \item Ignore all occurrences of the word \dotfill {\textcolor{accent}{\MakeUppercase{\textbf{ letter i }}}}
			            \item Skip only this occurrence of the word \dotfill {\textcolor{accent}{\MakeUppercase{\textbf{ Space + dot5 }}}}
			            \item Add word to the dictionary \dotfill {\textcolor{accent}{\MakeUppercase{\textbf{ letter a }}}}
			            \item Correct the word \dotfill {\textcolor{accent}{\MakeUppercase{\textbf{ letter c }}}}
			            \item Look up the word \dotfill {\textcolor{accent}{\MakeUppercase{\textbf{ letter l }}}}
			            \item Suggestions \dotfill {\textcolor{accent}{\MakeUppercase{\textbf{ letter s }}}}
		            \end{itemize}
		      \item An example of an error in my document is given here. Explore using the spelling checker in your own document. Use all options as appropriate.
	      \end{enumerate}
	\item Example:
	      \begin{enumerate}
		      \item "practice3. Option?" was displayed.
		      \item I pressed s to get a list of suggestions.
		      \item Use Space or backspace to move through the suggestions list.
		      \item The word practice was my first suggestion.
		      \item I press enter to replace practice3 with practice.
		      \item BrailleNote displays "correct all or first?".
		      \item I press letter a for all.
		      \item The next word is displayed.
	      \end{enumerate}
\end{enumerate}

\subsection{Quiz}:
Write or say the answer to each question.
\begin{enumerate}
	\item How can you open the document that was previously opened using the hotkey?
	\item How can you find the next occurrence of the last searched for string of text?
	\item What does space + dots1 6 do?
	\item What is the command used to jump to the bottom of a document?
	\item When in a list of folders or documents, what hotkey can be used to rename the folder or document?
	\item In a book or document, how can you go forward reading with the voice?
	\item What does space + f do?
	\item From Main Menu, what application will the letter f open?
	\item How can you turn the volume down?
	\item In the calculator, what is the multiplication symbol?
\end{enumerate}
%%%%%%%%%%%%%%%%%%%%%%%%%%%%%%%%%%%%%%%%%%
%%%%%%%%%%%%%%%%%%%%%%%%%%%%%%%%%%%%%%%%%%
%                                        %
%              Lesson 12                 %
%                                        %
%%%%%%%%%%%%%%%%%%%%%%%%%%%%%%%%%%%%%%%%%%
%%%%%%%%%%%%%%%%%%%%%%%%%%%%%%%%%%%%%%%%%%
\clearpage
\section{Set up WiFi}
The BrailleNote has a web browser application called Key Web. As with web browsers such as Internet Explorer and Firefox, Key Web can be used to read and interact with web pages. You can conduct a search using google, check your Facebook page, or download a book from Bookshare. There are many ways to use Key Web for completing Internet tasks.

The first job to be done is to set up the wireless connection. Students may help with this task so that setting up a wireless connection can be completed in other environments more independently. Students may want to use a wireless connection at home or in a public library.

\subsection{Tasks}
Below is a sample of how to connect the BrailleNote to wireless Internet that is an open connection with no password required. Setting up a wireless or LAN connection varies depending on the settings that are in place. For school connections, the IT staff may be able to help. It is often helpful to have HumanWare tech support on the phone while working with IT staff. Arrange times with all concerned ahead of time if possible. 
Sample of Setting Up a Wireless Connection:
\begin{enumerate}
	\item Turn BrailleNote on.
	\item Open Options Menu \dotfill {\textcolor{accent}{\MakeUppercase{\textbf{ space + o }}}}
	\item Connectivity Options \dotfill {\textcolor{accent}{\MakeUppercase{\textbf{ letter c }}}}
	\item Setup Options for Wireless Internet \dotfill {\textcolor{accent}{\MakeUppercase{\textbf{ letter w }}}}
	\item "Wireless Ethernet on? Currently no" is displayed.
	\item Toggle on \dotfill {\textcolor{accent}{\MakeUppercase{\textbf{ letter y }}}}
	\item Scan for Available Connections \dotfill {\textcolor{accent}{\MakeUppercase{\textbf{ letter s  }}}}
	\item Find your connection in list \dotfill {\textcolor{accent}{\MakeUppercase{\textbf{ space / backspace, enter }}}}
	\item "Entry list for connection configurations"
	\item Go to first field \dotfill {\textcolor{accent}{\MakeUppercase{\textbf{ down arrow }}}}
	\item "Connection configuration name?" is displayed along with the name of your connection name.
	\item Move to next field \dotfill {\textcolor{accent}{\MakeUppercase{\textbf{ advance thumb-key }}}}
	\item "Obtain an IP address automatically? Yes" is displayed.
	\item Move to next field \dotfill {\textcolor{accent}{\MakeUppercase{\textbf{ advance thumb-key }}}}
	\item "use a proxy server for this connection? No" is displayed.
	\item If I wanted the option to say yes \dotfill {\textcolor{accent}{\MakeUppercase{\textbf{ letter y  }}}}
	      \begin{itemize}\item if I did change to "yes" other fields and choices would become available.
	      \end{itemize}
	\item Context dependent help \dotfill {\textcolor{accent}{\MakeUppercase{\textbf{ space + h }}}}
	\item When you finish with the settings, exit \dotfill {\textcolor{accent}{\MakeUppercase{\textbf{ space + e  }}}}
	\item "Add this record?" \dotfill {\textcolor{accent}{\MakeUppercase{\textbf{ letter y for yes. }}}}
	\item "Wireless Ethernet menu" is displayed.
	\item "reconnect using existing configurations" \dotfill {\textcolor{accent}{\MakeUppercase{\textbf{ space + enter. }}}}
	\item If your settings are correct, you will be connected to the wireless connection.
	\item You will not have to repeat this process
\end{enumerate}
\clearpage 
\subsection{Quiz}:
If the statement is true, write or say true. Write or say false if the statement is false.
\begin{enumerate}
	\item Space + o will open the options menu.
	\item Space + dots34  will toggle among choices.
	\item To exit a document, use e chord (e + space).
	\item Press back thumb-key + space to turn the speech off.
	\item In the options menu, c will open the connectivity menu.
	\item Space + dots2 5 will speak the current word when speech on or speech on request is on.
	\item to check the power status, press p chord + o.
	\item Space + f will help you find text.
	\item When writing a Key Word document, the BrailleNote will automatically save the document for you.
	\item Space + b will open the block commands menu.
\end{enumerate}
%%%%%%%%%%%%%%%%%%%%%%%%%%%%%%%%%%%%%%%%%%
%%%%%%%%%%%%%%%%%%%%%%%%%%%%%%%%%%%%%%%%%%
%                                        %
%              Lesson 13                 %
%                                        %
%%%%%%%%%%%%%%%%%%%%%%%%%%%%%%%%%%%%%%%%%%
%%%%%%%%%%%%%%%%%%%%%%%%%%%%%%%%%%%%%%%%%%
\clearpage
\section{Google Search}
Once the wireless or other Internet connection has been established, you may start using Key Web, the web browser on the BrailleNote. By default, the HumanWare web page is the home page. In this lesson, we'll use google to conduct a simple search.	
\subsection{Tasks}
\begin{enumerate}
	\item First, ensure wireless is on:
	      \begin{enumerate}
		      \item Options menu \dotfill {\textcolor{accent}{\MakeUppercase{\textbf{ space + o }}}}
		      \item Connectivity menu \dotfill {\textcolor{accent}{\MakeUppercase{\textbf{ letter c }}}}
		      \item Wireless \dotfill {\textcolor{accent}{\MakeUppercase{\textbf{ letter w }}}}
		      \item "Wireless ethernet on?" Toggle to yes \dotfill {\textcolor{accent}{\MakeUppercase{\textbf{ space, enter }}}}
		      \item Reconnect \dotfill {\textcolor{accent}{\MakeUppercase{\textbf{ r }}}}
		      \item Return to Main Menu \dotfill {\textcolor{accent}{\MakeUppercase{\textbf{ space + dot123456 }}}}
	      \end{enumerate}
	\item Use the Internet:
	      \begin{enumerate}
		      \item From Main Menu, open the web browser \dotfill {\textcolor{accent}{\MakeUppercase{\textbf{ letter i }}}}
		      \item "Address?" \dotfill {\textcolor{accent}{\MakeUppercase{\textbf{ type www.google.com, enter }}}}
		      \item "Connect using which configuration?"
		            \begin{itemize}
			            \item If your connection name is displayed \dotfill {\textcolor{accent}{\MakeUppercase{\textbf{  enter }}}}
			            \item If not, toggle to your connection name \dotfill {\textcolor{accent}{\MakeUppercase{\textbf{ space, enter }}}}
		            \end{itemize}
		      \item You can move back and forward by input control.  An input control is a web page element, such as an edit box for writing, that you can interact with.
		            \begin{itemize}
			            \item Previous input control \dotfill {\textcolor{accent}{\MakeUppercase{\textbf{ space + dots13  }}}}
			            \item Next input control \dotfill {\textcolor{accent}{\MakeUppercase{\textbf{ space + dots46 }}}}
			            \item Text input / search bar \dotfill {\textcolor{accent}{\MakeUppercase{\textbf{ space + 46 }}}}
			            \item Write a simple search term
			            \item To search \dotfill {\textcolor{accent}{\MakeUppercase{\textbf{ enter }}}}
		            \end{itemize}
		      \item A results page should be displayed. Use review commands to navigate and read the page.
		            \begin{itemize}
			            \item To just hear page links \dotfill {\textcolor{accent}{\MakeUppercase{\textbf{ space / backspace }}}}
			            \item To follow or open a link \dotfill {\textcolor{accent}{\MakeUppercase{\textbf{ enter  }}}}
			            \item If link is known \dotfill {\textcolor{accent}{\MakeUppercase{\textbf{ initial letter of link }}}}
			            \item To read using the braille display \dotfill {\textcolor{accent}{\MakeUppercase{\textbf{ back or advance thumb-keys }}}}
		            \end{itemize}
		      \item If using speech
		            \begin{itemize}
			            \item read the previous sentence \dotfill {\textcolor{accent}{\MakeUppercase{\textbf{ space + dot1  }}}}
			            \item read the current sentence \dotfill {\textcolor{accent}{\MakeUppercase{\textbf{ space + dot14 }}}}
			            \item read the next sentence \dotfill {\textcolor{accent}{\MakeUppercase{\textbf{ space + dot4 }}}}
		            \end{itemize}
		      \item Other Browser Commands are available to you.
		            \begin{itemize}
			            \item Open new URL \dotfill {\textcolor{accent}{\MakeUppercase{\textbf{ enter + o  }}}}
			            \item Display name of current web page \dotfill {\textcolor{accent}{\MakeUppercase{\textbf{ space + i }}}}
			            \item Go forward a page in session history \dotfill {\textcolor{accent}{\MakeUppercase{\textbf{ enter + f }}}}
			            \item Go back a page in session history \dotfill {\textcolor{accent}{\MakeUppercase{\textbf{  enter + b  }}}}
			            \item Refresh the current web page \dotfill {\textcolor{accent}{\MakeUppercase{\textbf{ enter + r. }}}}
			            \item Hear current input control \dotfill {\textcolor{accent}{\MakeUppercase{\textbf{ space + dots1346  }}}}
		            \end{itemize}
	      \end{enumerate}
\end{enumerate}

\clearpage
\subsection{Quiz}:
Write or say the best answer choice to each question.
\begin{enumerate}
	\item From Main Menu, what letter should be pressed to go to the web browser?
	      \begin{enumerate}
		      \item b
		      \item w
		      \item i
		      \item g
	      \end{enumerate}
	\item What is the command to go back by a web page in the session history?
	      \begin{enumerate}
		      \item enter + b
		      \item backspace
		      \item previous thumb-key
		      \item enter + L
	      \end{enumerate}
	\item On a web page, what will space + 4 6 do?
	      \begin{enumerate}
		      \item jump to the next link
		      \item move to the next input control
		      \item move forward a page in the session history
		      \item refresh the page
	      \end{enumerate}
	\item To read forward by sentence or line with speech on a web page, what should you press?
	      \begin{enumerate}
		      \item space + g
		      \item space + dot1 + 4
		      \item space + dot1
		      \item space + dot4
	      \end{enumerate}
	\item What does space + x do?
	      \begin{enumerate}
		      \item exit the internet
		      \item read current page
		      \item read current input control
		      \item read current line
	      \end{enumerate}
	\item How can you read and move forward by a width of the braille display?
	      \begin{enumerate}
		      \item previous thumb-key
		      \item back thumb-key
		      \item advance thumb-key
		      \item next thumb-key
	      \end{enumerate}
	\item what does enter + dot1 do?
	      \begin{enumerate}
		      \item speak faster
		      \item speak slower
		      \item decrease the pitch
		      \item increase the pitch
	      \end{enumerate}
	\item How can you check the power status?
	      \begin{enumerate}
		      \item space with a full cell
		      \item enter + p
		      \item space + p
		      \item o chord, then p
	      \end{enumerate}
	\item How can you go to the last searched for text string?
	      \begin{enumerate}
		      \item enter + n
		      \item space + n
		      \item back + advance thumb-key
		      \item space + backspace + n
	      \end{enumerate}
	\item What is the name of the web browser on the BrailleNote?
	      \begin{enumerate}
		      \item Key Web
		      \item Fire Fox
		      \item Internet Explorer
		      \item HumanWare
	      \end{enumerate}
\end{enumerate}

%%%%%%%%%%%%%%%%%%%%%%%%%%%%%%%%%%%%%%%%%%
%%%%%%%%%%%%%%%%%%%%%%%%%%%%%%%%%%%%%%%%%%
%                                        %
%                                        %
%                                        %
%%%%%%%%%%%%%%%%%%%%%%%%%%%%%%%%%%%%%%%%%%
%%%%%%%%%%%%%%%%%%%%%%%%%%%%%%%%%%%%%%%%%%
\section{BrailleNote Apex Commands Taught in "School Skills" Lessons}
\begin{enumerate}
	\item General Commands
	      \begin{itemize}
		      \item Return to Main Menu with \dotfill {\textcolor{accent}{\MakeUppercase{\textbf{ space + dots123456 }}}}
		      \item Toggle speech modes \dotfill {\textcolor{accent}{\MakeUppercase{\textbf{ space + Previous thumb-key }}}}
		      \item Open word processor From Main Menu \dotfill {\textcolor{accent}{\MakeUppercase{\textbf{ letter w }}}}
		      \item Exit a document \dotfill {\textcolor{accent}{\MakeUppercase{\textbf{ space + e }}}}
		      \item Check the time \dotfill {\textcolor{accent}{\MakeUppercase{\textbf{ enter + t }}}}
		      \item Check the date \dotfill {\textcolor{accent}{\MakeUppercase{\textbf{ enter + d }}}}
		      \item Open Utilities menu from Main Menu \dotfill {\textcolor{accent}{\MakeUppercase{\textbf{ letter u }}}}
		      \item Toggle among choices \dotfill {\textcolor{accent}{\MakeUppercase{\textbf{ space + dots34 }}}}
		      \item Open File Manager From Main Menu \dotfill {\textcolor{accent}{\MakeUppercase{\textbf{ letter f }}}}
		      \item Turn the volume up \dotfill {\textcolor{accent}{\MakeUppercase{\textbf{ enter + dot4 }}}}
		      \item Turn volume down \dotfill {\textcolor{accent}{\MakeUppercase{\textbf{ enter + dot1 }}}}
		      \item Increase speech rate \dotfill {\textcolor{accent}{\MakeUppercase{\textbf{ enter + dot6 }}}}
		      \item Decrease speech rate \dotfill {\textcolor{accent}{\MakeUppercase{\textbf{ enter + dot3 }}}}
		      \item Make the pitch higher \dotfill {\textcolor{accent}{\MakeUppercase{\textbf{ enter + dot5 }}}}
		      \item Lower speech pitch \dotfill {\textcolor{accent}{\MakeUppercase{\textbf{ enter + dot2 }}}}
		      \item Move forward by word \dotfill {\textcolor{accent}{\MakeUppercase{\textbf{ space + dot5 }}}}
		      \item Move back by word \dotfill {\textcolor{accent}{\MakeUppercase{\textbf{ space + dot2 }}}}
		      \item Read current word \dotfill {\textcolor{accent}{\MakeUppercase{\textbf{ space + dots25 }}}}
		      \item Move forward by character \dotfill {\textcolor{accent}{\MakeUppercase{\textbf{ space + dot6 }}}}
		      \item Back by character \dotfill {\textcolor{accent}{\MakeUppercase{\textbf{ space + dot3 }}}}
		      \item Read current character \dotfill {\textcolor{accent}{\MakeUppercase{\textbf{ space + dots 36 }}}}
		      \item Move by line \dotfill {\textcolor{accent}{\MakeUppercase{\textbf{ space + dot4 }}}}
		      \item Back by line \dotfill {\textcolor{accent}{\MakeUppercase{\textbf{ space + dot1 }}}}
		      \item Read current line \dotfill {\textcolor{accent}{\MakeUppercase{\textbf{ space + dots14 }}}}
		      \item Check the power status \dotfill {\textcolor{accent}{\MakeUppercase{\textbf{ space + o, letter p }}}}
		      \item Move to top of document or list \dotfill {\textcolor{accent}{\MakeUppercase{\textbf{ space + dots123 }}}}
		      \item Move to bottom of document or list \dotfill {\textcolor{accent}{\MakeUppercase{\textbf{ space + dots456 }}}}
		      \item Move forward by display width \dotfill {\textcolor{accent}{\MakeUppercase{\textbf{ Advance thumb-key  }}}}
		      \item Move back by display width  \dotfill {\textcolor{accent}{\MakeUppercase{\textbf{ Back thumb-key }}}}
		      \item Place cursor \dotfill {\textcolor{accent}{\MakeUppercase{\textbf{ cursor routing keys }}}}
		      \item Open File Manager \dotfill {\textcolor{accent}{\MakeUppercase{\textbf{ enter + backspace + f }}}}
		      \item Open Word Processor \dotfill {\textcolor{accent}{\MakeUppercase{\textbf{ enter + backspace + w }}}}
		      \item Rename folder or document in list \dotfill {\textcolor{accent}{\MakeUppercase{\textbf{ backspace + r }}}}
		      \item Delete folder or document from list \dotfill {\textcolor{accent}{\MakeUppercase{\textbf{ backspace + dots14 }}}}
		      \item Open scientific calculator \dotfill {\textcolor{accent}{\MakeUppercase{\textbf{ enter + backspace + s }}}}
		      \item Contextual Help Menu \dotfill {\textcolor{accent}{\MakeUppercase{\textbf{ space + h }}}}
		      \item Go forward reading \dotfill {\textcolor{accent}{\MakeUppercase{\textbf{ space + g }}}}
		      \item Stop reading \dotfill {\textcolor{accent}{\MakeUppercase{\textbf{ space + backspace }}}}
		      \item Start automatic reading with braille display \dotfill {\textcolor{accent}{\MakeUppercase{\textbf{ space + dots23456 }}}}
		      \item While automatic reading, slow rate \dotfill {\textcolor{accent}{\MakeUppercase{\textbf{ previous thumb-key }}}}
		      \item While automatic reading, increase rate \dotfill {\textcolor{accent}{\MakeUppercase{\textbf{ next thumb-key }}}}
		      \item Stop automatic reading \dotfill {\textcolor{accent}{\MakeUppercase{\textbf{ previous + next thumb keys }}}}
		      \item Find a string of text \dotfill {\textcolor{accent}{\MakeUppercase{\textbf{ space + f }}}}
		      \item Find next occurrence of text string \dotfill {\textcolor{accent}{\MakeUppercase{\textbf{ space + n }}}}
		      \item Open the block commands menu \dotfill {\textcolor{accent}{\MakeUppercase{\textbf{ space + b. }}}}
		      \item Open last opened document \dotfill {\textcolor{accent}{\MakeUppercase{\textbf{ space + dots1256 }}}}
		      \item Open Spell Check \dotfill {\textcolor{accent}{\MakeUppercase{\textbf{ space + dots16  }}}}
		      \item Open options menu \dotfill {\textcolor{accent}{\MakeUppercase{\textbf{ space + o }}}}
		      \item In the options menu, open connectivity menu \dotfill {\textcolor{accent}{\MakeUppercase{\textbf{ letter c  }}}}
	      \end{itemize}
	\item Calculator Commands:
	      \begin{itemize}
		      \item Clear calculation \dotfill {\textcolor{accent}{\MakeUppercase{\textbf{ space + dots356.  }}}}
		      \item Delete the last key pressed \dotfill {\textcolor{accent}{\MakeUppercase{\textbf{ backspace }}}}
		      \item For addition  \dotfill {\textcolor{accent}{\MakeUppercase{\textbf{ dots5,235. }}}}
		      \item For subtraction \dotfill {\textcolor{accent}{\MakeUppercase{\textbf{ Dots5,36 }}}}
		      \item For multiplication \dotfill {\textcolor{accent}{\MakeUppercase{\textbf{ dots5,236. }}}}
		      \item For division \dotfill {\textcolor{accent}{\MakeUppercase{\textbf{ dots5,34 }}}}
		      \item For the decimal point \dotfill {\textcolor{accent}{\MakeUppercase{\textbf{ dots256. }}}}
		      \item To clear the last number or operator \dotfill {\textcolor{accent}{\MakeUppercase{\textbf{ space + dots25. }}}}
		      \item Toggle display result or input \dotfill {\textcolor{accent}{\MakeUppercase{\textbf{ previous + next thumb keys }}}}
	      \end{itemize}
	\item Web Browser-Specific Commands:
	      \begin{itemize}
		      \item Move to next input control \dotfill {\textcolor{accent}{\MakeUppercase{\textbf{ space + dots46 }}}}
		      \item Move to previous input control \dotfill {\textcolor{accent}{\MakeUppercase{\textbf{ space + dots13. }}}}
		      \item Move forward by link \dotfill {\textcolor{accent}{\MakeUppercase{\textbf{ space }}}}
		      \item Move backward by link \dotfill {\textcolor{accent}{\MakeUppercase{\textbf{ backspace }}}}
		      \item Jump to a link \dotfill {\textcolor{accent}{\MakeUppercase{\textbf{ press initial letter of link }}}}
		      \item Follow a link \dotfill {\textcolor{accent}{\MakeUppercase{\textbf{ enter }}}}
		      \item Open new URL  \dotfill {\textcolor{accent}{\MakeUppercase{\textbf{  enter + o }}}}
		      \item Display name of current web page \dotfill {\textcolor{accent}{\MakeUppercase{\textbf{ space + i }}}}
		      \item Go forward one page in session history \dotfill {\textcolor{accent}{\MakeUppercase{\textbf{ enter + f }}}}
		      \item Go back one page in session history \dotfill {\textcolor{accent}{\MakeUppercase{\textbf{ enter + b  }}}}
		      \item Refresh current web page \dotfill {\textcolor{accent}{\MakeUppercase{\textbf{  enter + r }}}}
		      \item Hear current input control \dotfill {\textcolor{accent}{\MakeUppercase{\textbf{ space + x }}}}
	      \end{itemize}
\end{enumerate}
%%%%%%%%%%%%%%%%%%%%%%%%%%%%%%%%%%%%%%%%%%
%%%%%%%%%%%%%%%%%%%%%%%%%%%%%%%%%%%%%%%%%%
%                                        %
%                                        %
%                                        %
%%%%%%%%%%%%%%%%%%%%%%%%%%%%%%%%%%%%%%%%%%
%%%%%%%%%%%%%%%%%%%%%%%%%%%%%%%%%%%%%%%%%%
\clearpage
\section{Extra Practice 1: Word Processing }
\
Put the following information in an accessible format for your student. They may follow the instructions to complete the assignment. Observe the student and give help as necessary or assign the work for independent practice at home. Check work and give feedback to ensure understanding. Adapt the assignment as you need to.
\subsection{Task}
\begin{enumerate}
	\item Begin at main menu
	\item Create a folder titled "Braille"
	\item In the Braille folder, create a file called "Practice 1"
	\item Write your name at the top of the page
	\item Go to the next line
	\item Write or insert the date
	\item Go to the next line
	\item Write Practice 1
	\item Go 2 lines down
	\item Use each word below in a complete sentence
	      \begin{itemize}
		      \item as
		      \item you
		      \item can
		      \item go
		      \item more
		      \item people
		      \item very
	      \end{itemize}
	\item Read and edit your sentences for errors. Make any corrections that you find
	\item Exit the document to the main menu
\end{enumerate}

\clearpage
\subsection{Scoring Rubric}
{
	\renewcommand{\arraystretch}{1.5}
	\begin{table}[!htbp]
		\centering
		\begin{tabular}{|l|r|}
			\hline
			\multicolumn{2}{|c|}{2 = independent, 1 = assistance required, 0 = no complete} \\
			\hline
			Step                                                               & Score \\[.5em]
			\hline
			Begin at main menu.                                                &       \\hline
			Create a folder called "Braille."                                  &       \\ \hline
			In the Braille folder, create a file called Practice 1.            &       \\ \hline
			Write your name at the top of the document.                        &       \\ \hline
			Write or insert the date on a separate line.                       &       \\ \hline
			Write the assignment title on the 3rd line.                        &       \\ \hline
			Leave one blank line between the heading and the document content. &       \\ \hline
			Write and edit sentences to meet teacher satisfaction.             &       \\ \hline
			Number or letter sentences appropriately.                          &       \\ \hline
			Print or give document to teacher as instructed.                   &       \\ \hline
		\end{tabular}
		
	\end{table}
}
%%%%%%%%%%%%%%%%%%%%%%%%%%%%%%%%%%%%%%%%%%
%%%%%%%%%%%%%%%%%%%%%%%%%%%%%%%%%%%%%%%%%%
%                                        %
%                                        %
%                                        %
%%%%%%%%%%%%%%%%%%%%%%%%%%%%%%%%%%%%%%%%%%
%%%%%%%%%%%%%%%%%%%%%%%%%%%%%%%%%%%%%%%%%%
\clearpage
\section{Extra Practice 2: Word Processing}
\
Put the following information in an accessible format for your student. They may follow the instructions to complete the assignment. Observe the student and give help as necessary or assign the work for independent practice at home. Check work and give feedback to ensure understanding. Adapt the assignment as you need to.
\subsection{Task}
\begin{enumerate}
	\item Begin at main menu
	\item Open the folder titled "Braille"
	\item In the Braille folder, create a file called "Practice 2"
	\item Write your name at the top of the page
	\item Go to the next line
	\item Write or insert the date
	\item Go to the next line
	\item Write Practice 2
	\item Go 2 lines down
	\item Write each sentence
	      \begin{enumerate}
		      \item I do have fun.
		      \item We like it at home!
		      \item It is from me?
		      \item I can go, but I will not.
		      \item It is so nice.
		      \item That bus is big.
		      \item Every boat is not small.
	      \end{enumerate}
	\item Read and edit your sentences for errors. Make any corrections that you find
	\item Exit the document to the main menu
\end{enumerate}

\clearpage
\subsection{Scoring Rubric}
{
\renewcommand{\arraystretch}{1.5}
\begin{table}[!htbp]
	\centering
	\begin{tabular}{|l|r|}
		\hline
		\multicolumn{2}{|c|}{2 = independent, 1 = assistance required, 0 = no complete} \\
		\hline
		Step                                                                               & Score \\[.5em]
		\hline
		Begin at main menu.                                                                &       \\ \hline
		Open the folder called "Braille."                                                  &       \\ \hline	
		In the Braille folder, create a file called Practice 2.                            &       \\ \hline	
		Write your name at the top of the document.                                        &       \\ \hline
		Write or insert the date on a separate line.                                       &       \\ \hline		
		Write the assignment title on the 3rd line.                                        &       \\ \hline	
		Leave one blank line between the heading and the document content of the document. &       \\ \hline
		Write and edit sentences to meet teacher satisfaction.                             &       \\ \hline		
		Number or letter sentences appropriately.                                          &       \\ \hline
		Print or give document to teacher as instructed.                                   &       \\ \hline
	\end{tabular}
	
\end{table}
%%%%%%%%%%%%%%%%%%%%%%%%%%%%%%%%%%%%%%%%%%
%%%%%%%%%%%%%%%%%%%%%%%%%%%%%%%%%%%%%%%%%%
%                                        %
%                                        %
%                                        %
%%%%%%%%%%%%%%%%%%%%%%%%%%%%%%%%%%%%%%%%%%
%%%%%%%%%%%%%%%%%%%%%%%%%%%%%%%%%%%%%%%%%%
\clearpage
\section{Extra Practice 3: Word Processing III}
\
Put the following information in an accessible format for your student. They may follow the instructions to complete the assignment. Observe the student and give help as necessary or assign the work for independent practice at home. Check work and give feedback to ensure understanding. Adapt the assignment as you need to.
\subsection{Task}
\begin{enumerate}
	\item Begin at main menu.
	\item Open the folder titled "Braille."
	\item In the Braille folder, create a file called "Practice 3."
	\item Write your name at the top of the page.
	\item Go to the next line.
	\item Write or insert the date.
	\item Go to the next line
	\item Write Practice 3.
	\item Go down 2 lines
	\item Write each question.
	      \begin{enumerate}
		      \item Do you have knowledge about that?
		      \item Will you play the game again?
		      \item Is your dog cute?
		      \item Would you help us?
		      \item Will it get very hot?
	      \end{enumerate}
	\item Read and edit your sentences for errors. Make any corrections that you find.
	\item Exit the document to the main menu
\end{enumerate}

\clearpage
\subsection{Scoring Rubric}
{
\renewcommand{\arraystretch}{1.5}
\begin{table}[!htbp]
	\centering
	\begin{tabular}{|l|r|}
		\hline
		\multicolumn{2}{|c|}{2 = independent, 1 = assistance required, 0 = no complete} \\
		\hline
		Step                                                               & Score \\[.5em]
		\hline
		Begin at main menu.                                                &       \\ \hline	
		Create a folder called "Braille."                                  &       \\ \hline	
		In the Braille folder, create a file called Practice 3.            &       \\ \hline		
		Write your name at the top of the document.                        &       \\ \hline		
		Write or insert the date on a separate line.                       &       \\ \hline		
		Write the assignment title on the 3rd line.                        &       \\ \hline		
		Leave one blank line between the heading and the document content. &       \\ \hline		
		Write and edit sentences to meet teacher satisfaction.             &       \\ \hline		
		Number or letter sentences appropriately.                          &       \\ \hline
		Print or give document to teacher as instructed.                   &       \\ \hline
	\end{tabular}
	
\end{table}
%%%%%%%%%%%%%%%%%%%%%%%%%%%%%%%%%%%%%%%%%%
%%%%%%%%%%%%%%%%%%%%%%%%%%%%%%%%%%%%%%%%%%
%                                        %
%                                        %
%                                        %
%%%%%%%%%%%%%%%%%%%%%%%%%%%%%%%%%%%%%%%%%%
%%%%%%%%%%%%%%%%%%%%%%%%%%%%%%%%%%%%%%%%%%
\clearpage
\section{Extra Practice: Scientific Calculator}
\
Put the following information in an accessible format for your student. They may follow the instructions to complete the assignment. Observe the student and give help as necessary or assign the work for independent practice at home. Check work and give feedback to ensure understanding. Adapt the assignment as you need to.
\subsection{Task}
\begin{enumerate}
	\item Set Up
	      \begin{enumerate}
		      \item Create a folder called "Math"
		      \item Create a file called "Addition"
		      \item Write a heading that includes your name, the date, and the assignment name
		      \item Leave a blank line between the heading and document
		      \item Write the number 1 followed by a period and a space
		      \item Use enter + backspace + s to switch to the scientific calculator
		      \item Complete number 1 in the Addition Problems below.
		      \item Obtain a result in the calculator
		      \item Switch back to the word processor by using enter + backspace + w
		      \item Insert the result into the document. Go to the insert menu with enter + i
		      \item Press c for calculation. Press r for result.
		      \item Press a new line between each math problem and number each problem appropriately.
		      \item Return to the calculator when you are ready for the next problem.
		      \item Remember to press backspace + c to clear the calculator before completing the next problem.
		      \item Continue switching between applications in order to complete the assignment.
		      \item Exit the document to the main menu
	      \end{enumerate}
	\item Addition Problems
	      \begin{enumerate}
		      \item 45 + 67 =
		      \item 59 + 77 =
		      \item 117 + 36 =
		      \item 404 + 29 =
		      \item 381 + 75
		      \item 8.7 + 5.5 =
	      \end{enumerate}
\end{enumerate}
\clearpage
\subsection{Scoring Rubric}
{
\renewcommand{\arraystretch}{1.5}
\begin{table}[!htbp]
	\centering
	\begin{tabular}{|l|r|}
		\hline
		\multicolumn{2}{|c|}{2 = independent, 1 = assistance required, 0 = no complete} \\
		\hline
		Step                                                  & Score \\[.5em]
		\hline
		Create a folder called "Math."                        &       \\ \hline	
		Create a file called "Addition."                      &       \\ \hline		
		Write an appropriate heading.                         &       \\ \hline
		Leave a blank line between your heading and document. &       \\ \hline		
		Number the problems appropriately.                    &       \\ \hline
		Switch between applications as necessary.             &       \\ \hline
		Use the calculator to obtain results.                 &       \\ \hline
		Insert the result into a word document.               &       \\ \hline
		Clear the calculator when needed.                     &       \\ \hline
		Print or turn the document in as required.            &       \\ \hline
	\end{tabular}
	
\end{table}
%%%%%%%%%%%%%%%%%%%%%%%%%%%%%%%%%%%%%%%%%%
%%%%%%%%%%%%%%%%%%%%%%%%%%%%%%%%%%%%%%%%%%
%                                        %
%                                        %
%                                        %
%%%%%%%%%%%%%%%%%%%%%%%%%%%%%%%%%%%%%%%%%%
%%%%%%%%%%%%%%%%%%%%%%%%%%%%%%%%%%%%%%%%%%
\section{Miscellaneous: Use of the Oxford Dictionary}
\
\begin{enumerate}
	\item You can open the dictionary/thesaurus from anywhere on the BrailleNote.
	      \begin{enumerate}
		      \item Open Options Menu
		      \item Select "Look up" or press l
		      \item "Use dictionary or thesaurus?" is the cue.
		      \item Press d for dictionary.
		      \item "Word to look up in dictionary?" is the cue.
		      \item Write the word you want to look up.
		      \item Press enter.
		      \item You can use the back or advance thumb-key to read the braille.
	      \end{enumerate}
	\item Other available commands in dictionary:
	      \begin{enumerate}
		      \item For information about the word \dotfill {\textcolor{accent}{\MakeUppercase{\textbf{ space + i  }}}}
		      \item Pronounce word (speech on) \dotfill {\textcolor{accent}{\MakeUppercase{\textbf{ enter + dots25 }}}}
		      \item Move to the next entry \dotfill {\textcolor{accent}{\MakeUppercase{\textbf{ space + dots56 }}}}
		      \item More to previous entry \dotfill {\textcolor{accent}{\MakeUppercase{\textbf{ space + dots23 }}}}
		      \item Insert word under cursor \dotfill {\textcolor{accent}{\MakeUppercase{\textbf{ backspace + i  }}}}
		      \item Copy entry to clipboard \dotfill {\textcolor{accent}{\MakeUppercase{\textbf{ backspace + k }}}}
		      \item Look up another word \dotfill {\textcolor{accent}{\MakeUppercase{\textbf{ space + e }}}}
	      \end{enumerate}
	\item Use dictionary from within a document:
	      \begin{enumerate}
		      \item Open your document.
		      \item Place the cursor on the word you want to look up.
		      \item Press space + o.
		      \item Press letter l for "look up".
		      \item Press d for dictionary.
		      \item Notice the word your cursor was on in the document is already displayed after the "word to look up" cue.
		      \item Press enter to look up the word.
		      \item Use dictionary commands to read.
	      \end{enumerate}
	\item Move forward or back by entry:
	      \begin{enumerate}
		      \item Let's say you look up the word "bobber". You are on that word in the entry and can read the definition.
		      \item To go to the next entry \dotfill {\textcolor{accent}{\MakeUppercase{\textbf{ space + dots56 }}}}
		      \item To go back the other way \dotfill {\textcolor{accent}{\MakeUppercase{\textbf{ space + 23.  }}}}
		      \item To move back and forward by word when reading the definition, use space + dot2 or space + dot5.
	      \end{enumerate}
\end{enumerate}

%%%%%%%%%%%%%%%%%%%%%%%%%%%%%%%%%%%%%%%%%%
%%%%%%%%%%%%%%%%%%%%%%%%%%%%%%%%%%%%%%%%%%
%                                        %
%                                        %
%                                        %
%%%%%%%%%%%%%%%%%%%%%%%%%%%%%%%%%%%%%%%%%%
%%%%%%%%%%%%%%%%%%%%%%%%%%%%%%%%%%%%%%%%%%
\section{Miscellaneous: Use the Pronunciation Dictionary}
\
Pronunciation Dictionary
\begin{enumerate}
	\item From Main Menu, open the Utilities Menu \dotfill {\textcolor{accent}{\MakeUppercase{\textbf{ letter u }}}}
	\item Locate the Pronunciation Dictionary and open it
	\item The menu items
	      \begin{itemize}
		      \item add a word
		      \item change a word
		      \item delete a world
	      \end{itemize}
	\item To add a word, focus on Add a Word and press enter.
	\item Type the word to add. Spell the word correctly. Use computer braille.
	\item Type the misspelling of the word so that it will be pronounced correctly.
	\item Press enter.
	\item The word has been added.
	\item Use the other menu options to change or delete a word.
\end{enumerate}

%%%%%%%%%%%%%%%%%%%%%%%%%%%%%%%%%%%%%%%%%%
%%%%%%%%%%%%%%%%%%%%%%%%%%%%%%%%%%%%%%%%%%
%                                        %
%                                        %
%                                        %
%%%%%%%%%%%%%%%%%%%%%%%%%%%%%%%%%%%%%%%%%%
%%%%%%%%%%%%%%%%%%%%%%%%%%%%%%%%%%%%%%%%%%
\section{Miscellaneous: View BrailleNote on Monitor via VGA}
\begin{enumerate}
	\item Connect BrailleNote to Monitor with VGA cable
	\item Open Options Menu \dotfill{\textcolor{accent}{\MakeUppercase{\textbf{ space + o }}}}
	\item Open Visual Display \dotfill {\textcolor{accent}{\MakeUppercase{\textbf{ letter v }}}}
	\item Turn on visual display \dotfill {\textcolor{accent}{\MakeUppercase{\textbf{ letter v, enter }}}}
	\item You should see what's happening on the BrailleNote on the computer monitor.
\end{enumerate}
%%%%%%%%%%%%%%%%%%%%%%%%%%%%%%%%%%%%%%%%%%%%%%%%%%%%%%%%%%%%%%%%%%%%%%%%%%%%
%%%%%%%%%%%%%%%%%%%%%%%%%%%%%%%%%%%%%%%%%%%%%%%%%%%%%%%%%%%%%%%%%%%%%%%%%%%%
%%%%%%%%%%%%%%%%%%                                        %%%%%%%%%%%%%%%%%%
%%%%%%%%%%%%%%%%%%     GENERAL BRAILLENOTE APEX USAGE     %%%%%%%%%%%%%%%%%%
%%%%%%%%%%%%%%%%%%                                        %%%%%%%%%%%%%%%%%%
%%%%%%%%%%%%%%%%%%%%%%%%%%%%%%%%%%%%%%%%%%%%%%%%%%%%%%%%%%%%%%%%%%%%%%%%%%%%
%%%%%%%%%%%%%%%%%%%%%%%%%%%%%%%%%%%%%%%%%%%%%%%%%%%%%%%%%%%%%%%%%%%%%%%%%%%%
\setcounter{section}{0}
%%%%%%%%%%%%%%%%%%%%%%%%%%%%%%%%%%%%%%%%%%
%%%%%%%%%%%%%%%%%%%%%%%%%%%%%%%%%%%%%%%%%%
%                                        %
%              Lesson 1                  %
%                                        %
%%%%%%%%%%%%%%%%%%%%%%%%%%%%%%%%%%%%%%%%%%
%%%%%%%%%%%%%%%%%%%%%%%%%%%%%%%%%%%%%%%%%%
\clearpage
\chapter{BrailleNote Apex General Usage Lessons}
\section{ Getting Started}
Skills Addressed in this Lesson:
Learn where the BrailleNote Apex keys are located.
\begin{itemize}
	\item Six Braille writer keys
	\item Space
	\item Backspace
	\item Enter
	\item Previous thumb-key
	\item Back thumb-key
	\item Advance thumb-key
	\item Next thumb-key
	\item on and off switch
	\item Headphone jack
\end{itemize}
\subsection{Tasks}
\begin{enumerate}
	\item Ask your student to open the cover of the BrailleNote
		\begin{itemize}
		\item The cover should open away from the BrailleNote user
		\item If the BrailleNote is positioned correctly, your student will recognize six Braille writer keys and a spacebar which are similar to that of a Perkins Braille writer
		\end{itemize}
	\item Assist your student in locating the on off switch located under the velcro flap toward the back of the left panel of the BrailleNote
	\item Before turning the BrailleNote on 
	    \begin{itemize}
	        \item Explain that when the rocker switch is indented toward the user, the BrailleNote is on 
	        \item When the on off switch is indented away from the user, the BrailleNote is of
	\end{itemize}
	\item Allow your student to feel the headphone jack on the left panel of the BrailleNote toward the front.
	\item Instruct your BrailleNote user to place his or her hands on the six Braille writer keys as he or she would do on a Braille writer. 
	\item The key near the left edge of the BrailleNote, to the left of the dot3 Braille writer key, is the backspace key.
	\item The key next to the right edge of the BrailleNote, to the right of the dot6 Braille writer key, is the enter key. 
	\item The refreshable Braille display is below the spacebar. 
	\item While your student is exploring the refreshable Braille display, he or she may place both thumbs down on the thumb-keys to explore them.  The thumb-keys are on the front panel of the BrailleNote.
	\item The thumb-keys are, starting at the left:
	\begin{itemize}
	    \item previous
	    \item back
	    \item advance
	    \item next thumb-keys.
	\end{itemize}
\end{enumerate}
\subsection{Quiz}
Have your student point to each of the following:
\begin{enumerate}
	\item Earphone jack
	\item Spacebar
	\item Next thumb-key
	\item on and off rocker switch
	\item Previous thumb-key
	\item Backspace
	\item Advance thumb-key
	\item Back thumb-key
	\item Return (enter) key
	\item Keys one through six
\end{enumerate}
%%%%%%%%%%%%%%%%%%%%%%%%%%%%%%%%%%%%%%%%%%
%%%%%%%%%%%%%%%%%%%%%%%%%%%%%%%%%%%%%%%%%%
%                                        %
%              Lesson 2                  %
%                                        %
%%%%%%%%%%%%%%%%%%%%%%%%%%%%%%%%%%%%%%%%%%
%%%%%%%%%%%%%%%%%%%%%%%%%%%%%%%%%%%%%%%%%%
\clearpage
\section{ Navigating Menus}
Skills Addressed in this Lesson:
\begin{itemize}
	\item Turn the BrailleNote on or off
	\item Use space to go forward through menu items
	\item Use backspace to go back through menu items
	\item Use the advance thumb-key to move forward through menu items
	\item Use the back thumb-key to move to previous menu items
	\item When on a menu item, press enter to open it
	\item Use initial letters to move to and open menu items
	\item Use space with e to exit a menu
	\item Toggle among BrailleNote speech settings
	\item Use and understand space with h to get help from anywhere
	\item Use the shortcut to return to the Main Menu
\end{itemize}
\subsection{Tasks}
\begin{enumerate}
	\item Instruct your student to turn on the BrailleNote using the \dotfill {\textcolor{accent}{\MakeUppercase{\textbf{on/off rocker switch}}}}
	\item Access Speech Options \dotfill {\textcolor{accent}{\MakeUppercase{\textbf{previous thumb-key + space}}}}
		\item Toggle Speech Options \dotfill {\textcolor{accent}{\MakeUppercase{\textbf{hold previous thumb-key, space}}}}
	    \begin{itemize}
	      \item speech on
	     \item speech off
	     \item speech on request modes
	    \end{itemize} 
	\item Toggle options within Main menu  \dotfill {\textcolor{accent}{\MakeUppercase{\textbf{space}}}}
	\item Select desired option  \dotfill {\textcolor{accent}{\MakeUppercase{\textbf{enter}}}}
	\item Exit current menu to return to previous  \dotfill {\textcolor{accent}{\MakeUppercase{\textbf{space + e}}}}
	\item Direct Selection of options in Main Menu
	    \begin{itemize}
	        \item To select the Word Processor \dotfill {\textcolor{accent}{\MakeUppercase{\textbf{letter w}}}}
	        \item To select the Scientific Calculator \dotfill {\textcolor{accent}{\MakeUppercase{\textbf{letter s}}}}
	        \item To select the Database Manager \dotfill {\textcolor{accent}{\MakeUppercase{\textbf{letter d}}}}
	        \item To select the Games \dotfill {\textcolor{accent}{\MakeUppercase{\textbf{letter g}}}}
	        \item To select the Planner \dotfill {\textcolor{accent}{\MakeUppercase{\textbf{letter p}}}}
	        \item To select the Address List \dotfill {\textcolor{accent}{\MakeUppercase{\textbf{letter a}}}}
	        \item To select the E-mail system \dotfill {\textcolor{accent}{\MakeUppercase{\textbf{letter e}}}}
	        \item To select the Book Reader \dotfill {\textcolor{accent}{\MakeUppercase{\textbf{letter b}}}}
	        \item To select the Internet browser, \dotfill {\textcolor{accent}{\MakeUppercase{\textbf{letter i}}}}
	        \item To select the Chat (instant messaging), \dotfill {\textcolor{accent}{\MakeUppercase{\textbf{letter c}}}}
	        \item To select the Media Player, \dotfill {\textcolor{accent}{\MakeUppercase{\textbf{letter m}}}}
	        \item To select the FM Radio, p\dotfill {\textcolor{accent}{\MakeUppercase{\textbf{letter f}}}}

	    \end{itemize}
	    \item To access KeySoft programs from anywhere in BrailleNote
	    	    \begin{itemize}
	        \item To select the Word Processor \dotfill {\textcolor{accent}{\MakeUppercase{\textbf{backspace + enter + w}}}}
	        \item To select the Scientific Calculator \dotfill {\textcolor{accent}{\MakeUppercase{\textbf{backspace + enter + s}}}}
	        \item To select the Database Manager \dotfill {\textcolor{accent}{\MakeUppercase{\textbf{backspace + enter + d}}}}
	        \item To select the Games \dotfill {\textcolor{accent}{\MakeUppercase{\textbf{backspace + enter + g}}}}
	        \item To select the Planner \dotfill {\textcolor{accent}{\MakeUppercase{\textbf{backspace + enter + p}}}}
	        \item To select the Address List \dotfill {\textcolor{accent}{\MakeUppercase{\textbf{backspace + enter + a}}}}
	        \item To select the E-mail system \dotfill {\textcolor{accent}{\MakeUppercase{\textbf{backspace + enter + e}}}}
	        \item To select the Book Reader \dotfill {\textcolor{accent}{\MakeUppercase{\textbf{backspace + enter + b}}}}
	        \item To select the Internet browser, \dotfill {\textcolor{accent}{\MakeUppercase{\textbf{backspace + enter + i}}}}
	        \item To select the Chat (instant messaging), \dotfill {\textcolor{accent}{\MakeUppercase{\textbf{backspace + enter + c}}}}
	        \item To select the Media Player, \dotfill {\textcolor{accent}{\MakeUppercase{\textbf{backspace + enter + m}}}}
	        \item To select the FM Radio, \dotfill {\textcolor{accent}{\MakeUppercase{\textbf{backspace + enter + f}}}}
	    \end{itemize}
    \item Return to Main Menu \dotfill {\textcolor{accent}{\MakeUppercase{\textbf{space + dots12356}}}}
	\item Turn the BrailleNote off when all tasks are complete \dotfill {\textcolor{accent}{\MakeUppercase{\textbf{on/off Rocker Switch}}}}
\end{enumerate}
\clearpage
\subsection{Quiz}
Require your student to demonstrate competency in the following tasks.  This could be accomplished by using a written test or by requiring the student to use the BrailleNote to demonstrate mastery of the skills.
\begin{enumerate}
	\item Turn on the BrailleNote.
	\item If the BrailleNote does not say Main Menu, your student can return to it by using space with e or a full cell with space.
	\item At the Main Menu, ask the student to show you how he or she can get to the Internet prompt and open it by using only one letter.  If necessary, give a verbal cue to the student to press the letter i.
	\item Instruct your student to return to the Main Menu.  He or she should do this by using either e with space or dots1 through 6 with space.
	\item From the Main Menu, encourage your student to show you how to move forward and backward through the menu items using two different methods.  These methods are using space and backspace keys and using the back and advance thumb-keys.
	\item From the Main Menu, instruct your student to go to the e-mail menu item and open it.  This is done by pressing an e.  Do not provide verbal cues unless they are necessary during this Quiz section.
	\item From the keymail menu, ask your student to show you how he or she would get help.  If a reminder is required, state that space with h will get help from anywhere.
	\item Ask your student to read what is in the help file.  The back and advance thumb-keys will work for now.
	\item Ask your student to return to the Main Menu.  Remember, pressing e with space or a full cell with space will accomplish this task.
	\item Your student may now turn the BrailleNote off.
\end{enumerate}
%%%%%%%%%%%%%%%%%%%%%%%%%%%%%%%%%%%%%%%%%%
%%%%%%%%%%%%%%%%%%%%%%%%%%%%%%%%%%%%%%%%%%
%                                        %
%              Lesson 3                  %
%                                        %
%%%%%%%%%%%%%%%%%%%%%%%%%%%%%%%%%%%%%%%%%%
%%%%%%%%%%%%%%%%%%%%%%%%%%%%%%%%%%%%%%%%%%
\clearpage
\section{ Word Processing}
\
Skills Addressed in this Lesson:
\begin{itemize}
	\item Open the word processor.
	\item Create a document.
	\item Write a heading and two sentences in a document.
	\item Use the backspace and cursor keys to make basic corrections.
	\item Review text moving forward and back by character and word.
	\item Review text backward and forward by sentence and paragraph.
	\item Use the back and advance thumb-keys to move back and forward by one width of the Braille display.
	\item Turn the speech on and off.
	\item Turn the Braille display on and off.
	\item Determine how much power is left and become aware of when the BrailleNote needs to be charged.
\end{itemize}
\subsection{Tasks}
\begin{enumerate}
	\item Turn on the BrailleNote \dotfill {\textcolor{accent}{\MakeUppercase{\textbf{on/off rocker switch}}}}
	\item Go to Main Menu \dotfill {\textcolor{accent}{\MakeUppercase{\textbf{space + dots123456}}}}
	\item From the Main Menu, open Keyword \dotfill {\textcolor{accent}{\MakeUppercase{\textbf{space, enter}}}} 
	\item In KeyWord Menu, create a document \dotfill {\textcolor{accent}{\MakeUppercase{\textbf{space, enter}}}} or {\textcolor{accent}{\MakeUppercase{\textbf{letter c}}}}
	\item Choose Folder for Document \dotfill {\textcolor{accent}{\MakeUppercase{\textbf{space, enter}}}}
	\item Name the file by typing a name (<250 characters), then press enter
	\begin{enumerate}
    	\item When in the document, instruct your student to write:
    	    \begin{enumerate} 
    	    \item Their name, then enter
    	    \item The date, then enter
    	    \item The word ``Practice'' 
    	    \item Presenter twice to place a blank line after ``Practice''
    	    \item Now your student may write two or three sentences to get a feel for what writing in the word processor is like.
    	    \end{enumerate}
    	\item Move Cursor Location \dotfill  {\textcolor{accent}{\MakeUppercase{\textbf{cursor routing keys}}}} 
    	\item Delete text \dotfill {\textcolor{accent}{\MakeUppercase{\textbf{backspace key}}}} 
	\end{enumerate}
	\item Listed below are commands that can be used on the BrailleNote.  The current document will be too small to practice them all, but a quick review won't hurt.  Review commands are especially helpful when the speech is being used.
	      \begin{enumerate}
		      \item To move back by character \dotfill {\textcolor{accent}{\MakeUppercase{\textbf{space + dot3}}}}
		      \item To move forward by character \dotfill {\textcolor{accent}{\MakeUppercase{\textbf{space + dot6}}}}
		      \item To move back by word \dotfill {\textcolor{accent}{\MakeUppercase{\textbf{space + dot2}}}}
		      \item To move forward by word \dotfill {\textcolor{accent}{\MakeUppercase{\textbf{space + dot5}}}}
		      \item To move back by sentence \dotfill {\textcolor{accent}{\MakeUppercase{\textbf{space + dot1}}}} 
		      \item To move forward by sentence \dotfill {\textcolor{accent}{\MakeUppercase{\textbf{space + dot4}}}} 
		      \item To move back b paragraph \dotfill {\textcolor{accent}{\MakeUppercase{\textbf{space + dots23}}}}
		      \item To move forward by paragraph \dotfill {\textcolor{accent}{\MakeUppercase{\textbf{space + dots56}}}} 
        	\item To move back the width of 1 braille display \dotfill {\textcolor{accent}{\MakeUppercase{\textbf{back thumb-key}}}}
    \end{enumerate}
	\item To move back the width of 1 braille display \dotfill {\textcolor{accent}{\MakeUppercase{\textbf{advance thumb-key}}}}
	\item Turn braille display on/off \dotfill {\textcolor{accent}{\MakeUppercase{\textbf{space + next thumb-key}}}}
	\item Access Options Menu \dotfill {\textcolor{accent}{\MakeUppercase{\textbf{space + o thumb-key}}}}
    	\begin{enumerate} 
    	    \item See Power/battery level \dotfill {\textcolor{accent}{\MakeUppercase{\textbf{letter p}}}}
    	    \item Clear power level and return to document \dotfill {\textcolor{accent}{\MakeUppercase{\textbf{next thumb-key}}}}
    	\end{enumerate}
\end{enumerate}

\clearpage
\subsection{Quiz}
Your student may answer the following true and false questions.  Then, you may require them to demonstrate knowledge of the skills learned in this lesson by actually using the BrailleNote.
\begin{enumerate}
	\item It is not necessary to check the BrailleNote power level.
	\item Using the space with dot4 will go forward by a sentence.
	\item The previous thumb-key with space will turn the Braille display on and off.
	\item Space with dot2 will move forward by character in the word processor.
	\item File names can be up to 350 characters long.
	\item Space with the previous thumb-key turns the speech on and off.
	\item Space with dot6 moves forward by character.
	\item Space with o will get you to the options menu.
	\item After checking the power level, a user must first return to the Main Menu to reset the BrailleNote.
	\item Keyword is the name of the BrailleNote word processor.
\end{enumerate}

%%%%%%%%%%%%%%%%%%%%%%%%%%%%%%%%%%%%%%%%%%
%%%%%%%%%%%%%%%%%%%%%%%%%%%%%%%%%%%%%%%%%%
%                                        %
%              Lesson 4                  %
%                                        %
%%%%%%%%%%%%%%%%%%%%%%%%%%%%%%%%%%%%%%%%%%
%%%%%%%%%%%%%%%%%%%%%%%%%%%%%%%%%%%%%%%%%%
\clearpage
\section{ Word Processing}
Skills Addressed in this Lesson:
\begin{itemize}
	\item Create folders in Keyword.
	\item Create files within folders.
	\item Center a line of text.
	\item Use the delete menu to delete a character, word, previous word, sentence, paragraph, or document.
	\item Explore the help menu in word processing.
	\item Speak the time and date.
	\item Return to Main Menu.
\end{itemize}

\subsection{Tasks}
\begin{enumerate}
	\item From the Main Menu open word processor  \dotfill {\textcolor{accent}{\MakeUppercase{\textbf{space + enter}}}} or {\textcolor{accent}{\MakeUppercase{\textbf{letter w, enter}}}}
	\item Create a document called ``science'' in a folder called ``homework''
	    \begin{enumerate}
	\item From KeyWord Menu, create a document \dotfill {\textcolor{accent}{\MakeUppercase{\textbf{space + enter}}}} or {\textcolor{accent}{\MakeUppercase{\textbf{letter c, enter}}}} 
	\item At ``folder name'' prompt, type ``Homework'', then press enter 
    \item At the cue, "folder does not exist, create new?" 
    	\begin{itemize}
    	\item For yes \dotfill {\textcolor{accent}{\MakeUppercase{\textbf{letter y}}}}.  \item For no \dotfill {\textcolor{accent}{\MakeUppercase{\textbf{letter n}}}}
	    \end{itemize}
	\item At ``File Name?'' prompt, write ``Science'' as the file name and press enter
	\item You will be placed in your blank document called Science.
	\end{enumerate}
	\item Return to Main Menu \dotfill {\textcolor{accent}{\MakeUppercase{\textbf{space + dots123456}}}}
	\item Create a document called ``math'' in the ``homework''folder
	    \begin{enumerate}
	\item From KeyWord Menu, create a document \dotfill {\textcolor{accent}{\MakeUppercase{\textbf{space + enter}}}} or {\textcolor{accent}{\MakeUppercase{\textbf{letter c, enter}}}} 
	\item At ``folder name'' prompt toggle to homework folder \dotfill {\textcolor{accent}{\MakeUppercase{\textbf{space + enter}}}}
	\item At ``File Name?'' prompt, write ``Math'' as the file name and press enter
	\item You will be placed in your blank document called Math.
	\end{enumerate}
	\item Return to Main Menu \dotfill {\textcolor{accent}{\MakeUppercase{\textbf{space + dots123456}}}}
	\item Give your student the time necessary to create other folders and files as desired.
	\item Open the ``Math'' document
	\begin{enumerate}
	    \item From KeyWord Menu, open a document  \dotfill {\textcolor{accent}{\MakeUppercase{\textbf{space + enter }}}} or {\textcolor{accent}{\MakeUppercase{\textbf{letter o, enter}}}}
	    \item Toggle options to desired folder  \dotfill {\textcolor{accent}{\MakeUppercase{\textbf{space, enter}}}}
	    \item Toggle options to desired file  \dotfill {\textcolor{accent}{\MakeUppercase{\textbf{space, enter}}}}
	\end{enumerate}
	\item When in the document, instruct your student to write:
    	    \begin{enumerate} 
    	    \item Their name, then enter
    	    \item The date, then enter
    	    \item The word ``Practice'' 
    	    \item Presenter twice to place a blank line after ``Practice''
    	    \item Center text command \dotfill {\textcolor{accent}{\MakeUppercase{\textbf{enter + c}}}}
    	    
    	    \item Write ``Math''
    	    \item exit the document \dotfill {\textcolor{accent}{\MakeUppercase{\textbf{space + e}}}}
	    \end{enumerate}
	\item Using the Delete Menu
	    \begin{enumerate} 
	    \item Have student a document and write five sentences about a pet or favorite animal.
	    \item Access delete menu \dotfill {\textcolor{accent}{\MakeUppercase{\textbf{space + d}}}}
	   \item Toggle options \dotfill {\textcolor{accent}{\MakeUppercase{\textbf{space, enter}}}} or {\textcolor{accent}{\MakeUppercase{\textbf{first letter, enter}}}}
	    \begin{itemize}
	    \item To delete current character \dotfill {\textcolor{accent}{\MakeUppercase{\textbf{backspace + dots36}}}}
	    \item To delete current word \dotfill {\textcolor{accent}{\MakeUppercase{\textbf{backspace + dots25}}}}
	    \item To delete previous word \dotfill {\textcolor{accent}{\MakeUppercase{\textbf{backspace + dot2}}}}
	    \item To delete to the end of the sentence (asks for confirmation)  \dotfill {\textcolor{accent}{\MakeUppercase{\textbf{backspace + dots14}}}}
	    \item To delete to the end of the paragraph (asks for confirmation) \dotfill {\textcolor{accent}{\MakeUppercase{\textbf{backspace + dots2356}}}}
	    \item To delete to the end of the document (asks for confirmation) \dotfill {\textcolor{accent}{\MakeUppercase{\textbf{backspace + dots456}}}}
	    \end{itemize}
	\end{enumerate}
	\item Open the help menu in KeyWord \dotfill {\textcolor{accent}{\MakeUppercase{\textbf{space + h}}}}
	\item When in the help menu, the options are:
	    \begin{itemize}
	    \item edit commands
	    \item review commands
	    \item braille thumb-key commands
	    \item miscellaneous commands
	    \end{itemize}
	\item To speak the time from where you are in keyword \dotfill {\textcolor{accent}{\MakeUppercase{\textbf{enter + t}}}}
	\item To speak the date \dotfill {\textcolor{accent}{\MakeUppercase{\textbf{enter + d}}}}
	\item Return to document \dotfill {\textcolor{accent}{\MakeUppercase{\textbf{advance thumb-key}}}}
\end{enumerate}
\clearpage
\subsection{Keyword Help Menu}
\begin{enumerate}
	\item Edit Commands
		\begin{enumerate}
			\item Delete character under cursor  \dotfill {\textcolor{accent}{\MakeUppercase{\textbf{backspace + dots36}}}}
			\item Delete word under cursor, press space with dots2 and 5.
			\item Delete previous word, press backspace with dot2.
			\item Delete to end of sentence, backspace with dots1 and 4.
			\item Delete to end of paragraph, press backspace with dots2 3 5 6.
			\item Delete to end of document, press backspace with dots4 5 6.
			\item Center a line, press enter with c.
			\item Find and replace text string, press backspace with f.
			\item Change or review the layout, press space with dots2 3 4 6, l or backspace with l.
			\item Review or change page settings, press space with dots2 3 4 6 p or backspace with p.
			\item To enter a Unicode character, press backspace with dots3 5.
		\end{enumerate}
	\item Review Commands
		\begin{enumerate}
			\item Go forward reading,   \dotfill {\textcolor{accent}{\MakeUppercase{\textbf{space+g}}}}.
			\item Top of file,   \dotfill {\textcolor{accent}{\MakeUppercase{\textbf{space + dots123}}}}
			\item Bottom of file, \dotfill {\textcolor{accent}{\MakeUppercase{\textbf{space + dots456}}}}
			\item Find text string \dotfill {\textcolor{accent}{\MakeUppercase{\textbf{space + f}}}}
			\item Find next occurrence of text string last searched \dotfill {\textcolor{accent}{\MakeUppercase{\textbf{space +n}}}}
			\item To go back a character, \dotfill {\textcolor{accent}{\MakeUppercase{\textbf{space + dot3}}}}
			\item To hear the current character \dotfill {\textcolor{accent}{\MakeUppercase{\textbf{space + dots36}}}}
			\item To move to the next character \dotfill {\textcolor{accent}{\MakeUppercase{\textbf{space + dot6}}}}
			\item To move back a word \dotfill {\textcolor{accent}{\MakeUppercase{\textbf{space + dot2}}}}
			\item Hear current word \dotfill {\textcolor{accent}{\MakeUppercase{\textbf{space + dots25}}}}
			\item To move forward a word \dotfill {\textcolor{accent}{\MakeUppercase{\textbf{space + dot5}}}}
			\item To move back a sentence\dotfill {\textcolor{accent}{\MakeUppercase{\textbf{space + dot1}}}}
			\item To hear the current sentence \dotfill {\textcolor{accent}{\MakeUppercase{\textbf{space + dots14}}}}
			\item To move forward a sentence \dotfill {\textcolor{accent}{\MakeUppercase{\textbf{space + dot4}}}}
			\item To move to the previous paragraph \dotfill {\textcolor{accent}{\MakeUppercase{\textbf{space + dots23}}}}
			\item Hear current paragraph \dotfill {\textcolor{accent}{\MakeUppercase{\textbf{space + dots2356}}}}
			\item To move forward a paragraph \dotfill {\textcolor{accent}{\MakeUppercase{\textbf{space +dots56}}}}
			\item To determine the current reading mode \dotfill {\textcolor{accent}{\MakeUppercase{\textbf{space + m}}}}
			\item To change the reading mode\dotfill {\textcolor{accent}{\MakeUppercase{\textbf{ space + m repeatedly}}}}
			    \begin{itemize}
			        \item Sentence
			        \item Paragraph
			        \item Line 
			        \item Columns
			    \end{itemize}
			\item Announce key names \dotfill {\textcolor{accent}{\MakeUppercase{\textbf{space + w}}}}
			\item To query the cursor position \dotfill {\textcolor{accent}{\MakeUppercase{\textbf{space + dots156}}}}
			\item Go to any page, line, or column \dotfill {\textcolor{accent}{\MakeUppercase{\textbf{space + dots126}}}}
			\item To enter or leave review-only mode \dotfill {\textcolor{accent}{\MakeUppercase{\textbf{space + x}}}}
		\end{enumerate}
	\item Braille Thumb-key Commands
		\begin{enumerate}
			\item To advance the display by one width\dotfill {\textcolor{accent}{\MakeUppercase{\textbf{ advance thumb-key}}}}
			\item To move the display back by one width\dotfill {\textcolor{accent}{\MakeUppercase{\textbf{back thumb-key}}}}
			\item To start the display advancing automatically \dotfill {\textcolor{accent}{\MakeUppercase{\textbf{ space + dots12456}}}}
			\item To control the speed \dotfill {\textcolor{accent}{\MakeUppercase{\textbf{previous thumb-key}}}} or {\textcolor{accent}{\MakeUppercase{\textbf{next thumb-key}}}}
			\item To stop \dotfill {\textcolor{accent}{\MakeUppercase{\textbf{previous thumb-key + next thumb-key}}}}
			\item To move the display back a word  \dotfill {\textcolor{accent}{\MakeUppercase{\textbf{previous thumb-key + back thumb-key}}}}
			\item To move the display on a word  \dotfill {\textcolor{accent}{\MakeUppercase{\textbf{ previous thumb-key + advance thumb-key}}}}
			\item To turn speech on or off  \dotfill {\textcolor{accent}{\MakeUppercase{\textbf{previous thumb-key + space}}}}
			\item To turn the Braille display on or off  \dotfill {\textcolor{accent}{\MakeUppercase{\textbf{next thumb-key + space}}}}
			\item To route the cursor to the beginning of the braille display  \dotfill {\textcolor{accent}{\MakeUppercase{\textbf{backspace + advance thumb-key}}}}
			\item To cycle among the four keyboard modes \dotfill {\textcolor{accent}{\MakeUppercase{\textbf{previous thumb-key + next thumb-key}}}}
		\end{enumerate}
	\item Miscellaneous Commands
	\begin{enumerate}
		\item Block commands menu  \dotfill {\textcolor{accent}{\MakeUppercase{\textbf{space + b}}}}
		\item Spelling checker  \dotfill {\textcolor{accent}{\MakeUppercase{\textbf{space + dots16}}}}
		\item Save file  \dotfill {\textcolor{accent}{\MakeUppercase{\textbf{space + s}}}}
		\item Abandon edit  \dotfill {\textcolor{accent}{\MakeUppercase{\textbf{backspace + q}}}}
		\item Open another document  \dotfill {\textcolor{accent}{\MakeUppercase{\textbf{space + dots1256}}}}
		\item To review or change the current presentation style  \dotfill {\textcolor{accent}{\MakeUppercase{\textbf{backspace + s}}}}
		\item To insert a field or control an address list template  \dotfill {\textcolor{accent}{\MakeUppercase{\textbf{space + dots2346}}}} or  {\textcolor{accent}{\MakeUppercase{\textbf{backspace + enter + dots2346}}}}
	\end{enumerate}t
\end{enumerate}
\clearpage
\subsection{Quiz}
\begin{enumerate}
	\item Require your student to enter the Homework folder.  Create a journal file.  Your student may write a journal entry at this time.
	\item Instruct your student to create a new folder and file for you.
	\item From within a keyword document, the student should show you how to speak or display the time and date.
	\item At the top of the document, the student can show you how he or she is able to center the top line.
	\item Watch and listen as the student shows you how to use the help menu in keyword.
	\item Your student should explain how to use the delete menu and what items can be deleted in this manner.
\end{enumerate}

%%%%%%%%%%%%%%%%%%%%%%%%%%%%%%%%%%%%%%%%%%
%%%%%%%%%%%%%%%%%%%%%%%%%%%%%%%%%%%%%%%%%%
%                                        %
%              Lesson 5                  %
%                                        %
%%%%%%%%%%%%%%%%%%%%%%%%%%%%%%%%%%%%%%%%%%
%%%%%%%%%%%%%%%%%%%%%%%%%%%%%%%%%%%%%%%%%%
\clearpage
\section{ Email}
\
Skills Addressed in this Lesson:
\begin{itemize}
	\item Connect to a server.
	\item Read new mail messages.
	\item Put unwanted messages in trash folder.
	\item Complete and understand fields required for creating an e-mail message.
	\item Write a simple e-mail message.
	\item Edit e-mail message as needed.
	\item Send e-mail message when ready.
	\item Create a new e-mail folder.
	\item Exit e-mail.
	\item Send messages and empty trash as required.
\end{itemize}
\subsection{Tasks}
This lesson assumes that your student has an up and running e-mail account and the service has already been added to the BrailleNote. This lesson assumes that the teacher is familiar with hooking the BrailleNote to the Internet and that this has been previously set up.
\begin{enumerate}
	\item From the Main Menu, open email application  \dotfill {\textcolor{accent}{\MakeUppercase{\textbf{letter e}}}}
	\item When in Keymail, cycle menu options:  \dotfill {\textcolor{accent}{\MakeUppercase{\textbf{space}}}} 
	    \begin{itemize}
	        \item write an e-mail
	        \item read e-mail
	        \item connect to a service
	        \item setup options
	   \end{itemize}
	\item At the connect to a service prompt, when service name is displayed  \dotfill {\textcolor{accent}{\MakeUppercase{\textbf{enter}}}}
	\item Access ``read email'' option  \dotfill {\textcolor{accent}{\MakeUppercase{\textbf{space}}}} or {\textcolor{accent}{\MakeUppercase{\textbf{advance thumb-key}}}}
	\item Toggle options to access the inbox  \dotfill {\textcolor{accent}{\MakeUppercase{\textbf{space, enter}}}} 
	\item At the cue to check for new messages  \dotfill {\textcolor{accent}{\MakeUppercase{\textbf{enter}}}}
	\item When in an e-mail message:
	    \begin{itemize}
	        \item move from field to field  \dotfill {\textcolor{accent}{\MakeUppercase{\textbf{space}}}}
	        \item Move back by field  \dotfill {\textcolor{accent}{\MakeUppercase{\textbf{Backspace will move back by field}}}}  
	        \begin{itemize}
	            \item subject
	            \item date
	            \item e-mail address
	            \item etc
	        \end{itemize}
	    \end{itemize}
	\item To move to the next e-mail message \dotfill {\textcolor{accent}{\MakeUppercase{\textbf{space + dots56}}}} 
	\item Move to previous email \dotfill {\textcolor{accent}{\MakeUppercase{\textbf{space + dots23}}}}
	\item To put an unwanted message in the trash  \dotfill {\textcolor{accent}{\MakeUppercase{\textbf{space + dots2356}}}}
	\item To read e-mail message  \dotfill {\textcolor{accent}{\MakeUppercase{\textbf{enter}}}}
	\item Read the e-mail message with the following commands:
	      \begin{itemize}
		      \item Go forward reading automatically \dotfill {\textcolor{accent}{\MakeUppercase{\textbf{space + g}}}}
		      \item To move back a character  \dotfill {\textcolor{accent}{\MakeUppercase{\textbf{space + dot3}}}}
		      \item To read the current character  \dotfill {\textcolor{accent}{\MakeUppercase{\textbf{space + dots36}}}}
		      \item To move forward by character \dotfill {\textcolor{accent}{\MakeUppercase{\textbf{space + dot6}}}}
		      \item To move to the top of the message  \dotfill {\textcolor{accent}{\MakeUppercase{\textbf{space + dots123}}}} 
		      \item Move to the end of message  \dotfill {\textcolor{accent}{\MakeUppercase{\textbf{space + dots456}}}}
		      \item Move back a word \dotfill {\textcolor{accent}{\MakeUppercase{\textbf{space + dot2}}}} 
		      \item Read current word  \dotfill {\textcolor{accent}{\MakeUppercase{\textbf{space + dots25}}}}  
		      \item Read forward by word  \dotfill {\textcolor{accent}{\MakeUppercase{\textbf{space + dot5}}}}
		      \item Read back a sentence  \dotfill {\textcolor{accent}{\MakeUppercase{\textbf{space + dot1}}}}
		      \item Read the current sentence  \dotfill {\textcolor{accent}{\MakeUppercase{\textbf{space + dots14}}}} 
		      \item Read forward by sentence  \dotfill {\textcolor{accent}{\MakeUppercase{\textbf{space + dot4}}}}
		      \item To read back by paragraph  \dotfill {\textcolor{accent}{\MakeUppercase{\textbf{space + dots23}}}}  
		      \item To read the current paragraph  \dotfill {\textcolor{accent}{\MakeUppercase{\textbf{space + dots2356}}}}
		      \item To read forward by paragraph  \dotfill {\textcolor{accent}{\MakeUppercase{\textbf{space + dots56}}}}
		      \item Move Back or Forward one display length  \dotfill {\textcolor{accent}{\MakeUppercase{\textbf{advance thumb-key}}}} or {\textcolor{accent}{\MakeUppercase{\textbf{back thumb-key}}}}
		      \item Start braille display moving automatically  \dotfill {\textcolor{accent}{\MakeUppercase{\textbf{ space + dots23456 }}}}
		        \begin{itemize}
		      \item Control the speed of braille display \dotfill {\textcolor{accent}{\MakeUppercase{\textbf{previous thumb-key, next thumb-key}}}}
		      \item Stop the display from automatically moving \dotfill {\textcolor{accent}{\MakeUppercase{\textbf{previous thumb-key + next thumb-key}}}}
		        \end{itemize}
		      \end{itemize}
	\item When your student has finished reading the e-mail message, he or she may press space with e.  "Move this e-mail to which folder?" will be the prompt.
	\item To create a new e-mail folder from here, your student may type the name he or she wishes to use and press enter.  Respond with a y for yes at the "create a new folder" prompt.  The e-mail can then be moved to this folder.
	\item To select an existing folder to move the e-mail to, use space or the advance thumb-key to locate the desired folder.  Press enter to move the e-mail to the folder.
	\item When e-mail has been read, use space with e to return to the keymail menu options menu.  Locate "write an e-mail" and press enter.
	\item Your student will first need to fill out the Send To field.  The e-mail address of the person who you wish to e-mail will go here.  Computer Braille is required for e-mail addresses.  Do not use Braille contractions.  For the at sign, use space with u followed by the dot4. To move to the next field, use the advance thumb-key. A dot is written with dots4 6.
	\item If you wish to e-mail your message to someone else, fill out the "also send to" field with the e-mail address of another person.  Use the advance thumb-key to move to the next field.
	\item The "blind copy field" can also be filled out with an e-mail address.
	\item This copy will go to the recipient without anyone but you and the recipient knowing about it.  Explain this to your student and help them to use and understand the fields.
	\item The "attach a file" field is for attaching a file from the BrailleNote to an e-mail message that can then be opened and read by the recipient.
	\item Your student may write a subject in the subject field.  The subject can be one or two words that will summarize the content of the e-mail message.
	\item In the message area, write a simple e-mail message.  Edit the message as you wish before sending it.  A review of some editing commands is given below.
	      \begin{enumerate}
		      \item Delete to end of sentence by using backspace with dots1 and 4.
		      \item To delete the word under the cursor, use backspace with dots2 and 5.
		      \item Use backspace with dot2 to delete the word before the cursor.
		      \item To delete to the end of the paragraph, press backspace with dots2 3 5 6.
		      \item Use dots4 5 6 with backspace to delete from the cursor to the end of the message.
		      \item Use the cursor keys and backspace to make corrections and insertions.
	      \end{enumerate}
	\item When you and your student are satisfied with the message, use e with space to begin the process of sending an e-mail.  Press y when asked if you are ready to send the message.
	\item When asked if you wish to save a copy, press n for no.  If you key in y for yes, you will be prompted to save the e-mail in the desired folder of your student's choice.
	\item To exit Keymail, press space with e.  To send the mail your student wrote, he or she should press e with space and y for yes when prompted to do so.  Remember, if you write an e-mail when your BrailleNote is not connected to the internet, you will not be able to send e-mail messages immediately.  Your mail will be sent the next time that you connect to the internet.
	\item You and your student will also be prompted as to whether or not the e-mail in the trash folder should be emptied.  Key in a y for yes if you would like the trash to be emptied.
	\item At this point you and your student should be placed back at the Main Menu.  As always, turn the BrailleNote off when work tasks are completed.
\end{enumerate}
\clearpage

\subsection{Quiz}
\begin{enumerate}
	\item Your student should now use the network card to connect the BrailleNote to the internet connection.
	\item Ask your student to demonstrate competence by opening the Keymail menu, connecting to the server, and checking e-mail messages.  Your student should do this with the speech on so that the instructor can observe and listen to what is going on.  Otherwise, use a VGA cable to connect the BrailleNote to a monitor. Listen or watch while your student writes and sends an e-mail message.
	\item As appropriate, your student should move messages to the trash or other folders.
\end{enumerate}

True or False:
\begin{enumerate}
	\item When reading, space with dot2 will move back by character.
	\item Space with dots1 and 4 will delete to the end of the sentence.
	\item Use space with g to go forward reading.
	\item Backspace with dots2 and 5 will delete the word under the cursor.
	\item Use space with dot3 to move back by a word.
	\item Use backspace with dot2 to delete the word before the cursor.
	\item Backspace with g will stop the display from moving automatically.
	\item The back and advance thumb-keys will move the display back and forwardby a width of the Braille display.
	\item Space with dots2 3 4 5 6 will start the display moving automatically.
	\item A new folder cannot be added to the existing e-mail folders.
\end{enumerate}
%%%%%%%%%%%%%%%%%%%%%%%%%%%%%%%%%%%%%%%%%%
%%%%%%%%%%%%%%%%%%%%%%%%%%%%%%%%%%%%%%%%%%
%                                        %
%              Lesson 6                  %
%                                        %
%%%%%%%%%%%%%%%%%%%%%%%%%%%%%%%%%%%%%%%%%%
%%%%%%%%%%%%%%%%%%%%%%%%%%%%%%%%%%%%%%%%%%
\clearpage
\section{ Spell Check}
\
Skills Addressed in this Lesson:
\begin{itemize}
	\item Access the spelling checker from within Keyword or Keymail.
	\item Check a document for misspelled words.
	\item Look up a word to determine how it is spelled.
	\item Check a word to see if it is spelled correctly.
	\item Determine if there are misspelled words in a paragraph or a section.
	\item Check for spelling errors from the cursor to the end of a document.
\end{itemize}

\subsection{Tasks}
\begin{enumerate}
	\item Request that the student enter the Homework or other folder and then open or create a journal file.
	\item Your student may go to the end of the document by using a space with dots4 5 6.
	\item Write the date followed by two enters for two new lines.
	\item Make a journal entry.  Do not allow your student to make corrections as they occur so that practice with the spelling checker can be gained.
	\item In the document, your student may locate a word that is not spelled correctly.  If all words appear to be correct, change and misspell some of the words for your student.
	\item Your student should now place the cursor on a misspelled word.  Use dots1 and 6 with space to enter the spelling checker.
	\item Space or use the advance thumb-key to get to the word check option.  Press enter.
	\item If a word cannot be found or you need to fix a word, use the spell checker to do so.  The following options are available when trying to fix a misspelled word:
	      \begin{enumerate}
		      \item Use i to ignore all occurrences of the word.
		      \item Space with dot5 will fix only the occurrence of the word that your student is currently working with.
		      \item The letter a will add the word to the spelling dictionary.
		      \item The letter c will help correct the word.
		      \item To look up a word, press l.
		      \item For suggestions, press s.
		      \item Practice checking words at this time until you and your student adjust to using the word check option.
	      \end{enumerate}
	\item If your student is unsure of the correct spelling of a word or unsure of the correct Braille sign, use the "lookup word" option in the spelling checker.  Press dots1 and 6 to enter the spelling checker.  Press space to get to the "lookup word" option.  Press enter.
	\item At the word to look up cue, write a word in grade 1 Braille and press enter.
	\item If the word that is found is the one your student wants, he or she should press enter to insert it into the document.  To move up and down the list of word suggestions, use backspace or space.  You may also use the back and advance thumb-keys.  Press enter on a word if you choose to insert it into a document.  When finished, press e with space to exit the spelling checker.
	\item Practice using the other spelling checker options with your student.  Using the BrailleNote is the best way to get to know it, so practice and explore as time allows.  Spelling checker options include document check, look up word, word check, paragraph or section check, and check document from the cursor to the end.
	\item Remember; use space with h to get help when you need it.
\end{enumerate}
\clearpage
\subsection{Quiz}
\begin{enumerate}
	\item Write two or three paragraphs for your student.  Misspell many words.  Instruct your student to use the spelling checker to fix the paragraph.
\end{enumerate}

Multiple Choice:
\begin{enumerate}
	\item Which of the following is not an option in the spelling checker?
	      \begin{enumerate}
		      \item document check
		      \item word check
		      \item paragraph or section check
		      \item sentence or line check
	      \end{enumerate}
	\item Which dots are used with the space to enter the spelling checker?
	      \begin{enumerate}
		      \item dots1 and 4
		      \item dots1 and 6
		      \item dots3 and 4
		      \item dots2 and 6
	      \end{enumerate}
	\item In the spelling checker, what will space with dot5 do?
	      \begin{enumerate}
		      \item get suggestions
		      \item change the current occurrence of the word
		      \item add the word to the dictionary
		      \item change all occurrences of the word
	      \end{enumerate}
	\item To get help from anywhere when using the BrailleNote, what should you do?
	      \begin{enumerate}
		      \item use backspace with h
		      \item use enter with h
		      \item use space with dots2 3 5
		      \item use space with h
	      \end{enumerate}
	\item What operation will space with a full cell complete?
	      \begin{enumerate}
		      \item return to the internet
		      \item return to the Main Menu
		      \item reset the BrailleNote
		      \item both b and c
	      \end{enumerate}
	\item Which thumb-key is the first thumb-key on the left?
	      \begin{enumerate}
		      \item previous
		      \item back
		      \item advance
		      \item next
	      \end{enumerate}
	\item Backspace with dots1 and 4 in a keyword document will delete what?
	      \begin{enumerate}
		      \item the current sentence
		      \item the previous sentence
		      \item to the end of a sentence
		      \item the next sentence
	      \end{enumerate}
	\item In an e-mail message list, pressing space with dots5 and 6 will complete which action?
	      \begin{enumerate}
		      \item move to next field
		      \item move to the previous field
		      \item move to the next e-mail message
		      \item move to the create a file option
	      \end{enumerate}
	\item In menu options, which key will move to the next menu item?
	      \begin{enumerate}
		      \item space
		      \item advance
		      \item enter
		      \item both a and b
	      \end{enumerate}
	\item To turn the speech on or off, use which command?
	      \begin{enumerate}
		      \item next thumb-key with space
		      \item advance thumb-key with space
		      \item back thumb-key with space
		      \item previous thumb-key with space
	      \end{enumerate}
\end{enumerate}

%%%%%%%%%%%%%%%%%%%%%%%%%%%%%%%%%%%%%%%%%%
%%%%%%%%%%%%%%%%%%%%%%%%%%%%%%%%%%%%%%%%%%
%                                        %
%              Lesson 7                  %
%                                        %
%%%%%%%%%%%%%%%%%%%%%%%%%%%%%%%%%%%%%%%%%%
%%%%%%%%%%%%%%%%%%%%%%%%%%%%%%%%%%%%%%%%%%
\clearpage
\section{ User's Guide}
\
Skills Addressed in this Lesson:
\begin{itemize}
	\item Enter the User Guide through the Options Menu.
	\item Read the table of contents within the User Guide.
	\item From within the User Guide, enter the index.
	\item Access an item from the table of contents.
	\item Access an item from the index options.
	\item Read the items in the User Guide as desired.
	\item Follow the BrailleNote prompts in the User Guide as necessary.
	\item Exit the User Guide.
\end{itemize}
\subsection{Tasks}
\begin{enumerate}
	\item Instruct your student to enter the User Guide.  He or she should do this by first going to the options menu with space and o.  Then press a u to enter the User Guide.
	\item Your student may be given a cue such as "continue reading alarm?" Press an n for no.
	\item A cue such as "press t for table of contents or i for index" should be displayed.  Your student should press an i for index.
	\item Read through the items in the index by using space.  Use backspace to move back by item.  The back and advance thumb-keys will also move back and forward by index item.
	\item Explore the index items.  The initial letter of an item can be used to jump directly to an index item.
	\item Move to the index item called "battery" and press enter.  The following commands can be used to read when in the User Guide index.  Practice reading about the battery.
	      \begin{enumerate}
		      \item To start auto-advance, press space with dots1 2 4 5 6.
		      \item To start continuous speech, press space with g.
		      \item To return to the index, press backspace.
		      \item Space with e will exit the User Guide.
		      \item To move by subsection, press space with dots4 5 6 or space with dots1 2 3.
		      \item Space with dot1 and space with dot4 will move by sentence.
		      \item To move by paragraph, use space with dots5 6 and space with dots2 3.
	      \end{enumerate}
	\item Exit the User Guide.  Return to the Main Menu.
	\item From Main Menu, open the User Guide.  At the cue, continue reading about the battery, press n for no.
	\item At the cue, press t for table of contents.  Use the commands below to read through the table of contents items.
	      \begin{enumerate}
		      \item Space or backspace will move forward and back by item.
		      \item Space with e will return to the table of contents.
		      \item Enter on the item you wish to read.
	      \end{enumerate}
	\item Go to the word processor chapter and press enter.  Sub-index items will then be given.
	\item Move to and from sub-indexes in the usual manner.  Press enter on a sub-index item you would like to read.
	\item Your student may use the same commands to read the table of contents as he or she did to read the index.
	\item When your student is finished, return to the Main Menu and turn the BrailleNote off.  Provide time for practice and exploration within the User Guide because it's a great way to learn!
\end{enumerate}
\clearpage

\subsection{Quiz}
Copy each sentence.  Fill in the blank with the correct answer. Use the BrailleNote to complete the written work.
\begin{enumerate}
	\item To start \_\_\_\_\_ press space with dots1 2 4 5 6.
	\item Enter the User Guide through the \_\_\_\_\_ menu.
	\item To move by \_\_\_\_\_ use space with dots2 3 and space with dots5 6.
	\item Space with \_\_\_\_\_ will exit the User Guide.
	\item In the User Guide you can read by \_\_\_\_\_ or \_\_\_\_\_.
	\item Press \_\_\_\_\_ to go into an index item.
	\item To get help in the User Guide, press \_\_\_\_\_.
	\item To enter the User Guide, press o with space followed by \_\_\_\_\_.
	\item Press space with dot1 or dot4 to move back or forward by \_\_\_\_\_.
	\item \_\_\_\_\_  is the last item in the User Guide index.
\end{enumerate}

%%%%%%%%%%%%%%%%%%%%%%%%%%%%%%%%%%%%%%%%%%
%%%%%%%%%%%%%%%%%%%%%%%%%%%%%%%%%%%%%%%%%%
%                                        %
%              Lesson 8                  %
%                                        %
%%%%%%%%%%%%%%%%%%%%%%%%%%%%%%%%%%%%%%%%%%
%%%%%%%%%%%%%%%%%%%%%%%%%%%%%%%%%%%%%%%%%%
\clearpage
\section{ Address Book}
\
Skills Addressed in this Lesson:
\begin{itemize}
	\item Enter the Address Book.
	\item Add an address.
	\item Look up an address.
	\item Move to next and previous address records using space with dots2 3 or space with dots5 6.
	\item Move to previous and next fields with space and backspace.
	\item Read all fields of address records by using space with dots2 3 5 6.
	\item Delete a record from the Address Book using the backspace with dots2 3 5 6.
	\item Change a field or add a field to an existing record using backspace with dots1 4.
	\item List or be aware of the various fields of an address record.
\end{itemize}
\subsection{Tasks}
\begin{enumerate}
	\item Instruct your student to start at the Main Menu of the BrailleNote. Press an "a" to open the keylist menu.
	\item Use space or the advance thumb-key to explore the areas of the keylist menu.  These choices include add address, look up address, copy addresses, emboss addresses, print addresses, select keylist file, and free database space.
	\item Instruct your student to go to the add address option and press enter.
	\item "Entry list for address list" is what will be displayed and spoken.
	\item Use "space or backspace" or "previous or advance" to move from field to field.  Think of a friend or family member.  Add an address to the address book.
	\item If you know the information for the fields, enter it.  If you are instructed to do so, use computer Braille.  Computer Braille uses no contractions.  The at symbol is entered by using space with dots1 3 6 followed by dot4. Address fields include the following:
	      \begin{enumerate}
		      \item Last name
		      \item First name
		      \item Middle name
		      \item Title
		      \item Home phone number
		      \item Business phone number
		      \item Cell phone number
		      \item Home e-mail address
		      \item Business e-mail address
		      \item Home fax number
		      \item Business fax number
		      \item Street address
		      \item City
		      \item State
		      \item Zip or postal code
		      \item State
		      \item Country
		      \item Business title
		      \item Company name
		      \item Department
		      \item Business street address
		      \item Business state
		      \item Business zip
		      \item Business country
		      \item Web page
		      \item Notes
	      \end{enumerate}
	\item When your student has entered information in the appropriate fields, he or she can press e with space. When prompted to "add record?" key in y and press enter.
	\item Press space with e to exit the address book.
	\item Now return to the address book.  Space to the "look up address" option, and then press enter.
	\item Your student will read "selection list for address list" on the Braille display.  Key in a choice.  A last name, first name, or even title can be used to look up an address.  Press enter after placing information in the desired field.
	\item If there is only one matching record, it will be displayed.  Space, backspace, next thumb-key, or advance thumb-key can be used to read the various fields.  To read all fields of a record, press space with dots2 3 5 6.  To change or add a field, use backspace with dots1 and 4.
	\item Practice adding address records at this time.
	\item When more than one record is added, space with dots2 3 and space with dots5 6 can be used to move from record to record.
	\item If necessary, delete an address record by using space with dots2 3 5 6 when you are focused on the record that you no longer want.
	\item Help your student to use the User Guide to find more information about the Address Book.
	      \begin{enumerate}
		      \item Press o with space, then u to open the User Guide.
		      \item Press t for table of contents or i for index.
		      \item Then use space with h to find out how to read the index or the table of contents.
	      \end{enumerate}
\end{enumerate}

\clearpage
\subsection{Quiz}
Observe your student while he or she is engaged in using the Address Book.
\begin{enumerate}
	\item Go to the address book from the Main Menu.
	\item Add an address.  Write in at least 6 fields.
	\item Add the record.
	\item Read the fields of the record.
	\item Return to the Main Menu.
	\item Go to the User Guide.
	\item Read information about the address list from the User Guide.  Then return to the Main Menu.
\end{enumerate}

Use at least two complete sentences to answer the question:
\begin{itemize}
	\item What is the difference between an address record and an address field?
\end{itemize}

%%%%%%%%%%%%%%%%%%%%%%%%%%%%%%%%%%%%%%%%%%
%%%%%%%%%%%%%%%%%%%%%%%%%%%%%%%%%%%%%%%%%%
%                                        %
%              Lesson 9                  %
%                                        %
%%%%%%%%%%%%%%%%%%%%%%%%%%%%%%%%%%%%%%%%%%
%%%%%%%%%%%%%%%%%%%%%%%%%%%%%%%%%%%%%%%%%%
\clearpage
\section{ Email}
\
Skills Addressed in this Lesson:
\begin{itemize}
	\item Send an attachment.
	\item Use the address selection list to attain a recipient e-mail address.
	\item Access and understand the e-mail action menu.
\end{itemize}
\subsection{Tasks}
This lesson assumes that you have addresses in your BrailleNote address book that contain e-mail addresses.
\begin{enumerate}
	\item Instruct your student to open Keymail.
	\item Space or use the next thumb-key to locate "write an e-mail".  Press enter.
	\item At the "send to" prompt, press backspace with dots1 2 3 to get to the "selection list for address" prompt.
	\item Your student should move to the field of his or her choice.  For this lesson, first or last name will probably be best.  On the desired field, write the name of the recipient you want to e-mail and press enter.
	\item ``One record selected. Send to all recipients?" will be the cue.  Press a y for yes.  At this time you and your student will be taken to the "also send to" field.
	\item If you would like your student to send the message to someone else, either write the e-mail address or use backspace with dots1 2 3 to go to the address selection option.
	\item The next step to assist your student with is to move through the e-mail fields.  At the subject field, your student should get in the habit of writing a one or two word descriptive subject.  Then use the next thumb-key to get to the attachment field.
	\item The BrailleNote user is then given the question, "attach a file to this e-mail?" Your student should press a y for yes to gain practice in attaching a file to an e-mail message.
	\item The next prompt is "folder name".  Use space or the next thumb-key to go to the folder where there is a keyword file stored that you can attach for practice. Press enter.
	\item ``File to attach?" is the next question.  Use the next thumb-key or space to the file you want to attach and press enter.
	\item ``Attach this keyword document in another file type?" is the next cue. Key in a y for yes.  Microsoft Word is a common file type that most computer users can access.  For now let's attach the file as a Microsoft Word document.
	\item ``Attach as which file type?" is the next question.  Your display may already say "Microsoft Word." Press enter to attach the file as a Microsoft Word document.
	\item Now your student will be asked if he or she would like to attach another file.  Write an n for no at this point.
	\item Your student will find the e-mail message.  He or she may write a message.  A note about the file that is attached and the reason for it would be very appropriate and polite.  When your student is finished writing the message, he or she may press space with e.
	\item ``Ready to send this message?" If the message is good to go, your student should press y.  "Save a copy?" can probably be responded to with an n for no at this time.  As you leave Keymail, you will be prompted to send the message.  Key in a y for yes or n for no.
	\item Now it's time for your student and you to learn about the e-mail action menu.  This feature comes in handy in cases where the BrailleNote user wants to forward an e-mail message, reply to an e-mail message, or move an e-mail message to a different folder.
	\item When you are focused on or in an e-mail message, you can access the e-mail action menu by pressing space with dots2 and 6.  "E-mail action menu" will be spoken or displayed.  The choices in the e-mail action menu are:
	      \begin{itemize}
		      \item Reply
		      \item Forward
		      \item Move e-mail to another folder
		      \item Copy this e-mail to another folder
		      \item Delete this e-mail from this folder
		      \item Print this e-mail
		      \item Emboss this e-mail
		      \item All. Mark all e-mail
	      \end{itemize}
	\item Encourage your student to access e-mail through the BrailleNote in order to use the options within the e-mail action menu.  Enter on the choice that you want to open.
	\item Return to the Main Menu at this time.
\end{enumerate}
\clearpage
\subsection{Quiz}
Instruct your student to make up a question for each of the answers written below.
\begin{enumerate}
	\item Space with dots2 and 6
	\item Access the address list from the to field of an e-mail
	\item Press an e
	\item Forward and reply
	\item Use space with the next thumb-key
\end{enumerate}

%%%%%%%%%%%%%%%%%%%%%%%%%%%%%%%%%%%%%%%%%%
%%%%%%%%%%%%%%%%%%%%%%%%%%%%%%%%%%%%%%%%%%
%                                        %
%              Lesson 10                  %
%                                        %
%%%%%%%%%%%%%%%%%%%%%%%%%%%%%%%%%%%%%%%%%%
%%%%%%%%%%%%%%%%%%%%%%%%%%%%%%%%%%%%%%%%%%
\clearpage
\section{ Planner}
\
Skills Addressed in this Lesson:
\begin{itemize}
	\item Open the planner.
	\item Navigate around the planner.
	\item Schedule a new appointment.
	\item Reschedule an appointment.
	\item Delete an appointment.
	\item Set or clear an alarm.
\end{itemize}
\subsection{Tasks}
\begin{enumerate}
	\item From the Main Menu, instruct your student to press a p to move directly to the KeyPlan menu.  Enter with backspace with p will also open the planner.
	\item Space to the open planner option and press enter.
	\item Now you and your student will be placed in the planner.  Hopefully you will be on today's date.
	\item Use the commands listed below to navigate around the planner.  Practice using them with your student.
	      \begin{enumerate}
		      \item Press space with dot3 or space with dot6 to move a day back and forward at a time.
		      \item Use space with dot2 or space with dot5 to move through the calendar by week.
		      \item In order to move by month, use space with dot1 or space with dot4.
		      \item Moving a year at a time can be accomplished by using space with dots2 3 or space with dots5 6.
	      \end{enumerate}
	\item Choose a date a month from now and set an appointment.  If you have a real appointment to make, use it for practice.  Your student may want to use the planner to write down when a homework assignment is due.
	\item First your student must select the date.  He or she can do this by moving through the calendar with the above commands to find the appropriate date. Press enter to select the date.
	\item There is another way to select the date.  Write the name of the date as number sign month, dumber sign day, and number sign year.  The month must be written as a two digit number (such as 09 for September.  In the calendar, just write a number sign to begin this process.  Press enter to select the date.
	\item Now your student is ready to schedule the appointment.  Press space with dots2 6.
	\item ``Appointment day?" will be the next question.  If the correct date is displayed, press enter.  Remind your student that space with h can be used at any time to get help.
	\item ``Appointment time?" will be the next cue for your student.  Type the time in 12 hour format as hours, colon, minutes, followed by a space, then am or pm.  Press enter to finish the task.
	\item ``Appointment title" is the next prompt.  Write a simple title such as, "science paper due." Press enter.
	\item The next question is, "Do you wish to set an alarm on this appointment?" Press y for yes or n for no.  Press y for yes to practice setting an alarm.
	\item ``Warning time?" may currently say 15 minutes.  To change the number of minutes before the appointment that the alarm will sound, type a number between 0 and 120.
	\item Respond to any additional prompts and your appointment will be set.
	\item Backspace with r can be used to reschedule an appointment.  With the cursor on the appointment you wish to reschedule, try backspace with r.  You will then be taken through the process of rescheduling an appointment.  Allow your student practice time if you think it is necessary.
	\item Backspace with dots1 and 4 will delete an appointment.  Your student may schedule and delete appointments for practice if you think it would be helpful in the learning process.
	\item On the specified date, you may use backspace with "a" to set or clear an alarm.  "An alarm is set, do you wish to clear it?" will be the cue.  Press y or n to walk through the steps.  If you press n for no, you will be taken to the minutes before the appointment that the alarm will sound.  Your student may change this for practice if you think it is necessary.
	\item When these skills have been practiced and mastered, stop the lesson and return to the Main Menu.
\end{enumerate}
\clearpage
\subsection{Quiz}
Give a short answer to each question below.
\begin{enumerate}
	\item How can you schedule a new appointment?
	\item In the planner, what will backspace with dots1 and 4 do?
	\item In the calendar, how can you move back by month?
	\item How can you clear an alarm?
	\item In the calendar, what will space with dot6 do?
	\item What is the command to reschedule an appointment?
	\item In the calendar, how can you move back by week?
	\item How can you move forward by year in the calendar?
\end{enumerate}
%%%%%%%%%%%%%%%%%%%%%%%%%%%%%%%%%%%%%%%%%%
%%%%%%%%%%%%%%%%%%%%%%%%%%%%%%%%%%%%%%%%%%
%                                        %
%              Lesson 11                  %
%                                        %
%%%%%%%%%%%%%%%%%%%%%%%%%%%%%%%%%%%%%%%%%%
%%%%%%%%%%%%%%%%%%%%%%%%%%%%%%%%%%%%%%%%%%
\clearpage
\section{ Internet}
\
Skills Addressed in this Lesson:
\begin{itemize}
	\item Access the Internet from the Main Menu.
	\item Type in and go to a specific web page.
	\item Read a web page.
	\item Move from link to link and open links when desired.
	\item Access and use input controls.
	\item Utilize the favorites list.
	\item Use the history list.
	\item Use computer Braille as necessary.
\end{itemize}

\subsection{Tasks}
Complete the tasks listed below with your student.  This lesson assumes that you have access to the Internet.
\begin{enumerate}
	\item Assist your student in removing the compact flash card from the back of the BrailleNote if there is one inserted.
	\item Help your student to insert the network card into the appropriate slot in the BrailleNote.
	\item From the Main Menu of the BrailleNote, access the internet by keying in the letter i. At this point the url (address) of a web page will be displayed.  By default, the web address of PulseData is displayed.
	\item The following tasks can be accomplished from the url address.  To learn these options listed for yourself, push space with h for help.
	      \begin{enumerate}
		      \item One option is to type in a url or a favorite name and press enter.  You will be taken to the web site that you typed in.
		      \item For a local file, type backspace, then the file name.
		      \item To access the favorite list and saved html files, press space.
		      \item Space with dots5 6 will access the browser history list.
		      \item Use computer Braille to type the name of a web address.  This means you may not use Braille contractions and you must use dots4 6 for a dot(period).
	      \end{enumerate}
	\item After you have pressed i from Main Menu to go to the internet, press enter to go to the web address displayed.
	\item You may need to press enter to access your internet service or space to it and press enter.
	\item Practice reading the web page you opened.  Use the following commands:
	      \begin{enumerate}
		      \item Press space with g to go forward reading.
		      \item Go to the top of the page when necessary with space and dots1 2 3.  Space with dots4 5 6 will go to the bottom of the page.
		      \item To find a text string, press space with f.
		      \item To move back a character and forward by character, use space with dot3 or space with dot6. Space with dots3 and 6 will read the current character.
		      \item To hear the current word, press space with dots2 5.  Space with dot2 or space with dot5 will move back or forward by word.
		      \item To read back and forward by sentence, use space with dot4 or space with dot1.
		      \item Space with dots2 3 will move back by paragraph.  Space with dots5 6 will move forward by paragraph.  Space with dots2 3 5 6 will read the current paragraph.
		      \item Space with m will determine the current reading mode.
		      \item Space with m repeated will change the current reading mode.
		      \item Enter with dots5 6 will move to the next text section.  Enter with dots2 3 will move to the previous text section.
		      \item Space with w will announce the key name.
		      \item To query the current cursor position, use space with dots1 5 6.
	      \end{enumerate}
	\item Practice with the Thumb-Key Commands
	      \begin{enumerate}
		      \item The advance thumb-key will advance the display by a width.  Use the back thumb-key to move the display back by a width of the Braille display.
		      \item Use space with dots1 2 4 5 6 to start the Braille display moving automatically.  Use the previous or next thumb-keys to control the speed.  To stop, press previous and next together.
		      \item Previous with back will move the display back a word.  Previous with advance will move the display on a word.
		      \item The previous thumb-key with space will change speech to on or off.
		      \item The next thumb-key with space will change the Braille display mode.
		      \item Back with advance will route the cursor to the beginning of the Braille display.
		      \item To cycle through the four Braille display modes, press previous and next repeatedly.
		      \item The advance thumb-key with the next thumb-key repeatedly will cycle through the text file display options.
		      \item On the Pulse Data page, go to the top of the page by using space with dots1 2 3.
		      \item Now type a g for the Google search engine link.  Then press enter and you will go to the Google url.
		      \item Conduct a search for a specific topic of interest on the Google page. To get to the text input control, press space with dots4 and 6.  Use computer Braille to key in the name of a topic.  Press space with dots4 and 6 to move to the next input control, which should be the search button.
		      \item Press enter to begin the search.  You may also just press enter after you write in the search edit box.
		      \item Use space to move forward by link.  Move back by link by pressing backspace.  Explore the results of your search.  Press enter on a link to open the url. Use the web browser and miscellaneous commands listed here. Practice as necessary.
	      \end{enumerate}
	\item Web Browser and Miscellaneous Commands
	      \begin{enumerate}
		      \item Use enter with o to open a new url.
		      \item Space with i will display the name of the current web page.
		      \item Press enter to click a button.
		      \item Enter with f will go forward in a session history list.
		      \item Enter with b will go back a page in the session history list.
		      \item To refresh the current web page, press enter with the letter r.
		      \item To access the favorites menu, press enter with a low f (dots2 3 5).
		      \item Enter with m will go to the internet options menu.  From the internet options menu, you may:
		            \begin{enumerate}
			            \item open an html file
			            \item set the current page as the home page
			            \item go to your home page
			            \item manage internet files
			            \item go to the display settings
			            \item print the current web page
			            \item check the error status
			            \item Enter with lower h (dots2 3 6 will go to the history list).
			            \item Space with dots1 3 will go to the previous input control.  Space with dots4 and 6 will go to the next input control.  Space with dots1 3 4 6 will read the current input control.
			            \item Space with b will take you to the block commands menu.
			            \item To access the spelling checker, use space with dots1 6.
			            \item Space with s will save a file.
		            \end{enumerate}
	      \end{enumerate}
	\item Remember, at any time you can press the initial letter of a link to go directly to it.  Practice using only the basic commands at first.
\end{enumerate}
\clearpage
\subsection{Quiz}
Give a short answer to each question.
\begin{enumerate}
	\item What will space with dots4 and 6 do?
	\item Space with what will display the name of the current web page?
	\item How can you start the Braille display moving automatically?
	\item What will using the previous thumb-key with the space turn on and off?
	\item When using the Internet, enter with dots2 3 5 will do what?
	\item When on a web page, what can you use to move from link to link?
	\item What will enter with m bring up?
	\item When using computer Braille, which Braille dotsmake a period?
	\item When using the Internet, what will space with dots4 5 6 do?
	\item What will take you to the history list?
\end{enumerate}
%%%%%%%%%%%%%%%%%%%%%%%%%%%%%%%%%%%%%%%%%%
%%%%%%%%%%%%%%%%%%%%%%%%%%%%%%%%%%%%%%%%%%
%                                        %
%              Lesson 12                  %
%                                        %
%%%%%%%%%%%%%%%%%%%%%%%%%%%%%%%%%%%%%%%%%%
%%%%%%%%%%%%%%%%%%%%%%%%%%%%%%%%%%%%%%%%%%
\clearpage
\section{ Book Reader}
\
Skills Addressed in this Lesson:
\begin{itemize}
	\item Enter the book reader.
	\item Select a book to read.
	\item Use the appropriate commands required in reading a book.
	\item Explore the Bookshare site to locate books.
	\item Download books to the BrailleNote.
	\item Unpack books as necessary.
\end{itemize}
\subsection{Tasks}
This lesson assumes you are able to connect the BrailleNote to the Internet and that you have an account with Bookshare.
\begin{enumerate}
	\item First we need to go to the Bookshare site to download and unpack books. With the BrailleNote connected to the Internet, press an i from the Main Menu of the BrailleNote.
	\item At the address prompt, write the Bookshare web address and press enter.  Connect to a service by pressing enter at the cue.  You should be taken to the Bookshare web site.  Press space with dots4 6 to go to the first edit box where your e-mail address can be entered.
	\item As a reminder, if you get a security alert, press enter on the yes button.  This lesson may vary.
	\item At the e-mail text edit box, write in the e-mail address of your account.  Then use space with 4 6 again to go to the password edit box.  Write the password of your account.
	\item To get to the submit button, press space with 4 and 6 again.  Press enter.
	\item Press h to go to the home page link.  Press enter.  At this point, help your student explore the various categories of books.  Press a b to get to the "browse category" link and press enter.
	\item Press c to get to the ``categories$"$  link. At this point, press an "a" to get to the first category "animals." Then use space or backspace to explore the various categories.
	\item Categories include areas such as animals, biographies and memoirs, children's books, entertainment, horror, mystery and thrillers, poetry, sports, and teen.
	\item Here are the steps I followed in getting a book from the teen category.
	\item Follow a similar pattern to help your student find and download books of interest. This lesson may vary as updates to the site are made.
	      \begin{enumerate}
		      \item Press space or a t to locate the "teen" category.  Press enter to go to the teen page.
		      \item Space to the "skip to main content" link. Press enter. You will be placed in a table of book titles, authors, and book formats.
		      \item If you find a book you want to download, enter on brf or daisy to begin downloading the book.
		      \item To learn more about a book before downloading it, press enter on the book title.
		      \item "Download file into which folder?" will be the cue.  I have a folder called Bookshare, so I download all of my books in one folder.  Press enter at the appropriate folder.
		      \item At the "download file name?" prompt, press enter if the name is suitable for you.  You will be given a cue when the file is finished downloading.  Use the command to go back to the previous web page you were at. Get three or four books at this time if you wish.
	      \end{enumerate}
	\item Now your student or you need to unpack the books.  To do this, go to the book reader from the Main Menu by pressing a b.  To get to the book reader from anywhere, press a b with enter and backspace.
	\item ``Read book in which folder?" will be the cue.  Press enter on the bookshare folder, or whatever folder you have saved the book you want to unpack in. You may need to press space with x to get a list of all file types in order to find the book in the list.
	\item Space to the title of the book you want to unpack.  Press enter. "Folder for unpacked bookshare book?" will be the next cue.  Press enter on the appropriate folder.
	\item ``Password" is the next cue.  Write your Bookshare password using computer Braille and press enter.
	\item At the prompt, "delete the original packed bookshare book?" press a y for yes.  You will be told when the book is unpacked by the BrailleNote giving the title of the unpacked book.  Press enter at "review the options" to go into the unpacked book.  If you are asked if you want to delete the packed book, press y for yes if you want to delete it.
	\item To delete a book from folder, press backspace with dots1 4 when pointed to the book you want to delete.  At the prompt to double check for erasing the book, press a y for yes to confirm that you want to delete the book.
	\item Return to the Main Menu and turn the BrailleNote off.
\end{enumerate}
You will need keysoft version 7.5, build 31 (or higher) to use bookshare.
\clearpage
\subsection{Quiz}
True or False
\begin{enumerate}
	\item You do not need an account with Bookshare to download books from the Bookshare site.
	\item Press a space with b to enter the BrailleNote book reader from anywhere.
	\item To move to the next input control on a web page, use space with dots1 and 3.
	\item Fishing is one of the Bookshare categories.
	\item Bookshare only has newspapers and magazines to download.
	\item You do not need build 31 or higher to use bookshare.
	\item For reading Braille books on the BrailleNote, it is best to use the wav format.
	\item It is good to delete the packed copy of a book after you have unpacked it to save space on the BrailleNote.
	\item Once you have downloaded a book onto the BrailleNote, you cannot erase it.
	\item The book reader should be used for reading books.
\end{enumerate}
%%%%%%%%%%%%%%%%%%%%%%%%%%%%%%%%%%%%%%%%%%
%%%%%%%%%%%%%%%%%%%%%%%%%%%%%%%%%%%%%%%%%%
%                                        %
%              Lesson 13                  %
%                                        %
%%%%%%%%%%%%%%%%%%%%%%%%%%%%%%%%%%%%%%%%%%
%%%%%%%%%%%%%%%%%%%%%%%%%%%%%%%%%%%%%%%%%%
\clearpage
\section{ Scientific Calculator}
\
Skills Addressed in this Lesson:
\begin{itemize}
	\item Enter the BrailleNote scientific calculator from the Main Menu.
	\item Enter the scientific calculator by using the shortcut key.
	\item Complete division, multiplication, addition, and subtraction problems by using the calculator.
	\item Use decimal points if they are needed.
	\item Get help from within the scientific calculator.
	\item Insert a calculation into a word document.
	\item Quickly switch between the scientific calculator and a keyword document.
	\item Exit the calculator.
\end{itemize}
\subsection{Tasks}
Adjust this lesson according to the grade and ability level of your student.
\begin{enumerate}
	\item From the Main Menu, instruct your student to space until scientific calculator is spoken or displayed.  Press enter to open the calculator.
	\item On the Braille display, your student will feel the number 0.
	\item Use e with space to return to the Main Menu.  A full cell with space will also get you back to the Main Menu.
	\item Ask your student to explain another way in which the scientific calculator can be opened from the Main Menu.  Yes, press the letter s for scientific calculator to open the application.  When the calculator is open, do not exit to the Main Menu this time.
	\item Press backspace with enter with w to open the keyword application.
	\item Encourage your student to create a document in a folder called Math.
	\item Create a folder if necessary by writing the word math at the "folder to open" cue.  When asked if you want to create the folder, press y for yes.  Make a new file within the folder called "practice 1" for use during this lesson.  Again, just write the file name when prompted to do so.  Answer y for yes when asked if you want to create the file.  Give your student help as needed.
	\item At the top of the blank document, your student may write a heading.
	\item The first line may contain the student's name.  Follow this with a new line by pressing the enter key.
	\item Your student may write the date on the second line.
	\item The third line of the document can contain the assignment name "Math Practice." Notice that both words of the assignment name are capitalized.
	\item Many teachers require students to include a heading on assignments and tests.  Students who are blind or visually impaired should not be exempt from requirements that their nondisabled peers must abide by.
	\item Follow the heading with two new lines.  Your student should write 1 followed by a period and a space to get ready for the first math problem.
	\item The shortcut for opening the calculator is enter with backspace with
	\item s. Your student should do this now.
	\item In the calculator, you do not need to use the numeric indicator "the number sign." You or your student may use upper or lower case numbers.
	\item Nemeth Code users may want to practice dropping their numbers while using the scientific calculator to remain in the habit of using Braille math.
	\item Help your student complete a simple addition problem.  The addition sign is dots3 4 6.  To obtain a result, press the enter key.
	\item Now your student may clear the calculator by pushing backspace with dots1 4.
	\item Try a subtraction problem.  The subtraction sign is dots3 6.  Use a variety of numbers so that your student can practice inputting numbers into the scientific calculator.
	\item After clearing the result, try some multiplication problems.  The multiplication sign is dots1 6.
	\item The division sign is dots3 4.  If you need a decimal point, use dots4 6.
	\item At any time you may get extensive help from within the scientific calculator.  Press space with h.  Your student should try this now.
	\item In the calculator help menu, your student can space to the areas of General Commands, Basic Functions, Fractions, Memory Commands, Scientific Functions, and Statistics.
	\item Enter on any item to read more about it.  Your student should explore commands and information in the scientific calculator help menu to learn specific commands and tasks based on his or her grade or ability level. The BrailleNote gives terrific help from anywhere.  You and your student can learn an awesome amount of information by taking advantage of this feature from anywhere on the BrailleNote.
	\item Use e with space to go back by level until you get back to the scientific calculator.
	\item At the scientific calculator entry point, complete an addition problem, then get the result.  Do not clear the answer this time.
	\item Press enter with backspace with w to return to the document you created earlier in this lesson.
	\item Ensure that your cursor is placed after number 1.
	\item Insert your calculator result into your document.  Press enter with i.
	\item Then press c for calculation.
	      \begin{enumerate}
		      \item To insert the result, press r.  Press c to insert the whole calculation.
		      \item Did this work? Allow your student time to check.
		      \item Practice this skill.  Press enter with backspace with s to go back to the scientific calculator.  Clear your result.
		      \item Complete a new math problem.
		      \item Press enter with backspace with w to return to your document.
		      \item Use enter with i to insert the result.
		      \item Practice these skills until your student is able to demonstrate independence in using the scientific calculator for basic operations.  Also ensure that your student is able to independently insert calculations into keyword documents.
		      \item Requiring your student to exhibit the skill of switching quickly from a keyword document to the scientific calculator and back to a keyword document will ensure mastery of skills outlined in this lesson.
	      \end{enumerate}
\end{enumerate}
\clearpage
\subsection{Quiz}
Write or say the best answer to each question.
\begin{enumerate}
	\item What is the shortcut to get to the scientific calculator from within a keyword document?
	      \begin{enumerate}
		      \item enter with s
		      \item enter with backspace with c
		      \item enter with backspace with s
		      \item enter with c
	      \end{enumerate}
	\item Which dotsmake the minus sign?
	      \begin{enumerate}
		      \item dots3 4 6
		      \item dots4 6
		      \item dots1 6
		      \item dots3 6
	      \end{enumerate}
	\item How can you clear the calculator?
	      \begin{enumerate}
		      \item enter
		      \item backspace with c
		      \item backspace with s
		      \item space with c
	      \end{enumerate}
	\item From Main Menu, what letter should you press to get to the scientific calculator?
	      \begin{enumerate}
		      \item c
		      \item s
		      \item b
		      \item z
	      \end{enumerate}
	\item When using the calculator, what are dots4 6 used for?
	      \begin{enumerate}
		      \item the decimal point
		      \item the plus sign
		      \item the minus sign
		      \item the percent key
	      \end{enumerate}
	\item To get started with inserting a calculation, what key combination should you press?
	      \begin{enumerate}
		      \item enter with i
		      \item backspace with o
		      \item space with i
		      \item enter with x
	      \end{enumerate}
	\item How can you exit the calculator?
	      \begin{enumerate}
		      \item space with p
		      \item space with m
		      \item space with c
		      \item space with e
	      \end{enumerate}
	\item What will dots1 and 6 do within the calculator?
	      \begin{enumerate}
		      \item divide
		      \item multiply
		      \item add
		      \item subtract
	      \end{enumerate}
	\item To get help from within the calculator, what key combination should you press?
	      \begin{enumerate}
		      \item enter with h
		      \item backspace with h
		      \item space with h
		      \item none of the above
	      \end{enumerate}
	\item The BrailleNote calculator is not scientific.
	      \begin{enumerate}
		      \item true
		      \item false
	      \end{enumerate}
\end{enumerate}
%%%%%%%%%%%%%%%%%%%%%%%%%%%%%%%%%%%%%%%%%%
%%%%%%%%%%%%%%%%%%%%%%%%%%%%%%%%%%%%%%%%%%
%                                        %
%              Lesson 14                  %
%                                        %
%%%%%%%%%%%%%%%%%%%%%%%%%%%%%%%%%%%%%%%%%%
%%%%%%%%%%%%%%%%%%%%%%%%%%%%%%%%%%%%%%%%%%
\clearpage
\section{ Book Reader}
\
Skills Addressed in this Lesson:
\begin{itemize}
	\item Open the book reader.
	\item Open a book.
	\item Locate the beginning of a book.
	\item Open a second book using the shortcut command.
	\item Navigate through a book.
	\item Use the thumb-keys within a book.
\end{itemize}
\subsection{Tasks}
This lesson assumes you have downloaded at least two books from a web site such as Web Braille or Bookshare. Give assistance as necessary.
\begin{enumerate}
	\item From Main Menu, press b for book reader.
	\item The book reader will open.  You will be asked to open a folder, then to open a book.
	\item You may have to unpack a book if you have used Bookshare. You will need to know your Bookshare user name and password.
	\item From the top of the book, you may want to locate the beginning of the book content without reading all of the initial copyright information.
	\item You will need to use the find command.  Press space with f.
	\item Press f to search forward.  The letter b will search "back" if you needed to find something behind the cursor.
	\item Key in "isbn" for the isbn number of the book.  Press enter.  This will get you closer to the beginning of the book content.
	\item You could also key in the words "begin content" to get closer to the beginning of the book.  Use other words of your choice to find the beginning content of the book.
	\item From within the book, you can always use space with h.  When pressing space with h, you can space to the help topics of:
	      \begin{enumerate}
		      \item Review commands
		      \item Braille thumb-key commands
		      \item Miscellaneous commands
	      \end{enumerate}
	\item If you are reading with speech, you can press space with g to go forward reading.  Press enter with backspace to stop reading.
	\item Space with f will help you search for a string of text in the book. Use this command if you loose your place and need to locate it again.
	\item Use space with n to find the next occurrence of the last searched-for text string.  If you searched for the word "chapter" and did not find the correct chapter, you could use space with n to search for the word "chapter" again.
	\item Allow your student time to practice these commands.
	\item Other navigation commands that are similar to the navigation commands in a keyword document can also be used in the book reader.  Practice the below commands as necessary.
	      \begin{enumerate}
		      \item Space with 1 2 3 will go to the top of the document.
		      \item Go to the end of the document with space and dots4 5 6.
		      \item Move back by character with space and dot3. Read the current character with space and dots3 6.  Read the next character with space and dot6.
		      \item Read back by word with space and dot2. Read the current word with space and dots2 5.  Read the next word with space and dot5.
		      \item Space with dot1 or space with dot4 will read back and forward by sentence.  Space with dots1 4 will read the current sentence.
		      \item To read back and forward by paragraph, use space with dots2 3 or space with dots5 6.  Space with dots2 3 5 6 will read the current paragraph.
	      \end{enumerate}
	\item To be an efficient Braille reader, use the BrailleNote without speech.  Use the thumb-keys and the Braille display.  Practice the below commands as appropriate.
	      \begin{enumerate}
		      \item Use the back thumb-key to move back by one Braille display width.  Use the advance thumb-key to move forward by one Braille display width.
		      \item To start the display moving automatically, press space with dots1 2 4 5 6.  Press the previous thumb-key to slow down the speed of automatic reading.  The next thumb-key will increase the rate of automatic reading.
		      \item To stop automatic reading, pres the previous and next thumb-keys at the same time.
		      \item For more thumb-key commands, visit the help menu from within a book.
	      \end{enumerate}
	\item Other commands you and your student can practice within a book are:
	      \begin{enumerate}
		      \item To go to the block commands menu, press space with b.
		      \item To move to another book, press space with dots1 2 5 6.
	      \end{enumerate}
	\item Read a book from within the book reader.  Practice using the commands with your student that will be most useful to your given situation.
\end{enumerate}
\clearpage
\subsection{Quiz}
True or False:
\begin{enumerate}
	\item Space with dots2 3 4 5 6 will start the display moving automatically.
	\item Space with dots3 6 will read the current line.
	\item The previous thumb-key will slow the rate of automatic reading.
	\item Space with e or space with a full cell will return to the Main Menu.
	\item Space with 1 2 3 will go to the end of a book.
	\item Space with l will help you find the location of a text string.
	\item Space with 4 will move forward by sentence.
	\item B from the Main Menu will open the book reader.
	\item Space with n will go to the top of the file.
	\item Space with dot3 will move forward by character.
\end{enumerate}
%%%%%%%%%%%%%%%%%%%%%%%%%%%%%%%%%%%%%%%%%%
%%%%%%%%%%%%%%%%%%%%%%%%%%%%%%%%%%%%%%%%%%
%                                        %
%              Lesson 15                  %
%                                        %
%%%%%%%%%%%%%%%%%%%%%%%%%%%%%%%%%%%%%%%%%%
%%%%%%%%%%%%%%%%%%%%%%%%%%%%%%%%%%%%%%%%%%
\clearpage
\section{ Braille Options Menu}
\
Skills Addressed in this Lesson:
\begin{enumerate}
	\item Go to the Options Menu.
	\item From within the Options Menu, go to the Braille Options menu.
	\item Alter the Braille option settings as necessary.
	\item Return to Main Menu.
\end{enumerate}
\subsection{Tasks}
\begin{enumerate}
	\item From within a keyword document, space with o to enter the Options Menu.
	\item For practice, we will change the Braille Option settings.
	\item Once in the Options Menu, you may press b to open the Braille Options menu.
	\item The first option is Braille on or off.  To toggle between yes and no, press space with dots3 4.
	\item Space to the Braille Display Mode.  Automatic, reading, editing, and layout are the available modes.  Press enter to leave the item unchanged. Press the initial letter of each item to change to that mode.  Read more about each item below.
	      \begin{enumerate}
		      \item In automatic mode, the cursor is on when editing.  When reading, the cursor is off.
		      \item The letter r stands for reading mode.  In this mode, the cursor is not on.
		      \item E is for editing.  The cursor is always on and formatting is displayed.
		      \item In layout mode, the document is displayed as it would be if you embossed it.
	      \end{enumerate}
	\item By pressing the previous and next thumb-keys together, you can change the Braille display mode.  This is the shortcut key.
	\item When in the Options Menu, you or your student can space to the next item or press backspace to move to the previous item.
	\item "Show new lines in reading mode" is next in the Braille Options menu. Press space with dots3 4 to cycle between the available options.  Press space with h to get help.
	\item "Function of previous and next thumb-keys" is next.  Use s for sentence or line.  For paragraph or section, press p.  To have the previous and next thumb-keys move up and down, press u.
	\item Next you may change the cursor shape.  Use b for both dots7 and 8.  For only dot7, press 7.  Press 8 for only dot8. For an 8 dotcell, press w.  6 is for a six dotcell.
	\item Preferred reading grade is next.  This will determine the grade of Braille used on the display for messages, menus, etc.  Press 1 for grade I Braille, 2 for grade II Braille, and c for computer Braille.  Press enter to leave the option unchanged.
	\item Text document reading grade is next.  Press space with h to get help, then make your choice.  Experiment to see what is best for you.
	\item Preferred English Braille code is next.  Space with h will give you the various options.
	\item Computer Braille table is next.  Press space with dots3 and 4 to step through the available options or use the letter hotkey.  Press s for use computer Braille.  K is for UK Braille.  Read more in the User Guide if you need to use this option.
	\item Continue through the additional areas of the Braille Display Options. Space with h to get help when necessary.  The available options are listed below.
	      \begin{enumerate}
		      \item Cursor shape for computer Braille.
		      \item Braille display intensity.
		      \item Display computer Braille as 6 or 8 dots.
		      \item Message display time is next.
		      \item Thumb-key set is next.
	      \end{enumerate}
	\item Use the Options Menu and experiment to see what is best for you.  Explore the other available areas in the options menu.  They may vary according to where you were on the BrailleNote when you opened the Options Menu.
\end{enumerate}
\clearpage
\subsection{Quiz}
Write or say each answer.
\begin{enumerate}
	\item Space with what letter will take you to the options menu?
	\item In what menu can you change the Braille settings?
	\item When making a choice, what does dots3 4 do among menu options?
	\item In what mode is the cursor on when editing.
	\item What does space with h do in the Options Menu?
\end{enumerate}
%%%%%%%%%%%%%%%%%%%%%%%%%%%%%%%%%%%%%%%%%%
%%%%%%%%%%%%%%%%%%%%%%%%%%%%%%%%%%%%%%%%%%
%                                        %
%              Lesson 16                  %
%                                        %
%%%%%%%%%%%%%%%%%%%%%%%%%%%%%%%%%%%%%%%%%%
%%%%%%%%%%%%%%%%%%%%%%%%%%%%%%%%%%%%%%%%%%
\clearpage
\section{ Games}
\
Skills Addressed in this Lesson:
\begin{enumerate}
	\item Play a game on the BrailleNote.
	\item Get help within games as needed
	       \subsection{Tasks}
	      \begin{enumerate}
		      \item To get to the Games Menu, space to games or press g from Main Menu.  You may also press backspace with enter with g from anywhere on the BrailleNote to open the games.
		      \item "Play which game?" will be displayed when the Games option is open. Computer Braille is required.
		      \item Space to the tutorial game.  Press enter.
		      \item Use the thumb-keys to read with the Braille display.  This will be a good way for older students to practice Braille reading skills in a fun situation.
		      \item Read the instructions. Type in "about" to learn more about the game.  Use computer Braille.  Press enter.
		      \item Read about the game.
		      \item If you have a student who needs voice, turn on the voice with the previous thumb-key and space.  The speech will help you learn the game during the tutorial session.
		      \item Some game commands are listed below.  You may want to read them all through the Games Help Menu to see if there are others you want to use.
		            \begin{enumerate}
			            \item Type a single character to indicate your next move.
			            \item Backspace with s to review the current status.
			            \item Space with 5 6 and space with 2 3 will move through your move history list.
			            \item Press space with s to save the current game.
			            \item Press enter with o to open a game that you previously saved.
			            \item Backspace with r will insert the last entered move.
			            \item To move back by sentence, press space with dot1. Press space with dots1 4 to read the current sentence.  Press space with dot4 to move forward by sentence.
			            \item Backspace with q will abandon the current game.
		            \end{enumerate}
		      \item Here are some commands you will use during games.
		            \begin{enumerate}
			            \item Use l to look around.
			            \item Use i to obtain an inventory and you will get a closer look.
			            \item The letter n is for north.
			            \item s south
			            \item w west
			            \item e east
		            \end{enumerate}
		      \item Explore using the commands to play the game.  Use space with h to get help when necessary.  Try other games as time allows.
	      \end{enumerate}
	      \clearpage
	      \newpage
	      Lesson 16
	      Quiz
	      If the statement is true, write or say t.  Write or say f if the statement is false.
	      \begin{enumerate}
		      \item Use w to move west.
		      \item In a game, the letter l will help you look around.
		      \item Backspace with q will quit the current game.
		      \item Use backspace with s to review the current game status.
		      \item Enter with s will save the current game.
		      \item Backspace with r will insert the last entered move.
		      \item There is more than one method to open the games menu.
		      \item Press space with dot1 to move back by sentence.
		      \item Press space with o to open a previously saved game.
		      \item There are graphics in BrailleNote games.
	      \end{enumerate}
\end{enumerate}

%%%%%%%%%%%%%%%%%%%%%%%%%%%%%%%%%%%%%%%%%%
%%%%%%%%%%%%%%%%%%%%%%%%%%%%%%%%%%%%%%%%%%
%                                        %
%              Lesson 17                  %
%                                        %
%%%%%%%%%%%%%%%%%%%%%%%%%%%%%%%%%%%%%%%%%%
%%%%%%%%%%%%%%%%%%%%%%%%%%%%%%%%%%%%%%%%%%
\clearpage
\section{ Dictionary}
\
Skills Addressed in this Lesson:
\begin{itemize}
	\item Look up a word in the dictionary.
	\item Read through a list of suggested words.
	\item Read through a list of word definitions.
	\item Insert a word from the dictionary into a document.
\end{itemize}
\subsection{Tasks}
This lesson assumes that you have the Concise Oxford Dictionary loaded onto your BrailleNote.
\begin{enumerate}
	\item In a document, write a word you would like to look up in the dictionary.
	\item Place the cursor somewhere on the word.
	\item Press o with space, then l.  L is for look up word.
	\item To look up a word in the dictionary, press d.
	\item Press enter to look up the word that is displayed.
	\item To move forward through the list of suggestions, press space.  Backspace will move back through the suggestions list.  Press e with space to exit the list.
	\item If you find the word you want to look up, press enter on the word.
	\item Space with i will give you information about the word you are looking up.
	\item Enter with dots2 5 will pronounce the word you are looking up.
	\item Space with 2 3 will move back to the previous entry.  Space with dots5 6 will move to the next entry.
	\item If you want to insert the word you are looking up under the cursor, press backspace with i.
	\item Follow the above steps in order to explore looking up a word in the dictionary.  Get help when necessary with space with h.
	\item From within a document, press space with o, then l to look up a word.
	\item Type in a word you want to look up.  If you are unsure of a letter in the word, press a question mark with a space.  This is dots2 3 6 with space.
	\item Press enter to look up the word.  Use the commands previously listed to look up the word.
	\item Use backspace with k to copy a selection to the clipboard.
	\item Practice looking up words in the dictionary.
\end{enumerate}
\clearpage
\subsection{Quiz}
Answer each question with one of the two provided choices.
\begin{enumerate}
	\item What should you press to move forward through a list of suggestions?
	      space  backspace
	\item What with i will give you information about the word you are looking up?
	      enter    space
	\item Space with what will move back to the previous entry?
	      dots2 3   dots1 2
	\item Backspace with what letter will copy an entry to the clipboard?
	      C  k
	\item What with dots2 3 will move back to the previous entry in the list?
	      space   backspace
	\item If you find a word you want to look up, which key should you press to look up the word?
	      enter      space
	\item Enter with which dotswill pronounce the word you are looking up?
	      dots1 3   dots2 5
	\item If you are unsure of a letter in the word you are looking up, what can you press with space in place of the letter you do not know?
	      question mark    h
	\item What will pressing backspace with i do to the word you are looking up?
	      delete the word    insert the word
	\item Is it possible to read a full definition in the Concise Oxford Dictionary?
	yes  no
\end{enumerate}

%%%%%%%%%%%%%%%%%%%%%%%%%%%%%%%%%%%%%%%%%%
%%%%%%%%%%%%%%%%%%%%%%%%%%%%%%%%%%%%%%%%%%
%                                        %
%              Lesson 18                  %
%                                        %
%%%%%%%%%%%%%%%%%%%%%%%%%%%%%%%%%%%%%%%%%%
%%%%%%%%%%%%%%%%%%%%%%%%%%%%%%%%%%%%%%%%%%
\clearpage
\section{ Printing}
\
Skills Addressed in this Lesson:
\begin{itemize}
	\item Print a document
\end{itemize}
\subsection{Tasks}
\begin{enumerate}
	\item From Main Menu, press w to open the keyword menu.
	\item To print a document, press p for print.
	\item Press s for printer settings.
	\item Space to move through the list of printer setup options.
	\item Change settings as necessary.  However, most settings should be fine as they are.
	\item The printer port may need to be changed.  The port choices are bluetooth, infra-red, serial, USB, or file.  Press space with dots3 4 to toggle to the printer port you are using.
	\item Space to continue through the list of options.  You may need to change the printer type. If you do, press y for "yes" on this option.  Then press enter.
	\item Use the thumb-keys to space to the type of printer you have.  Press enter.
	\item Press e with space when you are finished changing the settings.  When asked to confirm the changes, type y for yes.
	\item Now from the keyword options menu, press p for print.  When asked if you want to print or setup the printer, press p for print.
	\item Enter on the folder and then the file that you want to print.
	\item Respond with y when asked if the printer is ready.
	\item Hopefully your document will print.
\end{enumerate}
\clearpage
\subsection{Quiz}
True or False:
\begin{enumerate}
	\item Bluetooth is an option for printing.
	\item The type of printer cannot be changed.
	\item Enter with o will take you to the options menu.
	\item Space with c will clear the calculator.
	\item You can print from within the keyword menu.
	\item Press b from Main Menu to get to the book reader.
	\item Enter with dot1 will make the voice volume softer.
	\item It is not possible to use the internet with the BrailleNote.
	\item The back and advance thumb-keys will move back and forward by line.
	\item Use backspace to delete the last character that was written.
\end{enumerate}

%%%%%%%%%%%%%%%%%%%%%%%%%%%%%%%%%%%%%%%%%%
%%%%%%%%%%%%%%%%%%%%%%%%%%%%%%%%%%%%%%%%%%
%                                        %
%              Lesson 19                  %
%                                        %
%%%%%%%%%%%%%%%%%%%%%%%%%%%%%%%%%%%%%%%%%%
%%%%%%%%%%%%%%%%%%%%%%%%%%%%%%%%%%%%%%%%%%
\clearpage
\section{ Embossing}
\
Skills Addressed in this Lesson:
\begin{itemize}
	\item Emboss a document
\end{itemize}
\subsection{Tasks}
\begin{enumerate}
	\item From Main Menu, press w to open the keyword menu.
	\item To emboss a document, press e for emboss.
	\item The first time you emboss, press s for embosser settings.
	\item Space to move through the list of embosser setup options.
	\item Change settings as necessary.  However, most settings should be fine as they are.
	\item The embosser port may need to be changed.  The port choices are:
	      \begin{enumerate}
		      \item Bluetooth
		      \item Infra-red
		      \item  Serial
		      \item USB
		      \item File
	      \end{enumerate}
	\item Press space with dots3 4 to toggle to the embosser port you are using.
	\item Space to continue through the list of options.  You may need to change the embosser type. If you do, press y for "yes" on this option.  Then press enter.
	\item Use the thumb-keys to move to the type of embosser you have.  Press enter. Press e with space when you are finished changing the settings.  When asked to confirm the changes, type y for yes.
	\item Now from the keyword options menu, press e for emboss.  When asked if you want to emboss or setup the embosser, press e for emboss.
	\item Enter on the folder and then file that you want to emboss.
	\item Respond with y when asked if the embosser is ready.
	\item Hopefully your document will emboss.
\end{enumerate}
\subsection{Extra Commands}
\begin{enumerate}
	\item Enter with d will speak the date.
	\item Enter with t will speak the time.
	\item Backspace of dots4 5 6 will delete to the end of the file.
	\item Use backspace with 1 4 to delete to the end of the current sentence.
	\item To open a program from anywhere, press backspace with enter with the initial letter of the program.  For example, to open the planner press enter with backspace with $ \string^ $ people.
	\item Space with a full cell will return to the main menu.
\end{enumerate}
\clearpage
\subsection{Quiz}
Write or say the letter of the best answer choice.
\begin{enumerate}
	\item To look up a word in the Oxford Dictionary, press o with space, then what letter?
	      \begin{enumerate} 
		      \item w
		      \item d
		      \item l
		      \item b
	      \end{enumerate}
	\item WHEN pressing d, what will you be looking up a word in?
	      \begin{enumerate}
		      \item thesaurus
		      \item dictionary
		      \item daily menu
		      \item help menu
	      \end{enumerate}
	\item How can you delete a character?
	      \begin{enumerate}
		      \item space
		      \item backspace
		      \item space with d
		      \item enter with d
	      \end{enumerate}
	\item What will space with dots1 2 3 do?
	      \begin{enumerate}
		      \item give you a file list
		      \item display a list of links
		      \item look up a sentence
		      \item go to the top of a file or file list
	      \end{enumerate}
	\item What will backspace of dots4 5 6 do?
	      \begin{enumerate}
		      \item delete to the end of the file
		      \item read to the end of the file
		      \item speak to the end of the file
		      \item skip to the end of the file
	      \end{enumerate}
	\item Enter with backspace with what letter will go to the scientific calculator.
	      \begin{enumerate}
		      \item c
		      \item s
		      \item g
		      \item h
	      \end{enumerate}
	\item To speak the time press enter with what letter?
	      \begin{enumerate}
		      \item d
		      \item r
		      \item t
		      \item u
	      \end{enumerate}
	\item To open the address manager, press backspace with enter with which letter?
	      \begin{enumerate}
		      \item a
		      \item m
		      \item l
		      \item n
	      \end{enumerate}
	\item From within the keyword menu, which letter or key should you press to emboss?
	      \begin{enumerate}
		      \item p
		      \item f
		      \item e
		      \item enter
	      \end{enumerate}
	\item What will space with a full cell do?
	      \begin{enumerate}
		      \item enter the planner
		      \item delete a document
		      \item go to the bottom of a file
		      \item return to main menu
	      \end{enumerate}
\end{enumerate}
%%%%%%%%%%%%%%%%%%%%%%%%%%%%%%%%%%%%%%%%%%
%%%%%%%%%%%%%%%%%%%%%%%%%%%%%%%%%%%%%%%%%%
%                                        %
%              Lesson 20                  %
%                                        %
%%%%%%%%%%%%%%%%%%%%%%%%%%%%%%%%%%%%%%%%%%
%%%%%%%%%%%%%%%%%%%%%%%%%%%%%%%%%%%%%%%%%%
\clearpage
\section{ Set up Bluetooth}
\
Skills Addressed in this Lesson:
\begin{itemize}
	\item Set up Bluetooth for embossing or printing
	\item Print a document with Bluetooth
	\item Emboss a document through Bluetooth
\end{itemize}

\subsection{Tasks}
\begin{enumerate}
	\item Plug your Bluetooth adapter into the printer or the embosser.  Plug the electric end into the wall outlet.
	\item Now match the Bluetooth adapter with the embosser or the printer.  Press
	\item with space to go to the options menu.  Press c for connect.
	\item Press b for Bluetooth.
	\item ``Bluetooth on" will be your cue.  Press y for yes.
	\item ``Search for device?" is your next prompt.  Press y for yes.
	\item The display will read, "Searching for device, please wait."
	\item You will get a list of Bluetooth devices that the BrailleNote found. Space to your device and press enter.
	\item You will then be given a list of services for the device.  Space to the service you want and press enter.  For example, space to "printer" and press enter.
	\item You will be asked if you want to pair with the device and activate the printer.  Press y to pair with and activate the printer.
	\item ``Authentication code" is the next cue.  Press enter.
	\item You should receive a message saying that the BrailleNote is paired with the printer or embosser.
	\item Return to the Main Menu.
	\item Press w to open the keyword menu.
	\item Press p for print or e for emboss.
	\item Press s to setup the embosser or printer.
	\item Space to the "printer port" option.  Press space with dots3 4 to toggle among the choices.  The port you want is Bluetooth.  Press e with space to exit when you are finished.
	\item When asked to confirm the changes, press y for yes.
	\item Now print or emboss the document.
	\item From the Keyword menu, press p for print or e for emboss.
	\item Answer with y when you are ready to print or to emboss.
\end{enumerate}
\clearpage

\subsection{Quiz}
Write or say the best answer for each item.
\begin{enumerate}
	\item How can you go forward by menu item?
	      \begin{enumerate}
		      \item backspace
		      \item space
		      \item enter
		      \item o with space
	      \end{enumerate}
	\item From Main Menu, what letter will open the keyword menu?
	      \begin{enumerate}
		      \item k
		      \item w
		      \item m
		      \item t
	      \end{enumerate}
	\item What combination is used to get to the connectivity menu?
	      \begin{enumerate}
		      \item o with space, b
		      \item o with space, c
		      \item o with space, w
		      \item o with space, g
	      \end{enumerate}
	\item Space with what letter will go forward reading?
	      \begin{enumerate}
		      \item r
		      \item f
		      \item g
		      \item o
	      \end{enumerate}
	\item What should you press to stop reading?
	      \begin{enumerate}
		      \item backspace with enter
		      \item backspace with space
		      \item previous and back
		      \item none of the above
	      \end{enumerate}
	\item You need to plug in the Bluetooth adapter to the BrailleNote.
	      \begin{enumerate}
		      \item true
		      \item false
	      \end{enumerate}
	\item The Bluetooth adapter does not need electricity.
	      \begin{enumerate}
		      \item true
		      \item false
	      \end{enumerate}
	\item What will space with a full cell do?
	      \begin{enumerate}
		      \item reset the BrailleNote
		      \item return to the keyword menu
		      \item turn the BrailleNote off
		      \item return to the Main Menu
	      \end{enumerate}
	\item To open the planner from the Main Menu, what should you press?
	      \begin{enumerate}
		      \item p
		      \item a
		      \item c
		      \item backspace
	      \end{enumerate}
	\item Which company produces the BrailleNote?
	      \begin{enumerate}
		      \item Dell
		      \item Freedom Scientific
		      \item GW Micro
		      \item HumanWare
	      \end{enumerate}
\end{enumerate}

%%%%%%%%%%%%%%%%%%%%%%%%%%%%%%%%%%%%%%%%%%
%%%%%%%%%%%%%%%%%%%%%%%%%%%%%%%%%%%%%%%%%%
%                                        %
%              Lesson 21                  %
%                                        %
%%%%%%%%%%%%%%%%%%%%%%%%%%%%%%%%%%%%%%%%%%
%%%%%%%%%%%%%%%%%%%%%%%%%%%%%%%%%%%%%%%%%%
\clearpage

\section{ Web Braille / BARD}
\


Skills Addressed in this Lesson:
\begin{itemize}
	\item Open the Web Braille web site.
	\item Log into Web Braille.
	\item Locate a book or magazine.
	\item Save a book or magazine.
\end{itemize}

\subsection{Tasks}
\begin{enumerate}
	\item Enter the Web Braille web site.
	\item You will need to log in.
	\item Use the thumb-keys to locate the name and password edit boxes.  Write in your name and password and press enter to log in.
	\item On the Web Braille page, space or backspace to explore the links.
	\item For this exercise, let's download a magazine.  Locate the Braille magazines link and press enter.
	\item Space and backspace through the links to locate a magazine that you are interested in.  Press enter.
	\item Now move to the link of the magazine issue that you would like to download.  Press enter.
	\item You will then be at the accept page.  You will need to say "yes" to agree to the Web Braille conditions.  Press space with dots4 6 to go to the "I Accept" button.  You need the one that is best for Note Taker users.  Press enter.
	\item You will then be on a page that you need to save.  Press space with s to save.
	\item Space to the folder that you want to save the document in.  Press enter to select the given folder.
	\item Name the magazine appropriately and press enter.
	\item Use enter with b to go back by web page.  Download another magazine if you wish.
	\item If you want to download book volumes, use the same process, but enter on the volume number links to save them.  After entering on the volume link you want, you will be taken to the accept page.  Press space with 4 6 to move to the "best for note taker" option and press enter.  Using the available Braille Book Review on the Web Braille site is a good place to start.  Open the URL format version of the Braille Book Review.  Then you may choose a category of books and will be able to locate books from that point.
	\item When pressing enter on a book volume, you will be taken to the next page.  Press space with s to save the book volume.
	\item Save the book in a folder where you can find and open it later.  Name the book volume so that you can also find it when you want to read and open it through the book reader on the BrailleNote.
	\item From Main Menu, press b for book reader.  Select the folder and book volume you want.  Press enter to open the book.
	\item You should be able to read the book as usual from here.
\end{enumerate}
\clearpage
\subsection{Quiz}
Write the letter of the best answer choice.
\begin{enumerate}
	\item To check the time, what should you press?
	      \begin{enumerate}
		      \item space with t
		      \item backspace with t
		      \item enter with t
		      \item t
	      \end{enumerate}
	\item How can you save a book or magazine in order to read it later?
	      \begin{enumerate}
		      \item enter with s
		      \item space with s
		      \item backspace with s
		      \item s
	      \end{enumerate}
	\item Which of the following items can you get on Web Braille?
	      \begin{enumerate}
		      \item books
		      \item magazines
		      \item just ``a$"$
		      \item both ``a$"$  and ``b$"$
	      \end{enumerate}
	\item What will o with space, then p do?
	      \begin{enumerate}
		      \item takes you to one-hand mode
		      \item turns Braille on and off
		      \item turns speech on or off
		      \item checks the power status
	      \end{enumerate}
	\item How can you toggle speech on and off?
	      \begin{enumerate}
		      \item previous thumb-key with space
		      \item next thumb-key with space
		      \item backspace and enter
		      \item space with backspace
	      \end{enumerate}
	\item How can you get to the accept button on the accept page in Web Braille?
	      \begin{enumerate}
		      \item l for link
		      \item press space with dots4 6
		      \item press l with space
		      \item none of the above
	      \end{enumerate}
	\item To get to the web browser, what letter should you press?
	      \begin{enumerate}
		      \item w
		      \item b
		      \item i
		      \item p
	      \end{enumerate}
	\item Where should you go to read a Web Braille book?
	      \begin{enumerate}
		      \item keyword
		      \item keyplan
		      \item bookreader
		      \item keylearn mode
	      \end{enumerate}
	\item Which of the following are sites you can get books from?
	      \begin{enumerate}
		      \item Web Braille
		      \item NPR
		      \item Bookshshare
		      \item both a and c
	      \end{enumerate}
	\item How can you move to the next input control on a web page?
	      \begin{enumerate}
		      \item space
		      \item backspace
		      \item space with 1 3
		      \item backspace with 1 2 3
	      \end{enumerate}
\end{enumerate}

%%%%%%%%%%%%%%%%%%%%%%%%%%%%%%%%%%%%%%%%%%
%%%%%%%%%%%%%%%%%%%%%%%%%%%%%%%%%%%%%%%%%%
%                                        %
%              Lesson 22                  %
%                                        %
%%%%%%%%%%%%%%%%%%%%%%%%%%%%%%%%%%%%%%%%%%
%%%%%%%%%%%%%%%%%%%%%%%%%%%%%%%%%%%%%%%%%%
\clearpage

\section{ File List}
\

Skills Addressed in this Lesson:
\begin{itemize}
	\item Get information about a file.
	\item View other document types.
	\item Determine the order in which files are displayed.
	\item Copy a file.
	\item Move a file.
	\item Rename a file.
	\item Protect or unprotect a file.
	\item Mark one file or all files.
\end{itemize}
\subsection{Tasks}
	      \begin{enumerate}
		      \item From a list of files, use space with h to hear all available commands.  These items will be reviewed in this lesson.
		      \item You can get information about the "pointed to" file in the file list within a folder.  Space with i will allow you to get details about the file.  You will learn what type of file you are focused on, the file size, when the file was last modified, and if the file is protected or unprotected.  Try using space with i when pointed to a variety of files to get an idea of what information you can get.
		      \item Use space with x to view other types of files.  If you want to view all files in the folder, use space with x until the View All Files option is displayed.  Space with x will toggle to the various file types.
		      \item You can determine the order in which files are displayed by using space with v.  Repeat space with v to change the order in which the files are viewed.  Space with v is a toggle.  The choices are name order, date order, size order, or type of file order.  Explore this option by using space with v to toggle to a different order type.  Look at your list of files to see how this impacts how your files are ordered.
		      \item To copy the pointed to file, press backspace with y.  Try this now.  You will need to choose the destination folder for the file to be placed in.  However, the file is just being copied.  The original will remain in the location it was copied from in the first place.
		      \item Backspace with m will "move" a file.  If you move a file, it will no longer remain in its original location.  Keep in mind that moving a file and copying a file are similar, but not quite the same.
		      \item You can rename the "pointed to" file.  Press enter with r, then type the new name.  Press enter when you finish.
		      \item Backspace with p will protect or unprotect a "pointed to" file.  When you use this command on a file in the list, you will then need to press p to protect the file, u to unprotect the file, or enter to leave the option unchanged.  If you protect a file, when you are actually in the file you will be unable to make changes to the file.
		      \item You can mark a file or mark all files.  Space with m will mark the pointed to file.  Enter with low g will mark all files.
		      \item Backspace with m will move marked files if you have any marked.  Otherwise, backspace with m will move the pointed to file.
		      \item You can copy the marked items or delete the marked items.  Backspace with y will copy the marked items.  Backspace with dots1 4 will delete the marked items.
		      \item As time, use the above commands to become more familiar with how efficient they can be when managing files.
	      \end{enumerate}
	      \clearpage

\subsection{Quiz}
	      Answer each question. Items are taken from this lesson only.
	      \begin{enumerate}
		      \item What command can be used to rename a file?
		      \item How can you get information about the pointed to file?
		      \item What will space with m do?
		      \item How can you protect a file?
		      \item What is the command to view other file types?
		      \item How can you move a file?
		      \item Can a "protected file" be written in?
		      \item When moving a file, does the file remain in its original location?
	      \end{enumerate}

%%%%%%%%%%%%%%%%%%%%%%%%%%%%%%%%%%%%%%%%%%
%%%%%%%%%%%%%%%%%%%%%%%%%%%%%%%%%%%%%%%%%%
%                                        %
%              Lesson 23                  %
%                                        %
%%%%%%%%%%%%%%%%%%%%%%%%%%%%%%%%%%%%%%%%%%
%%%%%%%%%%%%%%%%%%%%%%%%%%%%%%%%%%%%%%%%%%
\clearpage

\section{ Block Commands}
\

Skills Addressed in this Lesson:
\begin{itemize}
	\item Open the block commands menu.
	\item Copy and paste a block of text to and from the clipboard.
	\item Delete or move a block of text.
	\item Learn about other aspects of the block commands menu.
\end{itemize}
\subsection{Tasks}
\begin{enumerate}
	\item When writing and editing in a document or e-mail,, you may use the block commands menu.  From within a document you have already written, press space with b to open the block commands menu.
	\item In the block commands menu, space or backspace to explore the available options of:
	      \begin{enumerate}
		      \item Append block to clipboard
		      \item Copy block to clipboard
		      \item Delete a block of text
		      \item Insert a file
		      \item Move a block of text to the clipboard
		      \item Paste a block of text from the clipboard
		      \item Read a block of text
		      \item Store, block of text
		      \item Top marker insertion
		      \item Bottom marker insertion
		      \item Erase file and exit keyword
		      \item Zap, erase the block markers
		      \item Grade, correct grade of block
	      \end{enumerate}
	\item Exit the block commands menu now.  Use e with space to exit the block commands menu.
	\item Practice a common task using the block commands menu.  Put the cursor at the beginning of a sentence in a practice document.
	\item Press space with b to go to the block commands menu.
	\item Press t for top block marker.  You will then be taken back to your document.
	\item Use the cursor keys to place the cursor at the end of the sentence.  Press space with b to go back to the block commands menu.
	\item Press b for bottom block marker.
	\item Since you now have a top block mark set and a bottom block mark set, you will remain in the block commands menu.  Press c to copy the block of text to the clipboard.
	\item You will be placed back in your document.  Go to the end of the document.
	\item Paste the block of text at the end of the document.  Then you will have two copies of the sentence.  One in the original location and one at the end of the document.
	\item You may use the block commands menu to help edit documents or to move text from one document to another.
	\item When you append a block to the clipboard, you are adding the block to what is already in the clipboard.
	\item Cutting text to the clipboard means the text will be placed on the clipboard.  It will no longer remain where it was in the document.  "Cut" means that you removed the block of text.
	\item Explore other block commands areas as time allows.
\end{enumerate}

Note: The block commands menu can be used on the Internet, in books, and various areas of the BrailleNote.
\clearpage
\subsection{Quiz}
True or False
\begin{enumerate}
	\item To append text to the clipboard means to add the text to the end of the clipboard.
	\item Space with b will open the block commands menu.
	\item With the block commands menu, it is possible to delete a block of text.
	\item There is no difference between cutting text from the clipboard and copying text from the clipboard.
	\item Press space with dot3 or space with dot6 to move a day back and forward at a time in the calendar.
	\item From the Main Menu of the BrailleNote, access the internet by keying in the letter b.
	\item To open a link on a web page, press enter.
	\item Enter with backspace with c will open the calculator.
	\item Space with f, then f will find text forward from the cursor.
	\item Space with b, then b will set a bottom block marker.
\end{enumerate}

%%%%%%%%%%%%%%%%%%%%%%%%%%%%%%%%%%%%%%%%%%
%%%%%%%%%%%%%%%%%%%%%%%%%%%%%%%%%%%%%%%%%%
%                                        %
%              Lesson 24                  %
%                                        %
%%%%%%%%%%%%%%%%%%%%%%%%%%%%%%%%%%%%%%%%%%
%%%%%%%%%%%%%%%%%%%%%%%%%%%%%%%%%%%%%%%%%%
\clearpage


\section{ Nemeth Tutorial}
\

Skills Addressed in this Lesson:
\begin{itemize}
	\item Open the Nemeth tutorial.
	\item Open the desired area in the contents.
	\item Choose a lesson and an activity.
	\item In a given lesson, work through the four exercises of explanation, writing, reading, and proofreading.
\end{itemize}
\subsection{Tasks}
\begin{enumerate}
	\item Follow the instructions that come with the Nemeth tutorial program to install the software on the BrailleNote.
	\item Start at Main Menu.  To open the Nemeth tutorial, press x.
	\item ``List of keysoft extensions" will be the cue.  Space to the Nemeth tutorial and press enter.
	\item If someone was previously using the Nemeth tutorial, you may be asked if you want to continue a lesson.  Key in the letter n for no and press enter.
	\item Read through the contents of lessons.  Press space with h to get help on reading through the items of the contents.  Contents lessons are listed below.
	      \begin{enumerate}
		      \item Braille Numbers and Basic Indicators
		      \item Plus, Minus, and Equals Signs
		      \item Decimal Point and Related Symbols
		      \item Multiplication Signs
		      \item Division and Fraction Signs
		      \item Spatial Arrangements
		      \item Roman Numerals and Odds and Ends
		      \item More Signs and Operations
		      \item Use of Letters, Symbols, and Numbers
		      \item Signs of Grouping
		      \item More Signs of Comparison
		      \item Level Indicators
		      \item More Radicals and Groups
		      \item The Shape Indicator
		      \item Different Type Forms
		      \item Formats for Geometric Proofs
		      \item Fractions, Hyper and hyper complex
		      \item Integrals, Sigma Notation, and Limits
	      \end{enumerate}
	\item Press enter on the content area that you would like to complete.
	\item In each main area, there are smaller lessons.
	\item In each main area there are exercises.
	\item The first listed exercise will be Explanation.  To read the Explanation, enter on it.  Then use the normal review and thumb-key commands to read.
	\item After the Explanation there will be a Reading Exercise.  In this exercise, you will be given instructions to read each item and compare your response to the spoken prompt.  Available commands for this exercise are:
	      \begin{enumerate}
		      \item Use space with i to hear the correct answer.
		      \item Move to the next exercise item by pressing enter.
		      \item Space with e will return to the menu.
		      \item Space with e when you finish the Reading Exercise.  Use space with h to get help when you need it.
		      \item After the Reading Exercise you will find the Writing Exercise.  You will be instructed to "Braille the following exactly as spoken." Use the commands listed below.
		      \item To hear an item again, use space with c.
		      \item Space with dot2 or space with dot5 will review by word.
		      \item Press enter to have your answer graded.
		      \item Space with r will stop your answer so you can try again.
		      \item To review the correct answer, press space with i.
		      \item The proof reading exercises are next.  You will need to edit the Braille so that it is spoken correctly.
		      \item Space with c will let you hear the item again.
		      \item Move back and forward by word in the usual manner.  Space with 2 5 will read the current word.
		      \item Press enter to have your answer graded.
		      \item Space with r will stop your current answer and allow you to start again.
		      \item Use space with h to get help and learn the commands at any time.
		      \item You can review without answering by stopping your answer and pressing space.
	      \end{enumerate}
	\item Explore using the Nemeth tutorial as time allows.
\end{enumerate}
\clearpage
\subsection{Quiz}
True or False

\begin{enumerate}
	\item From Main Menu, press x to start the process of opening the Nemeth tutorial.
	\item There are no Nemeth tutorial lessons related to decimal points and related symbols.
	\item In the Nemeth tutorial you can learn about the shape indicator.
	\item In the Nemeth Tutorial reading exercises, you will be given instructions to read each item and compare your response to the spoken prompt.
	\item Space with 1 will move back by word.
	\item Use space with h within an exercise to hear the commands.
	\item From Main Menu, k will enter the key learn mode.
	\item Space with e will exit key learn mode.
	\item When you append an item to the clipboard, you "add" it to the clipboard.
	\item In the spelling checker, you cannot just check one word.  You must check the entire document.
\end{enumerate}

%%%%%%%%%%%%%%%%%%%%%%%%%%%%%%%%%%%%%%%%%%
%%%%%%%%%%%%%%%%%%%%%%%%%%%%%%%%%%%%%%%%%%
%                                        %
%              Lesson 25                  %
%                                        %
%%%%%%%%%%%%%%%%%%%%%%%%%%%%%%%%%%%%%%%%%%
%%%%%%%%%%%%%%%%%%%%%%%%%%%%%%%%%%%%%%%%%%
\clearpage

\section{ Change Braille Grade}
\


Skills Addressed in this Lesson:
\begin{itemize}
	\item Force the Braille grade to grade 1, 2, or computer Braille.
\end{itemize}

 \subsection{Tasks}
\begin{enumerate}
	\item If you are writing a Braille document, you can force the Braille note to do computer Braille, grade 1 Braille, or grade 2 Braille.
	\item Create a new Braille document as usual.
	\item Write a Braille sentence telling what you like to search for on the Google web site.
	\item Then force computer Braille in order to correctly write the web site.  Switch to computer Braille by pressing backspace with j.
	\item Then write the web address using computer Braille.  For a period (a dot) use dots4 6.  Do not use Braille contractions.
	\item Then switch back to grade 2 Braille.  Backspace with b will go back to grade 2 Braille.  Try it out.
	\item You can also switch to grade 1 Braille when you want to.  Force grade 1 Braille with backspace and a together.
	\item Try switching between grade of Braille for practice.
\end{enumerate}
\clearpage

\subsection{Quiz}
Say or write the letter of the best answer
\begin{enumerate}
	\item What will enter with i do?
	      \begin{enumerate}
		      \item This gives information about the word count.
		      \item Enter with i opens the insert menu.
		      \item This opens the information options.
		      \item Enter i opens the User Guide index.
	      \end{enumerate}
	\item How can you force grade 2 Braille?
	      \begin{enumerate}
		      \item enter with b
		      \item space with b
		      \item backspace with b
		      \item backspace with low b
	      \end{enumerate}
	\item How can you start the Braille display moving automatically?
	      \begin{enumerate}
		      \item backspace with 1 2 4 5 6
		      \item enter with 1 2 4 5 6
		      \item space with 4 5 6
		      \item space with 1 2 4 5 6
	      \end{enumerate}
	\item What does backspace with j do?
	      \begin{enumerate}
		      \item forces grade 1 Braille
		      \item forces computer Braille
		      \item forces grade 1 Braille
		      \item forces a Braille table
	      \end{enumerate}
	\item How can you check the date?
	      \begin{enumerate}
		      \item enter with d
		      \item space with d
		      \item backspace with d
		      \item enter with low d
	      \end{enumerate}
	\item What will backspace with f do?
	      \begin{enumerate}
		      \item find text
		      \item find and replace text
		      \item returns to file manager
		      \item moves forward
	      \end{enumerate}
	\item When writing computer Braille, how do you write a period?
	      \begin{enumerate}
		      \item dots1 3
		      \item dots2 5 6
		      \item dots4 6
		      \item none of the above
	      \end{enumerate}
	\item When you are in a document, what will space with dots1 2 5 6 do?
	      \begin{enumerate}
		      \item switch between documents
		      \item find and replace text
		      \item opens the spelling checker
		      \item none of the above
	      \end{enumerate}
	\item What does space with a full cell do?
	      \begin{enumerate}
		      \item advances the Braille display one width
		      \item opens key learn mode
		      \item displays the power status
		      \item returns to main menu
	      \end{enumerate}
	\item When writing computer Braille, what is space with u, then dot4?
	      \begin{enumerate}
		      \item period
		      \item at @ symbol
		      \item comma
		      \item none of the above
	      \end{enumerate}
\end{enumerate}

%%%%%%%%%%%%%%%%%%%%%%%%%%%%%%%%%%%%%%%%%%
%%%%%%%%%%%%%%%%%%%%%%%%%%%%%%%%%%%%%%%%%%
%                                        %
%              Lesson 26                  %
%                                        %
%%%%%%%%%%%%%%%%%%%%%%%%%%%%%%%%%%%%%%%%%%
%%%%%%%%%%%%%%%%%%%%%%%%%%%%%%%%%%%%%%%%%%
\clearpage

\section{ Stopwatch}
\
Skills Addressed in this Lesson:
\begin{itemize}
	\item Open the stopwatch
	\item Start and stop the stopwatch
	\item 0 the stopwatch
	\item Explore using other commands in the stopwatch.
	\item Exit the stopwatch.
\end{itemize}

\subsection{Tasks}
\begin{enumerate}
	\item Enter the stopwatch with enter and w together.  This can be done from anywhere on the BrailleNote.
	\item Press space to start or stop the stopwatch.  Press space to start the stopwatch.  Complete a task such as saying the alphabet backwards.  Press space to stop the stopwatch.  How long did saying the alphabet backwards take you?
	\item Now use backspace with dots1 4 to zero the stopwatch.  This will clear the time.
	\item If you want to insert a stopwatch time into a document, you can use backspace with k.  The time on the stopwatch will be copied to the clipboard.  From the block commands menu, you can paste the time into your document.
	\item Use space with r to hear the elapsed time.  Explore this command when you are running the stopwatch.
	\item In the stopwatch, use space with h to hear the commands for yourself.
	\item The letter l will repeat the last elapsed time that was read.
	\item Time yourself when completing a variety of tasks to practice using the stopwatch.
	\item When finish, zero the stopwatch.
	\item Exit the stopwatch with space and e.
\end{enumerate}

\clearpage

\subsection{Quiz}
Write a short answer to each question


\begin{enumerate}
	\item How can you open the stopwatch?
	\item Read a page of Braille and time yourself.  How long did it take you to read the page of Braille?
	\item What is the command to 0 the stopwatch.
	\item How can you copy a stopwatch time to the clipboard?
	\item From within a document, how can you paste a stopwatch time from the clipboard?
	\item From within the stopwatch, what does space with h do?
	\item How can you exit the stopwatch?
	\item From main menu, how can you go into key learn mode?
	\item Give a situation in which you could use a stopwatch?
	\item From within a document, how can you get to the delete menu?
\end{enumerate}

%%%%%%%%%%%%%%%%%%%%%%%%%%%%%%%%%%%%%%%%%%
%%%%%%%%%%%%%%%%%%%%%%%%%%%%%%%%%%%%%%%%%%
%                                        %
%              Lesson 27                  %
%                                        %
%%%%%%%%%%%%%%%%%%%%%%%%%%%%%%%%%%%%%%%%%%
%%%%%%%%%%%%%%%%%%%%%%%%%%%%%%%%%%%%%%%%%%
\clearpage

\section{ Switch Among Documents}
\

Skills Addressed in this Lesson:
\begin{itemize}
	\item Switch between keyword documents.
\end{itemize}



 \subsection{Tasks}
\begin{enumerate}
	\item Open a document that you have recently worked on.
	\item Press space with dots1 2 5 6.
	\item You will read the question "document to open?" You may read the document that you had opened just before the document that is currently open.
	\item To switch to the previously opened document, just press enter.
	\item If you want to see a list of files in the folder, press space.
	\item If you want to open a document from a different folder, you can.  After pressing space with dots1 2 5 6, press backspace.
	\item Space to the folder you want, then press enter.
	\item Locate the file you want and press enter.
	\item You may use space with 1 2 5 6 to switch between documents or to create a new document.  At the "document to open" question, you may write the name of a document you want to create.  Press enter, then answer y for yes when asked if you want to create a new document.
	\item Practice using space with 1 2 5 6 to switch between documents.
\end{enumerate}

\clearpage

\subsection{Quiz}
True or False

\begin{enumerate}
	\item Space with dots4 5 6 will jump to the top of a document.
	\item Space with dots2 3 4 5 will switch between documents.
	\item When switching between documents, you can press backspace to start the process of opening a different folder.
	\item Using space with dots1 2 5 6 is a quick way to switch between documents.
	\item To locate a file you want from the list, you may press space.
	\item The previous thumb-key and space will toggle among speech options.
	\item Enter with dot3 will turn the voice volume up.
	\item Space with 1 2 4 5 6 will start the display moving automatically.
	\item Enter with o will take you to the options menu.
	\item It is not possible to play games on the BrailleNote.
\end{enumerate}

%%%%%%%%%%%%%%%%%%%%%%%%%%%%%%%%%%%%%%%%%%
%%%%%%%%%%%%%%%%%%%%%%%%%%%%%%%%%%%%%%%%%%
%                                        %
%              Lesson 28                  %
%                                        %
%%%%%%%%%%%%%%%%%%%%%%%%%%%%%%%%%%%%%%%%%%
%%%%%%%%%%%%%%%%%%%%%%%%%%%%%%%%%%%%%%%%%%
\clearpage

\section{ Change Fonts}
\

Skills Addressed in this Lesson:
\begin{itemize}
	\item Open the "font" menu.
	\item Turn bold on or off when necessary.
	\item Turn italics on or off.
	\item Underline text when desired.
	\item Explore fonts to determine what is best for your situation.
\end{itemize}



\subsection{Tasks}
\begin{enumerate}
	\item Create a new document.
	\item Change the font size.
	\item Press enter with f.  You will read the question "font?"
	\item At this point you can enter a letter of the alphabet.  For example, you may enter the letter h if you want to turn "20 characters per inch" on or off.
	\item To turn an option on, press n.  Press f to turn an option off.  Press enter to leave the item unchanged. Fonts may change according to which printer you are using.  I have selected the hp printer in my printer setup list.  Below are the items I can obtain after pressing enter with f to open the font menu.
	      \begin{enumerate}
		      \item 12 point character height
		      \item bold
		      \item courier
		      \item draft quality
		      \item 5 characters per inch
		      \item 10 characters per inch
		      \item 16.67 characters per inch
		      \item 20 characters per inch
		      \item italic
		      \item 8 lines per inch
		      \item 12 characters per inch
		      \item letter quality
		      \item 6 lines per inch
		      \item san sserif
		      \item overstrike
		      \item gothic
		      \item cg times
		      \item 6 point character height
		      \item superscript
		      \item subscript
		      \item underline
		      \item font v
		      \item font w
		      \item proportional spacing
		      \item 24 point character height
		      \item landscape
	      \end{enumerate}
	\item If you turn bold on by using enter with f, b, then n for on, you can write your text that you want to be in bold font.  If you never turn this off, the entire document will be in bold text.  Turn the font off when you no longer want text to be in bold.  To turn bold off, press enter with f, b, then the letter f for off.
	\item If you are unable to read printed text and you need to use a variety of font styles, ask someone to assist you with exploring different font types.
	      \begin{enumerate}
		      \item Explore to find what is best for you.
	      \end{enumerate}
\end{enumerate}
\clearpage
\subsection{Quiz}



Give a short answer to each question. Items may be taken from this lesson or prior lessons.



\begin{enumerate}
	\item What does enter with f do from within a BrailleNote document?
	\item In the font menu, what does n do?
	\item How can you force grade 1 Braille in a document?
	\item In the stopwatch, what does space with r do?
	\item How can you check the battery status?
	\item When word processing, what will space with dots1 2 5 6 do?
	\item How can you turn Bluetooth on?
	\item When changing embosser settings, how can you toggle to the embosser you want?
	\item What does o with space, l, d do?
	\item From Main Menu, what does g do?
\end{enumerate}

%%%%%%%%%%%%%%%%%%%%%%%%%%%%%%%%%%%%%%%%%%
%%%%%%%%%%%%%%%%%%%%%%%%%%%%%%%%%%%%%%%%%%
%                                        %
%              Lesson 29                  %
%                                        %
%%%%%%%%%%%%%%%%%%%%%%%%%%%%%%%%%%%%%%%%%%
%%%%%%%%%%%%%%%%%%%%%%%%%%%%%%%%%%%%%%%%%%
\clearpage

\section{ FM Radio Tuner}
\
It is best to use earbuds when using the fm radio tuner. The earbuds will act as an antenna to improve reception.

Skills Addressed in this Lesson:
\begin{enumerate}
	\item Access the fm radio tuner.
	\item Scroll through radio station presets.
	\item Scan forward or back for new preset stations.
	\item Create a new preset when necessary.
	\item Adjust the radio volume.
	\item Quickly switch between an application and the fm tuner.
\end{enumerate}

\subsection{Tasks}
	      \begin{enumerate}
		      \item There is more than one way to open the fm tuner. One way to open the tuner is to start from main menu. Press ``m$"$  to open the media center. Then space to the fm tuner and press enter. Do this now.
		      \item Press enter with dots1 3 to turn the volume down for the radio. To turn the volume up, press enter with dots4 6. You can use these commands any time the radio or other media files are playing. Volume of media can be adjusted even when you are in other BrailleNote applications.
		      \item In the radio tuner, press space or backspace to scroll through the preset stations. Do this now.
		      \item You can scan for new stations. Scan forward with space and dot4. Space with dot1 will scan back for new stations.
		      \item Use space with dot2 or space with dot5 to scan back or forward by the frequency of 1 megahertz.
		      \item You can increase or decrease the frequency by 11 kilohertz to more finely tune your station. Try this by using space with dot3 or space with dot6.
		      \item If you find a station to add to the presets, press backspace with r.
		      \item While in the tuner, press space with h to locate and read a list of fm tuner commands.
		      \item Explore using the radio tuner as time allows.
		      \item When you finish exploring, press e with space to exit the fm tuner. Press y to leave the radio on. Press n to turn the radio off.
		      \item Open any keyword document. You can work and listen to the radio. Press enter with backspace with f to quickly switch to the fm tuner.
		      \item To switch back to your document and keep the radio on, press enter with backspace with w. This will work with other BrailleNote applications as well.
	      \end{enumerate}
	      \clearpage
\subsection{Quiz}
Write or say the letter of the best answer
	      \begin{enumerate}
		      \item In the fm tuner, what will enter with dots1 3 do?
		  \begin{enumerate}
		      \item turn the volume up
		      \item turn the volume down
		      \item find a new preset station
		      \item turn the radio off
	      \end{enumerate}
	\item What is a quick method for opening the fm tuner?
	      \begin{enumerate}
		      \item backspace with enter with r
		      \item enter with f
		      \item space with f
		      \item enter with backspace with f
	      \end{enumerate}
	\item How can you add a station to the presets?
	      \begin{enumerate}
		      \item space with r
		      \item backspace with r
		      \item enter with r
		      \item backspace with f
	      \end{enumerate}
	\item In the radio, what will space or backspace do?
	      \begin{enumerate}
		      \item edit a preset
		      \item increase the rate
		      \item move by frequency
		      \item scan for presets
	      \end{enumerate}
	\item From main menu, what letter can you press to open the media center?
	      \begin{enumerate}
		      \item f
		      \item r
		      \item m
		      \item t
	      \end{enumerate}
	\item In the radio, what will space and dot4 do?
	      \begin{enumerate}
		      \item scan forward for new stations
		      \item scan back for new stations
		      \item turn the volume up
		      \item turn the volume down
	      \end{enumerate}
	\item What application will open if you press i from main menu?
	      \begin{enumerate}
		      \item address book
		      \item web browser
		      \item all applications
		      \item information menu
	      \end{enumerate}
	\item On the BrailleNote, what does o with space, then p do?
	      \begin{enumerate}
		      \item checks the time
		      \item checks the date
		      \item checks the power status
		      \item opens the fm tuner
	      \end{enumerate}
	\item From main menu, what will the letter b open?
	      \begin{enumerate}
		      \item web browser
		      \item book reader
		      \item spelling checker
		      \item both the radio and the media player
	      \end{enumerate}
	\item You can leave the radio on while working in other applications.
	      \begin{enumerate}
		      \item true
		      \item false
	      \end{enumerate}
\end{enumerate}

%%%%%%%%%%%%%%%%%%%%%%%%%%%%%%%%%%%%%%%%%%
%%%%%%%%%%%%%%%%%%%%%%%%%%%%%%%%%%%%%%%%%%
%                                        %
%              Lesson 30                  %
%                                        %
%%%%%%%%%%%%%%%%%%%%%%%%%%%%%%%%%%%%%%%%%%
%%%%%%%%%%%%%%%%%%%%%%%%%%%%%%%%%%%%%%%%%%
\clearpage

\section{ Insert Menu}

Skills Addressed in this Lesson:
\begin{itemize}
	\item Open the insert menu.
	\item Insert the date.
	\item Insert the time.
	\item Insert a calculation into a document.
\end{itemize}

\subsection{Tasks}
\begin{enumerate}
	\item Open a document in keyword.
	\item Insert today's date into the document.
	\item Place the cursor at the point you would like the date to be inserted. Then press enter with i to get started.
	\item When in the insert menu, space to date and press enter. For the shortcut, just press the letter.
	\item You may insert today's date or the printing date. Press t for today's date or p to have the date on which you print the document inserted.
	\item The date will be inserted into the document you are working on.
	\item Use the same method to insert the time into a document. Press enter with i to open the insert menu. Press t for time. The time will be inserted into your document.
	\item When you press enter with i, then the letter c, you will be able to insert a calculation from the scientific calculator into a document. If there is currently no result in the calculator, nothing will be inserted. Press c to insert a whole calculation or r to insert the result only.
	\item Though this has been explained in other lessons, review and explore using the insert menu at this time.
\end{enumerate}
\subsection{Quiz}
True or False
\begin{enumerate}
	\item Space with i will pen the insert menu.
	\item It is possible to insert a result or a who calculation into a document.
	\item There is no way to insert the printing date into a document.
	\item Enter with dot1 will make the voice volume louder.
	\item Space with a full cell will return to the main menu.
	\item Enter with backspace with c will open the calculator.
	\item The previous thumb-key with space will toggle among the speech options.
	\item Space with f will open the font menu.
	\item Enter with backspace with e will open the keymail application.
	\item Space with 2 6 will open the e-mail action menu.
\end{enumerate}

%%%%%%%%%%%%%%%%%%%%%%%%%%%%%%%%%%%%%%%%%%
%%%%%%%%%%%%%%%%%%%%%%%%%%%%%%%%%%%%%%%%%%
%                                        %
%              APPENDICES                %
%                                        %
%%%%%%%%%%%%%%%%%%%%%%%%%%%%%%%%%%%%%%%%%%
%%%%%%%%%%%%%%%%%%%%%%%%%%%%%%%%%%%%%%%%%%
\clearpage

\chapter{Checklists}
%%%%%%%%%%%%%%%%%%%%%%%%%%%%%%%%%%%%%%%%%%%%%%%%%%%%%%%%%%%%%%%%%%%%%%%%%%%%%%%%%%%%
%                              TABLES AND CHECKLISTS                              %
%%%%%%%%%%%%%%%%%%%%%%%%%%%%%%%%%%%%%%%%%%%%%%%%%%%%%%%%%%%%%%%%%%%%%%%%%%%%%%%%%%%%
\section{BrailleNote Apex Overall Skills Checklist: Teacher Version}
\clearpage
{
	\renewcommand{\arraystretch}{1.5}
	\begin{table}[!htbp]
		\begin{tabular}{|m{3cm}m{3.0cm}m{3.00cm}m{3.0cm}m{2.75cm}|}
			\multicolumn{3}{b{9cm}}{Student: \hrulefill} & \multicolumn{2}{b{5.75cm}}{Grade: \hrulefill} \\[.5em]
			\multicolumn{3}{b{9cm}}{School: \hrulefill} & \multicolumn{2}{b{5.75cm}}{TSVI: \hrulefill} \\[.5em] \hline
			\multicolumn{5}{|>{\centering\arraybackslash}m{14.75cm}|}{Prompt Hierarchy / Scoring} \\\hline
			\footnotesize{I = Independent} & \footnotesize{FP = Full Physical} & \footnotesize{PP = Partial Physical} & \footnotesize{VP = Verbal Prompt} & \footnotesize{VC = Verbal Cue} \\\hline
		\end{tabular}
	\end{table}
}
\vspace*{-.70cm}
{
	\renewcommand{\arraystretch}{1.5}
	\begin{longtable}[!htbp]{|m{12.0cm}||m{1.0cm}|m{1.0cm}||m{1.0cm}|}
		\hline
		BRAILLENOTE APEX SKILL                                                                                                                                                                           & Pre  & Post & \checkmark \textgreater80\% \\[.5em]\hline
		\endfirsthead
		\multicolumn{4}{r}%
		{\textit{Continued from previous page}} \\\hline
		\hline
		BRAILLENOTE APEX SKILL                                                                                                                                                                           & Pcre & Post & \checkmark \textgreater80\% \\[.5em]\hline
		\hline
		\endhead
		\hline \multicolumn{4}{r}{\textit{Continued on next page}} \\\hline
		\endfoot
		\hline
		\endlastfoot
		\multicolumn{4}{|>{\centering\arraybackslash}m{15cm}|}{\textcolor{heading}{\MakeUppercase{\textbf{ADVANCED SKILL LESSON 1}}}} \\\hline
		1. Identify six Braille writer keys.                                                                                                                                                             &      &      &                             \\\hline
		2. Utilize the \textcolor{accent}{\MakeUppercase{\textbf{space}}} and \textcolor{accent}{\MakeUppercase{\textbf{backspace}}}.                                                                    &      &      &                             \\\hline
		3. Use \textcolor{accent}{\MakeUppercase{\textbf{enter}}} when necessary.                                                                                                                        &      &      &                             \\\hline
		4. Identify and make use of \textcolor{accent}{\MakeUppercase{\textbf{thumb-keys}}}.                                                                                                             &      &      &                             \\\hline
		5. Turn BrailleNote on/off independently.                                                                                                                                                        &      &      &                             \\\hline
		6. Plug in and use earbuds when appropriate.                                                                                                                                                     &      &      &                             \\\hline
		7. Unplug and store earbuds when finished.                                                                                                                                                       &      &      &                             \\\hline
		8. Exhibit care of the BrailleNote.                                                                                                                                                              &      &      &                             \\\hline
		\multicolumn{4}{|>{\centering\arraybackslash}m{15cm}|}{\textcolor{heading}{\MakeUppercase{\textbf{ADVANCED SKILL LESSON 2}}}} \\\hline
		9. Use initial letters to move to and open menu items.                                                                                                                                           &      &      &                             \\\hline
		10. Use \textcolor{accent}{\MakeUppercase{\textbf{space + e}}} to go back by level.                                                                                                              &      &      &                             \\\hline
		11. Turn speech on and off (\textcolor{accent}{\MakeUppercase{\textbf{previous thumb-key + space}}}).                                                                                            &      &      &                             \\\hline
		12. Use on-screen help to solve problems on own (\textcolor{accent}{\MakeUppercase{\textbf{space + h}}}).                                                                                        &      &      &                             \\\hline
		13. Return to Main Menu from anywhere (\textcolor{accent}{\MakeUppercase{\textbf{space + dots123456}}}).                                                                                         &      &      &                             \\\hline
		14. Navigate menus using \textcolor{accent}{\MakeUppercase{\textbf{space}}}, \textcolor{accent}{\MakeUppercase{\textbf{backspace}}}, or \textcolor{accent}{\MakeUppercase{\textbf{thumb-keys}}}. &      &      &                             \\\hline
		\multicolumn{4}{|>{\centering\arraybackslash}m{15cm}|}{\textcolor{heading}{\MakeUppercase{\textbf{ADVANCED SKILL LESSON 3}}}} \\\hline
		15. Create a new document in keyword.                                                                                                                                                            &      &      &                             \\\hline
		16. Open an existing Word processing document.                                                                                                                                                   &      &      &                             \\\hline
		17. Use the backspace key to delete the previous character.                                                                                                                                      &      &      &                             \\\hline
		18. Use the \textcolor{accent}{\MakeUppercase{\textbf{cursor keys}}} for editing.                                                                                                                &      &      &                             \\\hline
		20. Review text using the \textcolor{accent}{\MakeUppercase{\textbf{thumb-keys}}}.                                                                                                               &      &      &                             \\\hline
		21. Use review commands as necessary.                                                                                                                                                            &      &      &                             \\\hline
		\multicolumn{4}{|>{\centering\arraybackslash}m{15cm}|}{\textcolor{heading}{\MakeUppercase{\textbf{ADVANCED SKILL LESSON 4}}}} \\\hline
		22. Create a new folder.                                                                                                                                                                         &      &      &                             \\\hline
		23. Center a line of text (\textcolor{accent}{\MakeUppercase{\textbf{enter + c}}}).                                                                                                              &      &      &                             \\\hline
		24. Use the delete menu in a document (\textcolor{accent}{\MakeUppercase{\textbf{space + d}}}).                                                                                                  &      &      &                             \\\hline
		25. Obtain the time (\textcolor{accent}{\MakeUppercase{\textbf{enter + t}}}).                                                                                                                    &      &      &                             \\\hline
		26. Obtain the date (\textcolor{accent}{\MakeUppercase{\textbf{enter + d}}}).                                                                                                                    &      &      &                             \\\hline
		\multicolumn{4}{|>{\centering\arraybackslash}m{15cm}|}{\textcolor{heading}{\MakeUppercase{\textbf{ADVANCED SKILL LESSON 5}}}} \\\hline
		27. Obtain and read e-mail messages.                                                                                                                                                             &      &      &                             \\\hline
		28. Delete messages when required.                                                                                                                                                               &      &      &                             \\\hline
		29. Empty trash as appropriate.                                                                                                                                                                  &      &      &                             \\\hline
		30. Create and send an e-mail message.                                                                                                                                                           &      &      &                             \\\hline
		31. Use e-mail for school and leisure tasks.                                                                                                                                                     &      &      &                             \\\hline
		32. Safely set up and remove cards and cables for internet use.                                                                                                                                  &      &      &                             \\\hline
		33. Move to the top of documents, etc (\textcolor{accent}{\MakeUppercase{\textbf{space + 123}}}).                                                                                                &      &      &                             \\\hline
		35. Check text for spelling errors (\textcolor{accent}{\MakeUppercase{\textbf{space + 16}}}).                                                                                                    &      &      &                             \\\hline
		36. Look up a word to determine how it is spelled.                                                                                                                                               &      &      &                             \\\hline
		37. Use other spelling checker options as appropriate.                                                                                                                                           &      &      &                             \\\hline
		\multicolumn{4}{|>{\centering\arraybackslash}m{15cm}|}{\textcolor{heading}{\MakeUppercase{\textbf{ADVANCED SKILL LESSON 7}}} }\\\hline
		38. Open the BrailleNote user guide (\textcolor{accent}{\MakeUppercase{\textbf{space + o}}}, \textcolor{accent}{\MakeUppercase{\textbf{u}}}).                                                    &      &      &                             \\\hline
		39. Use the index within the user guide.                                                                                                                                                         &      &      &                             \\\hline
		40. Explore the table of contents in the user guide.                                                                                                                                             &      &      &                             \\\hline
		41. Solve problems using the user guide.                                                                                                                                                         &      &      &                             \\\hline
		\multicolumn{4}{|>{\centering\arraybackslash}m{15cm}|}{\textcolor{heading}{\MakeUppercase{\textbf{ADVANCED SKILL LESSON 8}}}} \\\hline
		42. Open the Address Book.                                                                                                                                                                       &      &      &                             \\\hline
		43. Add an address when necessary.                                                                                                                                                               &      &      &                             \\\hline
		44. Look up an address.                                                                                                                                                                          &      &      &                             \\\hline
		45. Edit fields within an address record.                                                                                                                                                        &      &      &                             \\\hline
		46. Delete an address record.                                                                                                                                                                    &      &      &                             \\\hline
		\multicolumn{4}{|>{\centering\arraybackslash}m{15cm}|}{\textcolor{heading}{\MakeUppercase{\textbf{ADVANCED SKILL LESSON 9}}}} \\\hline
		47. Send an e-mail attachment.                                                                                                                                                                   &      &      &                             \\\hline
		48. Use the address selection list (\textcolor{accent}{\MakeUppercase{\textbf{backspace + L}}} in to field).                                                                                     &      &      &                             \\\hline
		49. Access and understand the e-mail action menu (\textcolor{accent}{\MakeUppercase{\textbf{space + 26}}}).                                                                                      &      &      &                             \\\hline
		\multicolumn{4}{|>{\centering\arraybackslash}m{15cm}|}{\textcolor{heading}{\MakeUppercase{\textbf{ADVANCED SKILL LESSON 10}}} }\\\hline
		50. Open the BrailleNote planner.                                                                                                                                                                &      &      &                             \\\hline
		51. Navigate around the Planner.                                                                                                                                                                 &      &      &                             \\\hline
		32. Independently schedule an appointment.                                                                                                                                                       &      &      &                             \\\hline
		53. Re-schedule or delete an appointment.                                                                                                                                                        &      &      &                             \\\hline
		54. Check for next appointment (\textcolor{accent}{\MakeUppercase{\textbf{enter + n}}}).                                                                                                         &      &      &                             \\\hline
		\multicolumn{4}{|>{\centering\arraybackslash}m{15cm}|}{\textcolor{heading}{\MakeUppercase{\textbf{ADVANCED SKILL LESSON 11}}} }\\\hline
		55. Open a web page.                                                                                                                                                                             &      &      &                             \\\hline
		56. Move back and forward by link (\textcolor{accent}{\MakeUppercase{\textbf{space}}}, \textcolor{accent}{\MakeUppercase{\textbf{backspace}}}, initial letter).                                  &      &      &                             \\\hline
		57. Open a link (\textcolor{accent}{\MakeUppercase{\textbf{enter}}} or click with \textcolor{accent}{\MakeUppercase{\textbf{cursor key}}}).                                                      &      &      &                             \\\hline
		58. Use input controls as necessary.                                                                                                                                                             &      &      &                             \\\hline
		59. Add to and use the favorites list.                                                                                                                                                           &      &      &                             \\\hline
		60. Use computer Braille when required.                                                                                                                                                          &      &      &                             \\\hline
		\multicolumn{4}{|>{\centering\arraybackslash}m{15cm}|}{\textcolor{heading}{\MakeUppercase{\textbf{ADVANCED SKILL LESSONS 12 \& 21}}}} \\\hline
		61. Locate a book from desired site.                                                                                                                                                             &      &      &                             \\\hline
		62. Save and/or unzip book as required.                                                                                                                                                          &      &      &                             \\\hline
		\multicolumn{4}{|>{\centering\arraybackslash}m{15cm}|}{\textcolor{heading}{\MakeUppercase{\textbf{ADVANCED SKILL LESSONS 13 \& 27}}}} \\\hline
		63. Open and exit the calculator.                                                                                                                                                                &      &      &                             \\\hline
		64. Complete math problems as necessary.                                                                                                                                                         &      &      &                             \\\hline
		65. Insert a calculation into a document (\textcolor{accent}{\MakeUppercase{\textbf{enter + i}}}).                                                                                               &      &      &                             \\\hline
		66. Switch between open programs (\textcolor{accent}{\MakeUppercase{\textbf{enter + backspace + letter}}}).                                                                                      &      &      &                             \\\hline
		\multicolumn{4}{|>{\centering\arraybackslash}m{15cm}|}{\textcolor{heading}{\MakeUppercase{\textbf{ADVANCED SKILL LESSON 14}}}} \\\hline
		67. Open the Book Reader.                                                                                                                                                                        &      &      &                             \\\hline
		69. Read and/or navigate through a book.                                                                                                                                                         &      &      &                             \\\hline
		69. Exit a book when finished.                                                                                                                                                                   &      &      &                             \\\hline
		\multicolumn{4}{|>{\centering\arraybackslash}m{15cm}|}{\textcolor{heading}{\MakeUppercase{\textbf{ADVANCED SKILL LESSON 15}}}} \\\hline
		70. Open the options menu.                                                                                                                                                                       &      &      &                             \\\hline
		71. Manipulate Braille options.                                                                                                                                                                  &      &      &                             \\\hline
		72. Exit options menu and return to BrailleNote task.                                                                                                                                            &      &      &                             \\\hline
		\multicolumn{4}{|>{\centering\arraybackslash}m{15cm}|}{\textcolor{heading}{\MakeUppercase{\textbf{ADVANCED SKILL LESSON 16}}}} \\\hline
		73. Choose and open a game.                                                                                                                                                                      &      &      &                             \\\hline
		74. Use help to learn to play a game.                                                                                                                                                            &      &      &                             \\\hline
		75. Exit the games area.                                                                                                                                                                         &      &      &                             \\\hline
		\multicolumn{4}{|>{\centering\arraybackslash}m{15cm}|}{\textcolor{heading}{\MakeUppercase{\textbf{ADVANCED SKILL LESSON 17}}}} \\\hline
		76. Open the dictionary (\textcolor{accent}{\MakeUppercase{\textbf{space + o}}}, L).                                                                                                             &      &      &                             \\\hline
		77. Look up a word.                                                                                                                                                                              &      &      &                             \\\hline
		78. Read through the list of suggested words.                                                                                                                                                    &      &      &                             \\\hline
		79. Read through a list of word definitions.                                                                                                                                                     &      &      &                             \\\hline
		80. Insert look-up word into a document.                                                                                                                                                         &      &      &                             \\\hline
		\multicolumn{4}{|>{\centering\arraybackslash}m{15cm}|}{\textcolor{heading}{\MakeUppercase{\textbf{ADVANCED SKILL LESSONS 18 \& 19}}}} \\\hline
		81. Print a document.                                                                                                                                                                            &      &      &                             \\\hline
		82. Emboss a document.                                                                                                                                                                           &      &      &                             \\\hline
		83. Obtain document from embosser or printer.                                                                                                                                                    &      &      &                             \\\hline
		\multicolumn{4}{|>{\centering\arraybackslash}m{15cm}|}{\textcolor{heading}{\MakeUppercase{\textbf{ADVANCED SKILL LESSON 20}}}} \\\hline
		84. Pair BrailleNote with Bluetooth for embossing or printing.                                                                                                                                   &      &      &                             \\\hline
		85. Adjust port settings for Bluetooth.                                                                                                                                                          &      &      &                             \\\hline
		86. Emboss and/or print through Bluetooth device.                                                                                                                                                &      &      &                             \\\hline
		\multicolumn{4}{|>{\centering\arraybackslash}m{15cm}|}{\textcolor{heading}{\MakeUppercase{\textbf{ADVANCED SKILL LESSON 22}}}} \\\hline
		87. Obtain information about a file when in a file list (\textcolor{accent}{\MakeUppercase{\textbf{space + i}}})                                                                                 &      &      &                             \\\hline
		88. View other or all file types (\textcolor{accent}{\MakeUppercase{\textbf{space + x}}}).                                                                                                       &      &      &                             \\\hline
		89. Copy a file to another location (\textcolor{accent}{\MakeUppercase{\textbf{backspace + y}}}).                                                                                                &      &      &                             \\\hline
		\multicolumn{4}{|>{\centering\arraybackslash}m{15cm}|}{\textcolor{heading}{\MakeUppercase{\textbf{ADVANCED SKILL LESSON 23}}}} \\\hline
		90. Open the block command menu (\textcolor{accent}{\MakeUppercase{\textbf{space + b}}}).                                                                                                        &      &      &                             \\\hline
		91. Set top and bottom block markers.                                                                                                                                                            &      &      &                             \\\hline
		93. Delete text blocks when necessary.                                                                                                                                                           &      &      &                             \\\hline
		\multicolumn{4}{|>{\centering\arraybackslash}m{15cm}|}{\textcolor{heading}{\MakeUppercase{\textbf{ADVANCED SKILL LESSON 24}}}} \\\hline
		94. Open the Nemeth Tutorial (x from Main Menu).                                                                                                                                                 &      &      &                             \\\hline
		95. Begin or continue a specific  lesson.                                                                                                                                                        &      &      &                             \\\hline
		96. Complete all exercises within a  lesson as instructed.                                                                                                                                       &      &      &                             \\\hline
		\multicolumn{4}{|>{\centering\arraybackslash}m{15cm}|}{\textcolor{heading}{\MakeUppercase{\textbf{ADVANCED SKILL LESSON 25}}}} \\\hline
		97. Force computer Braille when in a document (\textcolor{accent}{\MakeUppercase{\textbf{backspace + j}}}).                                                                                      &      &      &                             \\\hline
		98. For grade 2 Braille (\textcolor{accent}{\MakeUppercase{\textbf{backspace + b}}}).                                                                                                            &      &      &                             \\\hline
		\multicolumn{4}{|>{\centering\arraybackslash}m{15cm}|}{\textcolor{heading}{\MakeUppercase{\textbf{ADVANCED SKILL LESSON 26}}}} \\\hline
		99. Open the stopwatch (\textcolor{accent}{\MakeUppercase{\textbf{enter + w}}}).                                                                                                                 &      &      &                             \\\hline
		100. Start or stop the stopwatch (\textcolor{accent}{\MakeUppercase{\textbf{space}}}).                                                                                                           &      &      &                             \\\hline
		101. Clear the stop watch (\textcolor{accent}{\MakeUppercase{\textbf{backspace + c}}}).                                                                                                          &      &      &                             \\\hline
		102. Copy a stopwatch time onto the clipboard to paste into a document (\textcolor{accent}{\MakeUppercase{\textbf{backspace + k}}}).                                                             &      &      &                             \\\hline
		\multicolumn{4}{|>{\centering\arraybackslash}m{15cm}|}{\textcolor{heading}{\MakeUppercase{\textbf{ADVANCED SKILL LESSON 28}}}} \\\hline
		103. Enter the menu to change the font (\textcolor{accent}{\MakeUppercase{\textbf{enter + f}}}).                                                                                                 &      &      &                             \\\hline
		104. Enter a desired letter to change the font.                                                                                                                                                  &      &      &                             \\\hline
		105. Turn an option on or off (\textcolor{accent}{\MakeUppercase{\textbf{n for on}}} and \textcolor{accent}{\MakeUppercase{\textbf{f}}} for off).                                                &      &      &                             \\\hline
		106. Leave a font item unchanged (\textcolor{accent}{\MakeUppercase{\textbf{enter}}}).                                                                                                           &      &      &                             \\\hline
		\multicolumn{4}{|>{\centering\arraybackslash}m{15cm}|}{\textcolor{heading}{\MakeUppercase{\textbf{ADVANCED SKILL LESSON 29}}}} \\\hline
		107. Open and exit the FM tuner.                                                                                                                                                                 &      &      &                             \\\hline
		108. Make adjustments to the FM tuner as desired.                                                                                                                                                &      &      &                             \\\hline
		\multicolumn{4}{|>{\centering\arraybackslash}m{15cm}|}{\textcolor{heading}{\MakeUppercase{\textbf{ADVANCED SKILL LESSON 30}}}} \\\hline
		109. Insert the date and time into a document.                                                                                                                                                   &      &      &                             \\\hline
		110. Insert a calculation into a document.                                                                                                                                                       &      &      &                             \\\hline
		\multicolumn{4}{|>{\centering\arraybackslash}m{15cm}|}{Independent Total: } \\\hline
	\end{longtable}
}

\clearpage
\section{BrailleNote Apex Overall Skills Checklist: Student Version}
{
	\renewcommand{\arraystretch}{1.5}
	\begin{table}[!htbp]
		\begin{tabular}{|m{3.75cm}m{3.75cm}m{3.75cm}m{3.75cm}|}
			\multicolumn{2}{b{7.7cm}}{Student: \hrulefill} & \multicolumn{2}{b{7.6cm}}{Grade: \hrulefill} \\[.5em]
			\multicolumn{2}{b{7.7cm}}{School: \hrulefill} & \multicolumn{2}{b{7.6cm}}{TSVI: \hrulefill} \\[.5em] \hline
			\multicolumn{4}{|>{\centering\arraybackslash}m{15.5cm}|}{Rating Guide} \\\hline
			\footnotesize{I = Independent Skill} & \footnotesize{P = High Prirority Need} & \footnotesize{F = Future Need} & \footnotesize{NA = Not Applicable} \\ \hline
		\end{tabular}
	\end{table}
}
\vspace*{-.70cm}
{
	\renewcommand{\arraystretch}{1.5}
	\begin{longtable}[!htbp]{|m{10.0cm}||m{1.5cm}|m{1.5cm}||m{2.0cm}|}
		\hline
		BRAILLENOTE APEX SKILL & \multicolumn{2}{|>{\centering\arraybackslash}m{3.0cm}||}{Rating} & Passed  \\\hline
		                                                         & Student & TSVI &   \\\hline
		\endfirsthead
		\multicolumn{4}{r}%
		{\textit{Continued from previous page}} \\\hline
		\hline
		BRAILLENOTE APEX SKILL & \multicolumn{2}{|>{\centering\arraybackslash}m{3.0cm}||}{Rating} & Passed  \\\hline
		                                                         & Student & TSVI &   \\\hline
		\hline
		\endhead
		\hline \multicolumn{4}{r}{\textit{Continued on next page}} \\\hline
		\endfoot
		\hline
		\endlastfoot
		1. Navigate among menus.                                 &         &      &   \\ \hline
		2. Basic Word Processing (writing, editing).             &         &      &   \\ \hline
		3. Advanced Word Processing (spell check, etc).          &         &      &   \\ \hline
		4. Switching between open documents and applications     &         &      &   \\ \hline
		5. Use Help and the User Guide                           &         &      &   \\ \hline
		6. Complete e-mail tasks                                 &         &      &   \\ \hline
		7. Make use of the Address Book                          &         &      &   \\ \hline
		8. Organize by using the Planner                         &         &      &   \\ \hline
		9. Display basic Internet skills                         &         &      &   \\ \hline
		10. Download books from Internet to BrailleNote          &         &      &   \\ \hline
		11. Utilize the BrailleNote calculator                   &         &      &   \\ \hline
		12. Read books from within the Book Reader               &         &      &   \\ \hline
		13. Understand and use the Options Menu                  &         &      &   \\ \hline
		14. Play games during leisure time                       &         &      &   \\ \hline
		15. Use the dictionary                                   &         &      &   \\ \hline
		16. Print and/or emboss a document                       &         &      &   \\ \hline
		17. Lean and enhance skills by using the Nemeth Tutorial &         &      &   \\ \hline
		18. Explore use of the Stopwatch                         &         &      &   \\ \hline
		19. Demonstrate appropriate use of the FM tuner          &         &      &   \\ \hline
		20. Use the File Manager and Utilities Menus             &         &      &   \\ \hline
		\hfill Total Independent                                 &         &      &   \\ \hline
		\hfill Total Priority Need                               &         &      &   \\\hline
		\hfill Total Future Need                                 &         &      &   \\\hline
	\end{longtable}
}
\part{BrailleNote Touch}
%%%%%%%%%%%%%%%%%%%%%%%%%%%%%%%%%%%%%%%%%%
%%%%%%%%%%%%%%%%%%%%%%%%%%%%%%%%%%%%%%%%%%
%                                        %
%                                        %
%                                        %
%%%%%%%%%%%%%%%%%%%%%%%%%%%%%%%%%%%%%%%%%%
%%%%%%%%%%%%%%%%%%%%%%%%%%%%%%%%%%%%%%%%%%

\section{Overview of the BrailleNote Touch}

The BrailleNote Touch is a tablet with a Braille display. It comes in a protective carrying case that includes a Perkins-style keyboard.
\begin{itemize}
	\item The keyboard
	\begin{itemize}
		\item Six keys for Brailling
		\item Enter key on the right
		\item Backspace key on the left
	\end{itemize}
	\item The front edge of the tablet contains four thumb keys. From left to right:
	\begin{itemize}
		\item Previous
		\item Pan left
		\item Pan right
		\item Next
	\end{itemize}	
	\item In the middle of the front edge are three additional keys. From left to right:
	\begin{itemize}
		\item Triangular-shaped Back button
		\begin{itemize}
			\item Moves back one web page
			\item Functions as an Escape key
			\item Moves up through the directory hierarchy when using the File Manager.
		\end{itemize}
		\item Round Home button
		\begin{itemize}
		\item 	Takes you to the Main Menu. 
		\item 	Triple tap of the Home button unloads or loads Keysoft.
		\end{itemize}
	\item 	Square context key
		\begin{itemize}
			\item Tapping the square context key opens a context menu for any program that is running. 
			If you hold the context key down for about two seconds, a list of all running applications appears so that you can switch to any open application. 
		\item 	The applications list also contains a Clear All button which can be used to free up memory.
		\end{itemize}
	\end{itemize}
	\item The back of the machine, from left to right: 
	\begin{itemize}
		\item USB port
		\item SD card slot
		\item port for an HDMI monitor.
	\end{itemize}
\item 	The right edge, from back to front:
	\begin{itemize}
		\item a microphone jack
		\item a headphone jack distinguished by a slight ridge around it
		van Action button with a dot
		\begin{itemize}
			\item used to stop or start reading when using the Victor Reader
		\item 	used to scan a document when using KNFB reader.
		\end{itemize}
	\end{itemize}
	\item On the left edge of the machine, from back to front: 
	\begin{itemize}
		\item Micro USB power port
		\begin{itemize}
		\item 	The BrailleNote Touch comes with a 6-foot micro USB cable and a wall charger. Plug the larger end of the USB cable into the wall charger and the small end of the cable into the power jack with the points facing down.
		\end{itemize}
		\item Concave-shaped Power/lock button
		\begin{itemize}
			\item To start the BrailleNote Touch, hold down the Power button for about two seconds. You will see a Starting message in Braille, then a dot moving in a circular motion. 
			When the machine boots up, press space+U to unlock the screen. 
			\item You can lock or unlock the screen of the BrailleNote Touch by tapping the Power/lock button.
		\end{itemize}
		Volume Up and Volume down keys.
		\begin{itemize}
		\item 	The volume controls offer a quick way to increase or decrease the volume. 
		\item 	They do the same thing as Enter+dot4 to increase volume, and Enter+dot1 to decrease volume.
		\end{itemize}
	\end{itemize}
\end{itemize}
\subsection{Starting the BrailleNote Touch}
	Hold down the Power/lock button on the left side of the BrailleNote Touch for about two seconds. 
	You will feel a quick vibration, and the Braille display will turn on. 
	After 10 or 15 seconds, a Braille dot will move in a circular motion showing that the unit is starting. 
	When you hear the Screen Locked message, press space+U to unlock the screen. 
	When you want to lock the screen in order to save power, just tap the Power button; the Braille display will go blank. 
	To unlock, tap the Power button again; then press space+U to unlock.

\subsection{Starting the Quick Tutorial}
You can turn on the Quick Tutorial at any time as follows.
	Press space+O for Options. 
	Press M then Enter to open Miscellaneous Settings. 
	Press L then Enter to launch the tutorial.
		You can now navigate through the tutorial by pressing the Next thumb key. 
		You can exit the tutorial at any time by pressing space+dots4,5,6 to move to the end of the tutorial; then press Enter on the Exit Tutorial button.
	At this point, press the triangular-shaped Back button on the front of the machine to return to the Main Menu.

%%%%%%%%%%%%%%%%%%%%%%%%%%%%%%%%%%%%%%%%%%
%%%%%%%%%%%%%%%%%%%%%%%%%%%%%%%%%%%%%%%%%%
%                                        %
%                                        %
%                                        %
%%%%%%%%%%%%%%%%%%%%%%%%%%%%%%%%%%%%%%%%%%
%%%%%%%%%%%%%%%%%%%%%%%%%%%%%%%%%%%%%%%%%%
\clearpage

\section{The Word Processor}
	At the Main menu, open the Word Processor \dotfill letter w, then enter
		A menu appears with options 
			Create \dotfill letter c, then enter
			Open \dotfill letter o, then enter
			Print \dotfill letter p, then enter
			Emboss \dotfill letter e, then enter
			Settings \dotfill letter s, then enter
		Settings includes several options including 
			Manage personal dictionary, 
			Spell checker language and an 
			Option to set the Embosser IP Address to an embosser that supports embossing to the Cloud
		When done writing your document
			Exit and open the Save As dialog \dotfill space+e
				When prompted for a file name, type it and press enter
				If this is the first time that you have saved a document, it will be saved in the Documents folder on the Storage drive. 
				Open Save As dialog withotu exiting \dotfill backspace+s
	Following is a list of commands that you can use to move around your document
		Move to top of document \dotfill space+dots123
		Move to bottom of document \dotfill space+dots456
		Move to previous character \dotfill space+dot3
		Move to next character \dotfill space+dot6
		Move to previous word \dotfill space+dot2
		Move to next word \dotfill space+dot5
		Move to previous line \dotfill space+dot1 or Previous thumb-key
		Move to next line \dotfill space+dot 4 or Next thumb-key
		Move to previous paragraph \dotfill space+dots23
		Move to next paragraph \dotfill space+dots56
		Move to beginning of line \dotfill Enter+dots13
		Move to end of line \dotfill ENTER+dots46
		Page up \dotfill space+dots126
		Page down \dotfill space+dots345
		Pan Braille to the left \dotfill Left thumb-key
		Pan Braille to the right \dotfill Right thumb-key
		Change navigation level \dotfill space+t 
		Navigate to next navigation element \dotfill space+dots46
		Navigate to previous navigation element \dotfill space+dots13
	When you are writing UEB or English contracted Braille, the editing cursor is displayed as dots7,8. If you are editing computer Braille, the editing cursor is indicated by dot8 alone.

%%%%%%%%%%%%%%%%%%%%%%%%%%%%%%%%%%%%%%%%%%
%%%%%%%%%%%%%%%%%%%%%%%%%%%%%%%%%%%%%%%%%%
%                                        %
%                                        %
%                                        %
%%%%%%%%%%%%%%%%%%%%%%%%%%%%%%%%%%%%%%%%%%
%%%%%%%%%%%%%%%%%%%%%%%%%%%%%%%%%%%%%%%%%%
\clearpage
\section{Reading the Document Using Speech}

To start reading through your document using speech \dotfill space+g
To stop reading \dotfill backspace+enter. 
	the line on which you stopped reading will be displayed in braille.
	Continue reading through the document \dotfill right panning key
To continue reading using speech
	Route the cursor to your current braille location \dotfill cursor routing button
	Continue reading with speeech \dotfill  press space+g 
%%%%%%%%%%%%%%%%%%%%%%%%%%%%%%%%%%%%%%%%%%
%%%%%%%%%%%%%%%%%%%%%%%%%%%%%%%%%%%%%%%%%%
%                                        %
%                                        %
%                                        %
%%%%%%%%%%%%%%%%%%%%%%%%%%%%%%%%%%%%%%%%%%
%%%%%%%%%%%%%%%%%%%%%%%%%%%%%%%%%%%%%%%%%%
\clearpage
\subsection{Creating a New folder on the SD card}
From Main menu, open File Manager \dotfill letter f, then enter
	You will land in the last folder that was open.
To move among folders \dotfill triangular back button
	The BrailleNote Touch will say something like ``alarms folder," meaning that you are in the list of folders on the internal storage drive.
To move select internal deive or external SD card \dotfill triangular back button
	For internal storage \dotfill enter
	For SD Card \dotfill space+dot4 then enter. 
		You will now be on a list of the folders that are located on the SD card.
To create new folder \dotfill space+n
	Type desired name of folder, then enter.
Return to the Main Menu \dotfill triangular Back button twice
%%%%%%%%%%%%%%%%%%%%%%%%%%%%%%%%%%%%%%%%%%
%%%%%%%%%%%%%%%%%%%%%%%%%%%%%%%%%%%%%%%%%%
%                                        %
%                                        %
%                                        %
%%%%%%%%%%%%%%%%%%%%%%%%%%%%%%%%%%%%%%%%%%
%%%%%%%%%%%%%%%%%%%%%%%%%%%%%%%%%%%%%%%%%%
\clearpage
\section{Saving a Document in the New Folder}
	At the Main Menu, open Word Processor \dotfill letter w, then enter
	Create a new document \dotfill letter c, then enter
	You can save this document even before you write anything
		Open Save dialogue \dotfill space+s.
			Type a new name for the document, but do not press enter. Notice that the edit box has changed to show your new document name. But if you press Enter, it will be saved in your Documents folder.
		Display current location \dotfill Previous thumb-key one time
		To open locaton dialogue \dotfill Previous thumb-key, then enter
		Toggle through options to Drive Selection \dotfill triangular Back button
			Toggle internal drive or SD Card \dotfill space+dot4 then press enter
			Press the first letter of the desired folder, then enter
			Open ``select this folder" dialogue \dotfill letter s, then enter
			Open the save dialogue \dotfill Next thumb-key, then enter
\subsection{Task}
	At the Main Menu, open Word Processor \dotfill press w, then enter 
	Create a file \dotfill letter c, then enter
	Open Save As dialogue \dotfill space+e
	Type file name but do not press enter 
	Display current location \dotfill Previous thumb-key one time
	To open locaton dialogue \dotfill Previous thumb-key, then enter
	Toggle through options to Drive Selection \dotfill triangular Back button
		Open the Storage folder list \dotfill enter
	Toggle through options to Drive Selection \dotfill triangular Back button
			Toggle internal drive or SD Card \dotfill space+dot4 then press enter
			Press the first letter of the desired folder, then enter
			Open ``select this folder" dialogue \dotfill letter s, then enter
			Open the save dialogue \dotfill Next thumb-key, then enter
	You are now back in the Save As dialog, and you are located on the file name.
	Activate the save dialogue \dotfill Next thumb-key twice, then enter
	Your new document has now been saved in the Documents folder, and you are ready to write down any messages.
	All additional files that you create will continue to be saved in the Documents folder until the next time you change the location.
%%%%%%%%%%%%%%%%%%%%%%%%%%%%%%%%%%%%%%%%%%
%%%%%%%%%%%%%%%%%%%%%%%%%%%%%%%%%%%%%%%%%%
%                                        %
%                                        %
%                                        %
%%%%%%%%%%%%%%%%%%%%%%%%%%%%%%%%%%%%%%%%%%
%%%%%%%%%%%%%%%%%%%%%%%%%%%%%%%%%%%%%%%%%%
\section{Using the File Manager to move or copy files}

Many users will prefer to save all of their documents in the Documents folder without ever changing the location in which files are saved. They can then create folders wherever desired using the File Manager, and then copy or move those files to the new folder.

In the following example, create a ``Work" folder on your SD card. Then, quickly save a new phone messages file in the Documents folder, then move it to the Work folder.

\subsection{Creating the Folder}

At the Main Menu, press F followed by Enter to open the File Manager.
Press the Back button two times to get to Drive selection.
Press space+dot4 to get to the SD card and press Enter to open it.
Press space+N for New Folder; then type ``work" and press Enter to create it.
Press the Back button two times to return to the Main Menu.

\subsection{Creating the Phone Messages Document}

At the Main Menu, press W followed by Enter to open the Word Processor; then press C followed by Enter to create a new document.
Press space+S; then type ``phone Messages," and press Enter. The document will be Saved in the last location where files were saved; in this case, the Documents folder.
Press space+E two times to return to the Main Menu.

\subsection{Copying the Phone Messages Document to the Work Folder}

At the Main Menu, press F followed by Enter to open the File Manager.
Press D a couple of times to get to the Documents folder; then press Enter.
Press the letter P to get to ``Phone Messages.docx."
Press Backspace+X to cut the file.
Press the Back button two times to get to Drive selection.
Press space+dot4 to arrow down to the SD card; then press Enter.
Press W followed by Enter to open the Work folder.
Press Backspace+V to paste the file into the Work folder.
Press Home to return to the Main Menu.
You now have a handy place to save phone messages.

\subsubsection{Deleting Files}

For practice, delete any file as follows:
From the Main menu, press F, then Enter, to open the File Manager.
Press Enter to Open Storage.
Press D until you get to Documents, then press Enter.
Choose a file to delete.
Press Backspace+dots2356; you will be asked if you want to delete the file.
Press O for OK, then Enter to delete the file.

\subsubsection{Renaming Files}

When renaming files, it is best to use computer Braille. This allows inclusion of numbers or dates in file names if desired. It also helps with file extensions.

Select the file and Press Backspace+R for Rename.
Press Backspace+G to switch to computer Braille.
Write the new file name, (including the extension), using computer Braille; then press Enter to complete the process.
Remember to press Backspace+Space+G to switch back to literary Braille.

\subsection{Opening Multiple Documents}

New in the version 3.0 firmware is the ability to work with multiple documents.
Create a new document called ``Practice1." When using UEB grade 2, you would write the file name as "dot6 practice Number sign 2."
Press the context button; then press Enter to open File Functions.
Choose Create new additional document and press Enter. The shortcut is Backspace+N.
Press space+S; then type Practice2 and press Enter to create a second document.
Press Backspace+N and save a third document as ``Practice3."
Press space+dots1,2,5,6 for Switch Documents.

You can now copy or move text between the documents. This allows users to take notes on a document that they are reading.
Press space+E or the Back button to close out of the documents one at a time.

\subsection{Nemeth on the BrailleNote Touch}

Create a new document in the Word Processor and save it as Math Homework.
At the top, type ``Math Homework," and press Enter.

Center the title as follows:
Arrow up to the title by pressing space+Up Arrow.
Press the Context button.
Arrow down to Format Functions and press Enter.
Press Enter on the first choice which is Alignment.
Arrow down to Align Center, (Enter+C), and press Enter.
Press space+dots4,5,6 to move to the bottom of the document.
Write the following:

Problem 1. Compute the area of a circle having a radius of 9 feet.

Press Backspace+M for the math editor. In the math editor, you can write up to 10 lines of math.
Press Backspace+M to start the math editor.
Write the following:

When done, press Backspace+E to export your work to the clipboard. You will be returned to the Word document.
Press Backspace+V to paste the math into your document.
Now, type Answer: 
Press the Home button to go to the Main Menu.
Press C followed by Enter to start the calculator.
Using computer Braille, type 3.141592654$\ast$ 81 and press Enter.
Press space+Enter+dots1,2,3,4,5,6 to select the result.
Press Backspace+space+Y to copy the result to the clipboard.
Hold down the Context key, choose KeyWord Menu, and press Enter. You will be returned to the document.
Press Backspace+V to paste the answer into your document. See below.


Math Homework
A = pi$\ast$ r$ \string^ $ 2
A = 3.141592654$\ast$ 9$ \string^ $ 2
A = 3.141592654$\ast$ 81
Answer: 254.469005

Problem 2. Solve the following equation where Y = 2.
X$ \string^ $ 2+y$ \string^ $ 2 = 29
X$ \string^ $ 2+4 = 29
X$ \string^ $ 2 = 25
X = 5
See Nemeth printouts at end of document.


Lookup table. If you need to look up a math expression while using the math editor, press Backspace+dots3,5 to open the lookup feature. If you locate an item in the lookup table, pressing Enter will insert it into the equation editor.

Note. If you need to edit a math expression after pasting it into your document, do the following:
Route the cursor to the image using a cursor router button.
Press Backspace+M to load The image into the math editor.
Make any necessary changes.
Press Backspace+E to export the math to the clipboard.
Route your cursor to the paragraph following the image.
Press Backspace until the math is deleted.
Press Backspace+V to paste the corrected image into your document.

\subsubsection{Viewing the Result}

Press Enter+V for Preview.
Make sure that Drive PDF Viewer is checked.
Press J then Enter for Just Once.
Open the keyboard cover to display the problem.

\section{Setting up an Email Account on the BrailleNote Touch}

At the Main Menu, press E followed by Enter to start KeyMail. You will be prompted to set up an Email account when the Email app is started for the first time. You can have multiple Email accounts on the BrailleNote Touch. In the notes that follow, I will explain how to add a Gmail account.

Press E followed by Enter to start Email.
Press the square context button on the front of the machine.
Press space+dot4 several times to arrow down to the Add Account option; then press Enter.
You will be prompted for an Email address which must be written using computer Braille. So, use the Next or Previous thumb keys to navigate to the input field, and press one of the cursor routing buttons above the edit box to make sure you are in the edit field. Type the Email address using computer Braille. The At sign is entered by pressing dot4+dot7. The period is entered as dots4,6. Any numbers must be dropped Nemeth numbers, and no contractions are allowed.

Press the Next thumb key to move to the Password field, and enter the password using computer Braille.

Use the Next thumb key to move to the Next button, and press Enter. You will receive a ``Validating Server Settings," message, and you will then be placed in the Account Settings list.
Use Down Arrow and Up Arrow, (space+dot4 and space+dot1), to arrow through the list of settings. These include:
Inbox checking frequency" Every 15 minutes.
Notify me when Email arrives: This is checked by default. Many users uncheck this by pressing a cursor routing key above the checkbox so that they will not have interruptions.
Sync Email from this account: Leave this checked.
Automatically download attachments when connected to Wi-Fi: Leave this checked.

At this point, move to the Next button and press Enter. You will receive an ``Updating account," message.

Press the Next thumb key. You will be prompted to give the account a name. You can give it a name like ``Personal account," or just accept the default which is the Email address.

Press the Next thumb key. You will be prompted to insert the name that will be displayed on outgoing messages. The default will be the name that was used when the BrailleNote Touch was set up initially, but you can enter another name if necessary.

Move to the Next Button and press Enter. You will be returned to the Email menu. If you had a prior Email account, you will be returned to that account which is the default. For now, press space+E to exit from Email.

\subsection{Switching to Your New Email Account}

Press E followed by Enter to start Email.
Press the letter C to get to Current Account; then press Enter. You now have two choices.
Press space+dot4 to arrow down; press space+dot1 to arrow up.
So, choose your new account and press Enter.

Now, when you press Enter on the first menu option which is New Message, you will be creating a new Email message in your new account.
If you press Enter on the Read choice, you will land in the Inbox for your new account.

\subsection{Switching Back to Your Primary Account}

Press the Back button, (the left-pointing triangle on the front of the machine, to return to the Keymail menu.
Press C for Current Account; then press Enter.
Arrow to your primary account and press Enter. You will now be in the menu for your primary account.

\subsection{Sending and Receiving Email Messages}

Like the other Keysoft applications, the Keymail app can be run using either the context menus or the shortcuts which are displayed in the context menu.

\subsubsection{Sending an Email Containing an Attachment}

Press E then Enter to start Email.
Press Enter on the New Message option which is the first choice.
Type the recipient's Email address in the To field using computer Braille; then press the Next thumb key to get to the Subject field. When you route the cursor into the subject field, dots7,8 indicates that literary Braille can be used. So, type the Subject; then press the Next thumb key to get to the Compose edit box. Again, routing the cursor into this field indicates that literary Braille can be used.
Type your message. At this point, if you just want to send the message, press Backspace+S to send it.
But we need to add an attachment.
Tap the square context button.
Arrow down to the Attach file option; then press Enter. You can locate a file to attach in your documents folder as follows:
Press the Triangular Back button until you get to Drive Selection.
Press Enter on Storage.
Press the letter D until you get to the Documents folder; then press Enter.
Use first-letter navigation or arrow down to the file that you wish to attach; then press Enter. The file will be attached, and you will be returned to the Compose field.
Pressing the Previous thumb key will show that an Open Attachments View field has been added.
Press Backspace+S to send the message containing your attachment. You will be returned to the Keymail menu.

\subsubsection{Receiving Messages Containing Attachments}

Press E then Enter to start Email.
Press R for Read; then press Enter.
Arrow down until you come to an Email that ``has attachments," then press Enter.
Press the Previous thumb key until you get to the ``Open Attachments View," button and press Enter.
Or, tap the square context button and press Enter on the first choice which is ``Toggle View attachments."
Arrow down to the attachment; then press Enter. You will land on a list of three items including Open, Save, and Download again.
Choose Save and press Enter.
The attachment will be saved in the Downloads folder. You can open it as follows:
Press Home to go to the Main Menu.
Press the letter F then Enter to open the File Manager.
Press the Back button until you get to the root of the storage area. When you are there, you may hear something like ``Alarms," which is an Android folder at the root of the internal storage drive.
Press the letter D several times to get to the Download folder; then press Enter.
Locate the attachment; then press Enter to open it.
Or, you can press Control+X to cut the attachment to the clipboard and then paste it into another folder. 

\subsection{Handy Email Shortcuts}

Create new message: space+N.
Change Email folder: Enter+B.
Search: space+F.

Move message to another folder: While in Inbox, tap Context button, then locate Move To and press Enter. Choose the folder to which you wish to move the message; then press Enter.

Send message: Backspace+S.
Mark message: dots1,2,3,7.
Delete single message or marked messages: Backspace+dots2,3,5,6.

Add Cc/Bcc fields: When in Compose field, tap Context button, locate Cc/Bcc choice; then press Enter. You can now press your Previous thumb key several times to locate your Cc and Bcc fields. After filling them in, press the Next thumb key several times to return to the Compose field; then press Backspace+S to send the Email.

Refresh Email: While in the Inbox, locate Refresh in the context menu, or press Enter+R for Refresh.

\section{Applications for the Classroom}

Popular applications that can be installed on the BrailleNote Touch include Simple Dictionary for word look-up, Go Read for downloading books from Bookshare, Powerpoint from Microsoft for viewing Powerpoint documents, Easy PDF for converting PDF documents to Word format, HP Print Service Plugin for printing to HP printers using the USB port, and Easy Unrar for unzipping archives. In the following sections, I will show how to download an application from the Play Store, and then use these applications.
these applications.

\subsection{Simple Dictionary}

\subsubsection{Installing the Simple Dictionary}

If this app has not been pre-installed, you can install it once the student's school account has been set up. Do this as follows:

At the Main Menu, press the letter P until you get to Play Store; then press Enter.
Press S to get to the Search button; then press Enter.
In the Search box, type ``simple dictionary," without the quotation marks; then press Enter.
Arrow down to the Simple Dictionary app.
Arrow down one more time to get to the Options button; then press Enter.
Choose Install and press Enter.
When prompted for App Permissions, press space+dots4,5,6 to get to the Accept button at the bottom of the screen and press Enter.
When installation is complete, you will receive the "Successfully installed Simple Dictionary message.

\subsubsection{Using the Word processor and File Management with Simple Dictionary}

At the Main menu, press W followed by Enter to open the Word Processor menu.
Press C followed by Enter to create a new document.
Press Space+S to open the Save As dialog.
Type ``Dictionary Word List," for the file name; then press Enter. By default, the document will be saved in the Documents folder unless the location for saving files has been changed. You are now ready to write your document.

For the title, write ``Word Lookup Assignment," and press Enter two times to create a new paragraph.
Press Space+S to save the changes.
Press Space+E to close the word processor.

\subsubsection{Using File Manager to Create a Folder for Class Documents}

At the Main menu, press F then Enter to open the File Manager.
Press Enter to open Storage; then press D until you get to Documents; then press Enter.
Press W to get to the Word Definitions document: then press Backspace+X to cut it to the Clipboard.
Press the triangle-shaped Back button until you get to Drive Selection; then arrow down to the SD Card and press Enter.
Press Space+N for New Folder; then type ``English," without the quotation marks and press Enter to create the folder.
At this point, press Enter to open the folder; then press Backspace+V to paste the word definitions document into the English folder.
Now, press Enter on the ``word definition.docs," file to open it.

\subsubsection{Using the Dictionary}

Open the dictionary program as follows:
Press Home to go to the Main Menu.
Press A then Enter to open All applications.
Press D then Enter to open the dictionary.
In the search box, type ``ubiquitous" and press Enter.
You can now read through the definition.

While reading the definition, press a cursor router button. Buttons will appear that allow you to copy either the word or its definition. Press the Next thumb key a couple of times to get to the Copy Definition button; then press Enter. The definition will be copied to the clipboard.

At this point, hold down the square context key to show all applications.
Choose KeyWord and press Enter.
Press Backspace+V to paste the definition into your document.

\subsection{Using Go Read to Search for Bookshare Books}

At the Main Menu, press A followed by Enter to open All Applications.
Press G until you get to Go Read; then press Enter.
Press O followed by Enter to access the Open Drawer menu.
Press S to locate the Search Bookshare option; then press Enter.
Arrow down through the search options which include:
Title Search;
Author Search;
ISBN Search;
Latest Books;
Popular Books, and
Newspapers and Magazines.

For this example, choose Title Search and press Enter.
In the edit box, type ``Call of the Wild," and press Enter.
The second result is Call of the Wild by Jack London. Choose this one and press Enter. Book Details will open.
Use the Next thumb key to navigate through the book details.
These include Title, Author, Download (text only)
Download book with images
ISBN: and
Language:
 
If you decide that you want another book, press the Back key, (the triangular button on the front of the Touch, to move to the Previous page.

For now, choose Call of the Wild. To do this, use the Next or previous thumb keys to choose the Download (text only) button and press Enter. You will hear a message when the book has been downloaded.

Now, press the Home button to return to the Main Menu.

\subsubsection{Using Victor Reader}

At the Main Menu, press V followed by Enter to open Victor Reader.
Press space+B to open the Bookshelf.
Press space+dot4 two times to arrow down to Talking books; then press Enter.
Arrow to the book that was downloaded, Call of the Wild, then press Enter to open it.
Tap the square context menu button located on the front of the Touch.
The following menu appears:
Bookshelf menu: space+B.
Go to options menu: Enter+G.
Bookmark menu: Enter+M.
Navigation level menu: space+T.
Move to previous element: space+dots1,3.
Move to Previous element while playing: Previous Thumb Key.
Move to next element: space+dots4,6.
Move to next element while playing: Next thumb key.

\subsubsection{Reading the book}

When the book opens, press space+T to open the navigation level menu. You will land on level 1. Press Enter to accept navigation by heading 1.
At this point, press space+dots4,6 to move forward one section at a time.
The following headings will be displayed in Braille:
Contents.
Title Page.
Introduction.
Chapter 1.
Chapter 2.
And so forth.

Note that, as you move forward by pressing space+dots4,6, the headings are shown in Braille, but are not spoken.

When you get to the point where you wish to start reading using speech, press the Action button located on the right edge of the tablet. This button has a small dot, and is located just below the headphone jack. Pressing this button one time starts reading with speech; pressing it again stops reading. Alternatively, you can use space+G to start reading. When you wish to stop reading, press Backspace+Enter to stop.

For practice, choose Chapter 1 and press the Action button. When you wish to stop reading, press the Action button to stop.

Suppose that you are reading with speech, and wish to skim through the book.
Press the Action button to start reading.
Now, while the book is being read, press Space with dots4,6 to move to the next section. The book will continue reading at the new section.

\subsubsection{Bookmarks}

Normally, when you return to a book that you have been reading, reading will be resumed where you left off. However, it is a good idea to use bookmarks to keep you from losing your place.

You can create a bookmark as follows:
Open the bookmark menu from the menu, or press Enter with M.
Press the letter I followed by Enter to insert a bookmark.
Type a 1 followed by Enter to insert Bookmark 1. You must use computer Braille, so this would be a dot2.
Press space with dots1,2,3 to return to the beginning of the book.
Press Space with M; then G followed by Enter to open the Go to Bookmark option.
Type 1 followed by Enter to go to bookmark 1. The message, ``bookmark 1,"
will appear for a couple of seconds. Then, you will be at the point where you stopped reading.

Note that you can use the panning keys to pan forward or backward through the book using Braille.

When done reading, press Space with E to exit from Victor Reader and return to the Main Menu.

\subsection{Converting PDF Documents to Word Using Easy PDF}

Suppose that you have a PDF file in Google Drive that you need to convert.
Press A then Enter to open All Applications.
Press D until you get to Drive; then press Enter.
In this example, I am looking for a file called ``The Outsiders Comprehension Questions." So, press the letter T to get to it.
Once the file is located, arrow down one time to get to the More Actions button; then press Enter.
Use space+dot4 to arrow down to the Download option; then press Enter. You will be notified that one file will be downloaded.

\subsubsection{Converting the File}

At the Main Menu, press A then Enter for All Applications.
Press E to get to EasyPDF; then press Enter.
Press P for PDF-to-word and press Enter.
An Open From message appears at the top of the screen, so arrow down to get to Downloads and press Enter.
choose the Outsiders Questions file and press Enter. You will be advised that the file is converting. When it is done, it will open in Word.

Save the file in your Documents folder as follows:
Press Backspace+S to open the Save As dialog.
Press the Previous thumb key two times to get to the Location button; then press Enter.
Press the triangular Back button several times to get to the list of folders in storage.
Press the letter D until you get to Documents; then press Enter.
Press S to get to the Select This Folder button; then press Enter.
Press the Next thumb key two times to get to the Save button; then press Enter. The word document will be saved in your documents folder.

You can now answer the questions, use Save As to add your name to the file name, and then share it back to google Docs as follows.
Open the Documents folder using File Manager.
Locate the file that you wish to share.
Tap the context button; press S for Share; then Enter.
Check the Save to Drive option.
Press J followed by Enter for Just once.
Press the Next thumb key until you get to the Save button; then press Enter to upload the file.

From time to time, it is a good idea to clear the folder where converted PDF files are normally saved.
Locate the EasyPDF app, but do not press Enter.
Tap the square context button and press Enter on Open App Info.
Press C to locate the Clear Data button; then press Enter. Now, press Enter on OK to clear the data.

\subsection{Using the Powerpoint App from Microsoft}

First, go into the Play Store and search for ``Powerpoint Microsoft."
Install the app.

For this example, I downloaded a Powerpoint called ``Sea Floor Spreading," that I had stored on Google Drive. So, I will open it from the Download folder.

Locate the file using File Manager; then press Enter.
You will be prompted to open the app using:
Make sure that Powerpoint is checked.
Press J for Just once; then press Enter.
When the file loads, press P to get to the Present button; then press Enter.
To navigate using gestures, do the following:
Press the Previous+Next thumb keys to turn off Touch Braille; then lift the keyboard cover to display the screen of the tablet.
When you touch near the top of the screen, the slide title will be spoken and displayed in Braille.
When you touch the slide content, you can pan through the text using your panning keys.
Swipe left with two fingers to move to the next slide.
Swipe right with two fingers to move to the previous slide.

When done, close the keyboard cover and press the Previous+Next thumb keys to turn Touch Braille on.
Press Home to return to the Main Menu.

\section{Printing to the HP Printer with a USB Cable}

\subsection{Setting up the HP Printer}

Install the HP Service Plugin from the Play Store. 
From the Main Menu, press the letter P two times to get to the Play Store, then press Enter. 
Choose Search Google Play; and press Enter. 
In the search box, type HP Service Plugin; then press Enter. 
Locate the HP Print Service Plugin app; click the Options button, Then click Install and accept permissions using the App Permissions button. The plugin will be installed. 
Return to the main menu. 

Press Space with O for Options. 
Press A for Android System Settings; then press Enter. 
Press P for Printing; then press Enter. Printing Services will open. 
Locate the HP Inc. Off setting and press Enter. 
Check the checkbox; then use your Next thumb key to locate the Ok button and press Enter. This will turn on HP print services. 

\subsection{Printing a Document}

Press the Home button to return to the Main menu. 
Plug in the printer's USB cable. You will be advised that a USB device has been attached. Press the Home button or space+dots1,2,3,4,5,6 to return to the Main menu. You can now print a document as follows. 

Press W for Word Processor; then press Enter. 
Press P for Print; then press Enter. 
Choose a document to print; then press Enter. You will receive a Complete Action Using message from the Android system. 
Make sure that Drive PDF Viewer is checked. 
Press J for Just Once; then press Enter. A PDF version of the document will be loaded. 
Press Space with M or the square context menu button. This will open the Send File option. 
Now, press P for print then press Enter to open the document and bring up the Print dialog. 
Press space+dots4,5,6 to get to the Print button at the bottom of the dialog, and press Enter. Your document will be printed. 

\subsection{Checking your Document with KNFB Reader}

Unplug the USB port; the BrailleNote will say USB device detached. 
At the Main menu, press K for KNFB Reader; then press Enter. 
After the program loads, you can use Space with dot 4 to get down to the Take picture option. Or, you can take a picture using the Action button on the right edge of the BrailleNote Touch. The Action button is easier if you are holding the BrailleNote. After a second or two, the text will appear in Braille. 

Note. To take the picture, you can close the BrailleNote keyboard case, and then hold the machine 8 to 10 inches above the page. Press the Action button to take the picture. 
Note on settings. 
You can check Tilt Guidance which will let you know if the Touch is being held level. 
Right below the Take Picture option is a Field of View Report option. You can press Enter or click the field of view report. If it says all 4 edges and all 4 corners are available, you will get the best picture. 

By default, the pages are not saved. There is a Save Document button, as well as automatic and batch modes. For these to work efficiently, you would want to set up a stand of some type. But even without a stand, KNFB Reader is a very handy way to make sure that your document actually printed, and it is very helpful for reading a handout that you might receive in class.

\section{Appendix A setting up the BrailleNote touch for the First Time}

To turn on the BrailleNote Touch, hold down the Power button for about 2 seconds. A ``Starting Keysoft," message will appear in Braille. After about 5 seconds, a Braille hourglass will appear represented by a dot moving in a circular motion. The first time you start the BrailleNote Touch, a Welcome message will appear. At this point, pressing the right-most thumb key will take you through an excellent tutorial that will help you get started. If you need to increase speech volume, press the Volume Up key several times.

To exit the tutorial, press Space+dots4,5,6; you will land on the Exit Tutorial button. Press Space+dot1 a couple of times to arrow up to the ``Do not launch this tutorial next time I Restart," checkbox, and check it by pressing a cursor router key above the box. Now, press Space+dots4,5,6 to return to the "Exit Tutorial," button and press Enter. This will exit the tutorial and open the Setup Wizard dialog. You can re-start the tutorial any time from the Options menu, (see below).

When you exit the tutorial, a Startup dialog will open. By default, the English language is selected – so press Space+dots4,5,6 to move to the end of the dialog; you will land on the Start button. Here, press Enter.

You will be prompted to select a Wi-fi connection. Arrow to your Wi-fi connection by using the arrow keys: Space+dot4 for Down Arrow or Space+dot1 for up arrow; then press Enter.

You will be prompted for a password which must be typed in computer Braille. Press a cursor routing key above the edit box; then type the password. An eight-dot computer Braille code is used. Letters are typed normally with no contractions. Use dot7 to capitalize a letter – for example, you would make a capital A by pressing dots1,7. dot7 is created by pressing the Backspace key. Numbers are dropped Nemeth numbers. Period is created using dots4,6. An At sign is created by pressing dots4,7. A computer Braille table is included as Appendix A.

After typing the Password, Press the Next thumb key until you get to the Connect button; then press Enter, or press one of the cursor routing keys above the Connect button.

When the connection has been established, you will receive a ``Got Google," message. Since most school accounts require that you use Chrome to log into the school account, I skip this account setup process and have a District IT representative set up the account. But for this example, I will show how to set up a standard Google account.

Press Space+dots4,5,6 to get to the end of the dialog and click the Yes button.
Enter the account Email and Password using computer Braille; then press Enter on the Next button.
When the Google Services message appears, press Space+dots4,5,6 to get to the OK button; then press Enter. You will get a ``Signing in" message.
When the Google Terms window appears, press space+dots4,5,6 to get to the bottom of the screen and press Enter on the Next button.

At this point, you will be prompted for the user's First and Last Name. These can be entered using either contracted or uncontracted Braille. This is important, because it is used to identify the machine and is automatically inserted when, for example, you create a new Email address. The user's first and last name is required even if you skip account setup.

Enter the first name; press the Next thumb key, and then enter the last name. When you press the Next thumb key after entering the last name, the account will be created and you will be at the Main Menu.

\subsection{Connecting to Wi-fi}

Students can quickly connect to Wi-fi at school or at home as follows:

At the Main Menu, Press Enter+Q for quick settings.
Press Space+dot4 to arrow down to Wifi disconnected and press Enter.
Arrow down to desired network and press Enter.
In the edit box, type the network password using computer Braille; then press the Next thumb several times to get to the Connect button and press Enter.

\subsection{Connecting to a School Account}

It is possible to set up the BrailleNote Touch and add the user's name, but skip account setup. To set up the account at a later time, press Space+O for Options, and then press A followed by Enter to open Android settings.
Press A two times to get to Add account; then press Enter.
Press G for Google; then press Enter.
Press E to get to the existing account button; then press Enter.
Enter the Email address using computer Braille; then press the Next thumb key to get to the password box and enter the password. Now, press the Next thumb key to get to the Next button and press Enter. On the next screen for the Google agreement, press Space+dots4,5,6 to get to the OK button and press Enter. On the next screen, Google Services, choose the Next button and press Enter.
At this point, press the Home button to return to the main menu.

\subsection{Note on School Accounts}

Many\ school accounts  require that you connect to the Internet using the Chrome browser before the account can be set up. An IT person can do this quickly as follows:
Triple-tap the Home button to unload Keysoft.
Click All Applications; then choose Chrome and complete the account setup.
Triple-Tap the Home button to Enable Keysoft.
You will now be able to download apps from the Play Store and share documents with Google Drive.
 
\subsection{User Settings}

Personalized preferences can be set up as follows:

Press Space+O for the Options menu.

Use Space+dot4 to arrow down to Configure Primary Language profile and press Enter. Leave the text-to-speech engine set to BrailleNote Touch Acapela for now.

Continue pressing the Next thumb key until you get to Preferred Braille Grade for entry. Leave this on the default which is Literary Braille.
Leave Preferred Braille Grade for Display set to Literary Braille.
Leave Computer Braille Table set to English US, Liblouis.

Now, when you press the Next thumb key, you get to Literary Braille Table, English UEB Duxbury grade 1. Press Enter to open available options.
Choose the first choice, English UEB Grade 2, then press Enter to select it.

Now, press the triangular Back button on the front of the machine, or press Space+E to return to the Options menu. And press the Next thumb key.

If you wish, you can configure the Secondary Language profile to English UEB Duxbury as well. That way, if someone presses Enter+L, (the profile toggle), and inadvertently loads the secondary English profile, the Braille setting won't change.

For now, arrow down to Keyboard Settings and press Enter.
The first option, Keyboard Echo, is set to both characters and words by default. Most students prefer either Word echo or no echo. For now, change it to Word echo as follows:
Press Enter on Keyboard Settings.
Arrow down to Words and press Enter to set it.
Now, arrow down to the Keyboard Clicks checkbox and press a cursor routing key above the box to uncheck it, (most students prefer a quiet keyboard when taking notes in class).
We don't need to reconfigure the thumb keys at this point, so press the Back button or Space+E to return to the Options menu.

Next, arrow down to Miscellaneous Settings and press Enter.
Arrow down to Navigation Sounds and press a cursor routing key above the box to uncheck it.

Tutorial. If you arrow down two more times, you will come to Launch Tutorial. If you press Enter on this option, the introductory tutorial will be loaded.

For now, press Down Arrow, (Space+dot4), two more times to get to Format Markers settings. This is an option that students may wish to change from time to time. The default is to include a $\$$ P as a marker whenever there is a hard line break; example, to show one or two blank lines between paragraphs. This looks like ``edp" in Braille. This is good for editing; for example, when changing paragraph settings while formatting a document, or when editing a document containing Nemeth where it may be necessary to delete a paragraph containing a math image.

However, when reading documents or books, most students prefer that a paragraph starts on a new line with a two-space indent at the left. You can set this as follows:

Press Enter to open Format Markers settings.
Arrow down to New Line Rendering, and press Enter. This is set, by default, to New Line Marker. Press Enter to change it.
Arrow down to the two spaces option; then press Enter to set it.

Now, arrow down about 7 times to get to Braille Message Display time in seconds. This determines how long a message is displayed in Braille. The default is 3 seconds; however, fast Braille readers may prefer 1 or 2 seconds. Press Enter. You will be prompted to enter a value of 1 to 30 seconds in computer Braille. For now, press 2 and then Enter to set it.

At this point, press the Back button or Space+E to return to the Options menu.

Here, press the letter A to get to Android System settings; then press Enter. We want to change the display duration setting. This is because the default setting of 2 minutes will cause the screen to lock whenever the BrailleNote Touch is inactive for two minutes. At this point, the student must tap the Power button and press Space+U to unlock the screen.
Press the letter D two times to get to Display; then press Enter. The default is shown as 2 minutes.
Now,\ press Enter to show the options.  Most students arrow down to 30 minutes and press Enter to set it. If this is done, it is important to lock the screen manually by tapping the Power button to lock the screen whenever the BrailleNote Touch will be inactive. This saves power.

We are done with Settings, so press the triangle-shaped Back button about five times to return to the Main Menu.

\section{Appendix B Tutorials}


\subsection{Using the BrailleNote Touch Tutorial with the Victor Reader App}

The BrailleNote Touch audio tutorial was developed by Mystic Access, and is available on the SD card that comes with the BrailleNote Touch. It is provided in both DAISY and MP3 format. You can listen to this tutorial as follows:
At the Main Menu, press V then Enter to open the Victor Reader Application.
Press space+B to open the Bookshelf menu.
Press T then Enter to open the Talking Books bookshelf. You will land in the booklist menu.
Press B !locate the BrailleNote Touch audio tutorial; then press Enter. The tutorial will start with an introductory message.
\subsubsection{The Victor Reader Context Menu}
Tap the square context menu button on the front edge of the machine to open the context menu. Use space+dot4 to move down through the menu; press space+dot1 to move up through the menu. A review of this menu shows the following choices:
Bookshelf menu: space+B.
Go to Options menu, space+G
Bookmark menu: Enter+M
Navigation level menu: space+T. Allows you to choose the element for navigating through the book. For example, you might move by heading or by page.
Move to Previous Element: space+dots1,3. 
Move to Previous element while playing: Previous thumb key.
Move to Next Element: space+dots4,6.
Move to Next Element while playing: Next thumb key.

The context menus allow students to use the programs right away; and they include helpful shortcuts for frequently-used functions. 
Programs with context menus include Victor Reader, Contacts, Planner, Email, Internet, Word Processor, File Manager, and the Calculator.
\subsubsection{Reading the Tutorial}
Press space+T to open the Navigation menu; then press Enter to choose level 1.
Press space+dots4,6 to move to the next section; audio reading will begin. You can pause by pressing Backspace+Enter, by pressing the Action button on the right edge of the machine, or by pressing space+G.
Now, press space+dots1,3 to move back to Part One, Getting to Know the BrailleNote Touch.
\subsubsection{Navigating by Subsection}
Press space+T to open the Navigation level menu.
Press space+dot4 to move down to level 2; then press Enter.
Now, pressing space+dots4,6 moves through the subsections of Part One.

\subsubsection{Changing the Speech Rate in Victor Reader}
You can press Enter+dot6 to increase the speech rate.
Press Enter+dot3 to decrease the speech rate.
This allows students to choose the speed that is most comfortable for the material that they are studying.
\subsubsection{Creating a Bookmark in Victor Reader}
 The shortcut for the Bookmark menu is Enter+M. Press Enter+M to open the Bookmark menu. Press space+dot4 to arrow down to the Insert Bookmark choice; then press Enter. You will be prompted to enter a number in computer Braille. For now, type a computer Braille 1, <dot2>, and press Enter. You will be advised that bookmark 1 was inserted.
To test the bookmark, press space+dots4,6 a couple of times to move away from the bookmark. Now, return to the bookmark as follows:
Press Enter+M to open the Bookmark menu. Press space+dot4 to arrow down to the Go to Bookmark option and press Enter.
Type a 1 using computer Braille; then press Enter. Now, when you press the Action button, reading will begin at the bookmark.

The Rewind function. When you return to a bookmark, it is a good idea to rewind a bit so that the text at the bookmark will be read in context. Press the left panning key on the front edge of the machine to rewind, or press the right panning key for fast forward. You can press Rewind or Fast Forward while you are reading.
\subsubsection{Moving to a Heading in Victor Reader}
First, choose the heading level that you want. Press Enter+G for Options. Choose the Go to Heading option and press Enter. Enter a heading number and press Enter. Press the Action button to start reading.

\subsection{Reading the MP3 tutorial.}
Press space+B for Bookshelf; then choose Other Audio. Press Enter on the BrailleNote Touch Mp3 tutorial. Press the Action button to start reading. When playing an Mp3 file, you can use the Next or Previous thumb keys to move forward or backward one file at a time.

\subsection{The User Guide}

The entire user guide for the BrailleNote Touch is available on the unit.

Press space+O for Options.
Press U followed by Enter to open the User Guide; KeyWeb will open.
Press space+T until you get to the Links navigation level.
Press space+dots4,6 to move to the ``Access the Table of contents" link; then press Enter.

You can now use space+dots4,6 to navigate through the table of contents. When you reach a section that you wish to read, press Enter. You can now press space+G to read through the text. When you wish to stop reading, press space+G to stop.
When done reading the user guide, Press the Home button to return to the main menu.

\section{Appendix C Writing Computer Braille}

The BrailleNote Touch uses the eight-dot computer Braille Table. Computer Braille is required when entering Email addresses, entering dates in the calendar, when entering passwords, and when using the calculator. When computer Braille is required, a dot8 will appear in the edit box. Press space+G to switch to computer Braille if it is required in other places. Return to literary Braille by pressing dot7-chord, (Backspace+space+G).

When writing computer, words or character strings must be written without any contractions. To capitalize a letter, add dot7, (the Backspace key).
Numbers are written as dropped Nemeth numbers without the number sign.
The At sign is created by pressing dots4,7.
The period is created using dots4,6.

\subsection{Punctuation}

exclamation mark: ``!" 2,3,4,6
quote: ``"" 5
pound: ``$\#$ " 3,4,5,6
dollar sign: ``$\$$ " 1,2,4,6
percent: ``$\%$ " 1,4,6
ampersand: ``" 1,2,3,4,6
apostrophe: ``'" 3
left paren: ``(" 1,2,3,5,6
right paren: ``)" 2,3,4,5,6
asterisk: ``$\ast$ " 1,6
plus sign: ``+" 3,4,6
comma: ``," 6
dash: `` " 3,6
period: ``." 4,6
forward slash: ``/" 3,4
colon: ``:" 1,5,6
semi colon: ``;" 5,6
less than: ``<" 1,2,6
equals: ``=" 1,2,3,4,5,6
greater than: ``>" 3,4,5
question mark: ``?" 1,4,5,6
at symbol: ``@" 4,7
left square bracket: ``[" 2,4,6,7
back slash: ``$\textbackslash$ " 1,2,5,6,7
right square bracket: ``]" 1,2,4,5,6,7
carat sign: ``$ \string^ $ " 4,5,7
underscore: ``\_" 4,5,6
grave accent: ```" 4
left curly bracket: ``$ \{ $ " 2,4,6
vertical bar: ``$ \vert $ " 1,2,5,6
right curly bracket: ``$ \} $ " 1,2,4,5,6
tilde: ``$ \sim $ " 4,5

\subsection{Numbers}
These are written without a numeric indicator as they are Nemeth numbers
``0": 3,5,6
``1": 2
``2": 2,3
``3": 2,5
``4": 2,5,6
``5": 2,6
``6": 2,3,5
``7": 2,3,5,6
``8": 2,3,6
``9": 3,5
\subsection{Capital letters}
``A": 1,7
``B": 1,2,7
``C": 1,4,7
``D": 1,4,5,7
``E": 1,5,7
``F": 1,2,4,7
``G": 1,2,4,5,7
``H": 1,2,5,7
``I": 2,4,7
``J": 2,4,5,7
``K": 1,3,7
``L": 1,2,3,7
``M": 1,3,4,7
``N": 1,3,4,5,7
``O": 1,3,5,7
``P": 1,2,3,4,7
``Q": 1,2,3,4,5,7
``R": 1,2,3,5,7
``S": 2,3,4,7
``T": 2,3,4,5,7
``U": 1,3,6,7
``V": 1,2,3,6,7
``W": 2,4,5,6,7
``X": 1,3,4,6,7
``Y": 1,3,4,5,6,7
``Z": 1,3,5,6,7
\subsection{Letters Lower Case}
``a": 1
``b": 1,2
``c": 1,4
``d": 1,4,5
``e": 1,5
``f": 1,2,4
``g": 1,2,4,5
``h": 1,2,5
``i": 2,4
``j": 2,4,5
``k": 1,3
``l": 1,2,3
``m": 1,3,4
``n": 1,3,4,5
``o": 1,3,5
``p": 1,2,3,4
``q": 1,2,3,4,5
``r": 1,2,3,5
``s": 2,3,4
``t": 2,3,4,5
``u": 1,3,6
``v": 1,2,3,6
``w": 2,4,5,6
``x": 1,3,4,6
``y": 1,3,4,5,6
``z": 1,3,5,6
\part{JAWS and NDVA on Windows 10}
\part{VoiceOver on iOS and MacOS}
\part{Refreshable Braille Display with Screen Reader}
\end{document}
